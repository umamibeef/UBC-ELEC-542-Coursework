\documentclass[10pt, oneside, letterpaper]{article}
\usepackage[margin=1in]{geometry}
\usepackage[english]{babel}
\usepackage[utf8]{inputenc}
\usepackage{xcolor}
\definecolor{mygreen}{rgb}{0,0.6,0}
\definecolor{mygray}{rgb}{0.5,0.5,0.5}
\definecolor{mymauve}{rgb}{0.58,0,0.82}
\usepackage{listings}
\lstset{
  backgroundcolor=\color{white}, % choose the background color
  basicstyle=\footnotesize\ttfamily, % size of fonts used for the code
  breaklines=true, % automatic line breaking only at whitespace
  frame=single, % add a frame
  captionpos=b, % sets the caption-position to bottom
  commentstyle=\color{mygreen}, % comment style
  escapeinside={\%*}{*)}, % if you want to add LaTeX within your code
  keywordstyle=\color{blue}, % keyword style
  stringstyle=\color{mymauve}, % string literal style
}
\usepackage{enumitem}
\usepackage{blindtext}
\usepackage{datetime2}
\usepackage{fancyhdr}
\usepackage{amsmath}
  \newcommand{\angstrom}{\textup{\AA}} % for units of angstrom
\usepackage{arydshln} % dash line package for matrices
\usepackage{mathtools} % for things like \Aboxed
\usepackage{float}
\usepackage{pgf}
\usepackage{enumitem} % to easily change style of counters in item lists
\usepackage{xurl} % to easily insert URLs in the LaTeX source
\usepackage{braket} % for bra-ket notation
\usepackage{bm} % for bold vector variables
\usepackage{cases} % for piecewise definitions
\usepackage[makeroom]{cancel} % for crossing out terms
\usepackage{graphicx} % for \scalebox
  \newcommand\scalemath[2]{\scalebox{#1}{\mbox{\ensuremath{\displaystyle #2}}}}

\setcounter{MaxMatrixCols}{32} % increase the maximum number of matrix columns

\title{Assignment 3}
\author{Electronic Structures of the Helium atom and Hydrogen Molecule}
\date{Due: 2022/02/25}

\pagestyle{fancy}
\setlength{\headheight}{23pt}
\setlength{\parskip}{1em}
\fancyhf{}
\chead{Assignment 3}
\rhead{Michel Kakulphimp \\ Student \#63542880}
\lhead{ELEC542 \\ UBC MEng}
\cfoot{\thepage}

\begin{document}
\maketitle
\thispagestyle{fancy}

\section{Directions}

\begin{enumerate}[label=(\alph*)]
  \item Test item
\end{enumerate}

\section{Derivation of Solution}

\newpage
\section{Program Implementation}


\newpage
\section{Results}

\begin{figure}[H]
  \begin{center}
    % \includegraphics[scale=0.75]{he_N28_l5.png}
  \end{center}
  \caption{Calculated orbitals for $He$ atom using limits of [-5,5]}
  \label{he-plot}
\end{figure}

\begin{table}[H]
\begin{center}
\begin{tabular}{l|llllll}\hline
$n$    & $1$    & $2$     & $3$     & $4$      & $5$      & $6$      \\\hline
$E_n$  & $-0.049968$  & $-0.049878$  & $-0.041970$  & $0.011558$  & $-0.007816$  & $0.065703$ \\\hline
\end{tabular}
\end{center}
  \caption{The first six orbital energy levels obtained from applying HF to the Helium atom using limits of [-5,5]}
  \label{orbital-energies-he}
\end{table}

\newpage
\section{Discussion}

\begin{itemize}
    \item 
\end{itemize}

\newpage
\section{Code Listings and Data}

\subsection{Python Code Listing}
\label{code-listing-python}
The following is the code written in Python to generate the solutions and plots used in this report.
% \lstinputlisting[language=Python]{exact-hartree-fock-sim.py}

\newpage
\section{References}

These aren't citing anything, but they were useful in helping me figure out this assignment.

\begin{itemize}
  \item \url{https://www.12000.org/my_notes/mma_matlab_control/KERNEL/KEse83.htm} % laplacian matrix derivation
  \item \url{https://www.value-at-risk.net/numerical-integration-multiple-dimensions/} % numerical integration
  \item \url{https://chem.libretexts.org/Bookshelves/Physical_and_Theoretical_Chemistry_Textbook_Maps/Book%3A_Quantum_States_of_Atoms_and_Molecules_(Zielinksi_et_al)/09%3A_The_Electronic_States_of_the_Multielectron_Atoms/9.07%3A_The_Self-Consistent_Field_Approximation_(Hartree-Fock_Method)} % good overview of HF
  \item \url{http://vergil.chemistry.gatech.edu/notes/hf-intro/hf-intro.html} % another good overview of HF
  \item \url{https://chem.libretexts.org/Bookshelves/Physical_and_Theoretical_Chemistry_Textbook_Maps/Physical_Chemistry_(LibreTexts)/08%3A_Multielectron_Atoms/8.07%3A_Hartree-Fock_Calculations_Give_Good_Agreement_with_Experimental_Data} % energy extraction

  % Other not relevant
  % \item https://physics.stackexchange.com/questions/20703/why-does-iteratively-solving-the-hartree-fock-equations-result-in-convergence
  % \item https://scicomp.stackexchange.com/questions/1297/why-does-iteratively-solving-the-hartree-fock-equations-result-in-convergence
  % \item https://nznano.blogspot.com/2018/03/simple-quantum-chemistry-hartree-fock.html % basis
  % \item https://github.com/CrawfordGroup/ProgrammingProjects
  % \item https://github.com/aromanro/HartreeFock
  % \item https://github.com/ipudu/SCFpy % solves using a basis set, can't use
  % \item https://medium.com/analytics-vidhya/practical-introduction-to-hartree-fock-448fc64c107b % solves using a basis set, can't use
  % \item https://adambaskerville.github.io/posts/HartreeFockGuide/ % relies on Roothaan basis functions, can't use
\end{itemize}

\end{document}

