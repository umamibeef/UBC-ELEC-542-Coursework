\documentclass[10pt, oneside, letterpaper]{article}
\usepackage[margin=1in]{geometry}
\usepackage[english]{babel}
\usepackage[utf8]{inputenc}
\usepackage{xcolor}
\definecolor{mygreen}{rgb}{0,0.6,0}
\definecolor{mygray}{rgb}{0.5,0.5,0.5}
\definecolor{mymauve}{rgb}{0.58,0,0.82}
\usepackage{listings}
\lstset{
  backgroundcolor=\color{white}, % choose the background color
  basicstyle=\footnotesize\ttfamily, % size of fonts used for the code
  breaklines=true, % automatic line breaking only at whitespace
  frame=single, % add a frame
  captionpos=b, % sets the caption-position to bottom
  commentstyle=\color{mygreen}, % comment style
  escapeinside={\%*}{*)}, % if you want to add LaTeX within your code
  keywordstyle=\color{blue}, % keyword style
  stringstyle=\color{mymauve}, % string literal style
}
\usepackage{enumitem}
\usepackage{blindtext}
\usepackage{datetime2}
\usepackage{fancyhdr}
\usepackage{amsmath}
\usepackage{arydshln} % dash line package for matrices
\usepackage{mathtools} % for things like \Aboxed
\usepackage{float}
\usepackage{pgf}
\usepackage{enumitem} % to easily change style of counters in item lists
\usepackage{xurl} % to easily insert URLs in the LaTeX source
\usepackage{braket} % for bra-ket notation
\usepackage{bm} % for bold vector variables
\usepackage{cases} % for piecewise definitions
\usepackage[makeroom]{cancel} % for crossing out terms
\usepackage{graphicx} % for \scalebox
  \newcommand\scalemath[2]{\scalebox{#1}{\mbox{\ensuremath{\displaystyle #2}}}}

\setcounter{MaxMatrixCols}{32} % increase the maximum number of matrix columns

\title{Assignment 1}
\author{Introduction: Infinite Potential Well}
\date{Due: 2022/01/28}

\pagestyle{fancy}
\setlength{\headheight}{23pt}
\setlength{\parskip}{1em}
\fancyhf{}
\chead{Assignment 1}
\rhead{Michel Kakulphimp \\ Student \#63542880}
\lhead{ELEC542 \\ UBC MEng}
\cfoot{\thepage}

\begin{document}
\maketitle
\thispagestyle{fancy}

\section{Directions}

Neglect spin in this problem.

\begin{enumerate}[label=(\alph*)]
  \item Consider one electron in a one-dimensional, infinitely-deep potential well, with a width of 10 in atomic units. Find the electron wavefunctions and allowed energy levels by analytically solving the Schrödinger equation. Plot the first six wave functions and give their associated eigenvalues.
  \item Same as (a), but this time solve the equation numerically using a language of your choice (Python, Matlab, Maple, Mathematica, Basic, C, Pascal, Fortran, assembly, machine code, etc.). Discretize the width of the potential well into at least 10 segments. Finite difference is preferred for solving the equation numerically.
  \item Same as (b), but this time assume that there are two non-interacting electrons in the well. Note that in this case the wave function will have two variables (the positions of the two electrons).
  \item Same as (c), but this time include the Coulomb interaction between the two electrons. Comment on the solutions and their difference with those obtained in part (c).
\end{enumerate}

\section{Solution}

For this assignment, we need to find the solutions of the following one-dimensional, time-independent Schr\"{o}dinger's equation:

\begin{align*}
  \hat{H}\Phi_n(x_i) &= {E_n}\Phi_n(x_i)
\end{align*}

For this assignment, $\hat{H}$ is the Hamiltonian operator for a system of electrons described by their position $x_{i}$, $E_n$ is the energy eigenvalue, and $\Phi_n(x_i)$ is the wavefunction that we want to solve for. Note that we will have multiple solutions and we are tasked to obtain the first six. Nuclei are ignored for this problem as only electrons and their interactions are involed, so their contribution is not included in the operator. The expanded form of this Hamiltonian operator for N electrons is as follows (in Hartree atomic units where $\hbar = m_e = e^2 = 1$):

\begin{align*}
  \hat{H} &= \hat{T} + \hat{U} + \hat{V} & \text{total energy of the system}\\
  \hat{H} &= -\sum_{i=1}^{N}\frac{1}{2}\nabla^{2}_i + \sum_{i=1}^{N}\sum_{j>i}^{N}\frac{1}{x_{ij}} + \sum_{i=1}^{N}v(x_i)\\
  \hat{T} &= -\sum_{i=1}^{N}\frac{1}{2}\nabla^{2}_i & \text{kinetic energy of each electron}\\
  \hat{U} &= \sum_{i=1}^{N}\sum_{j>i}^{N}\frac{1}{x_{ij}} & \text{Coulomb repulsion between electrons}\\
  \hat{V} &= \sum_{i=1}^{N}v(x_i) & \text{potential energy of each electron}
\end{align*}

Our one-dimensional, infinitely-deep potential well with a width of 10 atomic units gives us the following boundary conditions for each electron's potential energy $v(x_i)$:

\begin{align*}
  v(x_i) &= \begin{cases}
          0       \quad \, & |x| < 5a \\
          \infty  \quad \, & |x| > 5a \\
     \end{cases}
\end{align*}

\subsection{Analytical Solution: Single Electron}

For the first two steps, we will analytically solve the Schr\"{o}dinger equation for a single electron bounded within the 10 atomic unit infinite well to obtain the first six wavefunctions and the associated eigenvalues. For a single electron, the Hamiltonian shown above is simplied as follows:

\begin{align*}
  \hat{H} &= -\frac{1}{2}\nabla^{2}_i + v(x_i)\\
\end{align*}

We only have one electron, so the summations are dropped and there is no Coulomb interaction to consider. We can then plug this new Hamiltonian into Schr\"{o}dinger's equation:

\begin{align*}
  \hat{H}\Phi_n(x) &= {E_n}\Phi_n(x) \\
  -\frac{1}{2}\Phi^{''}_n(x) + v(x)\Phi_n(x) &= {E_n}\Phi_n(x)
\end{align*}

With the following boundary conditions:

\begin{align*}
  v(x_i) &= \begin{cases}
          0       \quad \, & |x| < 5a \\
          \infty  \quad \, & |x| > 5a \\
     \end{cases}
\end{align*}

We know that the wave function will have the following value at the boundaries of the well (the walls):

\begin{align*}
  \Phi_n(\pm5a) = 0 
\end{align*}

Which leaves us with the following equation to solve for within the well:

\begin{align*}
  -\frac{1}{2}\Phi^{''}_n(x) &= {E_n}\Phi_n(x)
\end{align*}

This is a second order linear differential equation. The general solution to this form of the second order linear differential equation is as follows:

\begin{align*}
  \Phi_n(x) = Ae^{ikx} + Be^{-ikx}
\end{align*}

Plugging in our boundary conditions, we get the following relationships:

\begin{align*}
  \Phi_n(-5a) &= Ae^{-ik5a} + Be^{ik5a} = 0 \\
  \Phi_n(5a) &= Ae^{ik5a} + Be^{-ik5a} = 0
\end{align*}

With some algebra, we can then get the following relationships:

\begin{align*}
  A + Be^{2ik5a} &= 0 \Longrightarrow A = -Be^{i10ka} \\
  A + Be^{-2ik5a} &= 0 \Longrightarrow A = -Be^{-i10ka} \\
  -Be^{i10ka} &= -Be^{-i10ka} \\
  e^{i10ka} &= e^{-i10ka} \\
  e^{i20ka} &= 1
\end{align*}

Using Euler's identity, we know that:

\begin{align*}
  e^{i2\pi n} &= 1
\end{align*}

Therefore our possible values for k are as follows:

\begin{align*}
  2\pi n &= 20ka \\
  k &= \frac{\pi n}{10a}
\end{align*}

Where $n$ is an integer. With this result, we can progress further into defining the wave function as follows:

\begin{align*}
  A &= -Be^{i10ka} \\
  A &= -Be^{i\pi n} \\
  A &= -B(e^{i\pi})^n \\
  A &= -B(-1)^n \\
  \Aboxed{A &= -B} & \text{for even n} \\
  \Aboxed{A &= B} & \text{for odd n} \\
  \Phi_n(x) &= Ae^{-ikx} - Ae^{ikx} = 2iA\sin(kx) \\
  \Aboxed{\Phi_n(x) &= 2iA\sin(\frac{\pi nx}{10a})} &\text{for even n} \\
  \Phi_n(x) &= Ae^{-ikx} + Ae^{ikx} = 2A\cos(kx) \\
  \Aboxed{\Phi_n(x) &= 2A\cos(\frac{\pi nx}{10a})} &\text{for odd n}
\end{align*}

We are left with finding the value for the constant $A$. We can normalize the wavefunction to calculate this value. This represents the fact that the probability of finding the wave function in all space is equal to 1.

\begin{align*}
  1 = \int_{-\infty}^{\infty}\left|\Phi_n(x)\right|^2dx
\end{align*}

Since the problem defines the bounds of the wavefunction to within an infinitely-deep potential well of a fixed size, we can use the bounds of the box as the integration limits.

\begin{align*}
  1 = \int_{-5a}^{5a}\left|\Phi_n(x)\right|^2dx
\end{align*}

We can now evaluate the integral for both even and odd cases as follows.

For the even case:

\begin{align*}
  1 &= \int_{-5a}^{5a}\left|\Phi_n(x)\right|^2dx \\
  1 &= 4\left|A\right|^2\int_{-5a}^{5a}\sin^2\left(\frac{\pi nx}{10a}\right)dx \\
  1 &= 4\left|A\right|^2\int_{-5a}^{5a}\frac{1}{2}\left[1-\cos\left(\frac{\pi nx}{5a}\right)\right]dx \\
  1 &= 2\left|A\right|^2\int_{-5a}^{5a}\left[1-\cos\left(\frac{\pi nx}{5a}\right)\right]dx \\
  1 &= 2\left|A\right|^2\left[x\cancel{-\frac{5a}{\pi n}\sin\left(\frac{\pi nx}{5a}\right)}\right]_{-5a}^{5a} \\
  1 &= 2\left|A\right|^2 10a \\
  \left|A\right|^2 &= \frac{1}{20a} \\
  A &= \frac{1}{\sqrt{20a}}, \frac{-i}{\sqrt{20a}} \\
  \Aboxed{\Phi_n(x) &= \frac{2}{\sqrt{20a}}\sin\left(\frac{\pi nx}{10a}\right)} &\text{for even n}
\end{align*}

We choose the imaginary value for A so that we can cancel out the imaginary value in the wave equation to obtain a real result. We can perform the same steps for the odd case as follows:

\begin{align*}
  1 &= \int_{-5a}^{5a}\left|\Phi_n(x)\right|^2dx \\
  1 &= 4\left|A\right|^2\int_{-5a}^{5a}\cos^2\left(\frac{\pi nx}{10a}\right)dx \\
  1 &= 4\left|A\right|^2\int_{-5a}^{5a}\frac{1}{2}\left[1+\cos\left(\frac{\pi nx}{5a}\right)\right]dx \\
  1 &= 2\left|A\right|^2\int_{-5a}^{5a}\left[1+\cos\left(\frac{\pi nx}{5a}\right)\right]dx \\
  1 &= 2\left|A\right|^2\left[x\cancel{+\frac{5a}{\pi n}\sin\left(\frac{\pi nx}{5a}\right)}\right]_{-5a}^{5a} \\
  1 &= 2\left|A\right|^2 10a \\
  \left|A\right|^2 &= \frac{1}{20a} \\
  A &= \frac{1}{\sqrt{20a}}, \frac{-i}{\sqrt{20a}} \\
  \Aboxed{\Phi_n(x) &= \frac{2}{\sqrt{20a}}\cos\left(\frac{\pi nx}{10a}\right)} &\text{for odd n}
\end{align*}

In this case, we choose the real value for A so that the wave equation remains real. We can now plot the first six wave functions for $n = 1, 2, 3, 4, 5, 6$ as follows:

\begin{figure}[H]
  \begin{center}
    %% Creator: Matplotlib, PGF backend
%%
%% To include the figure in your LaTeX document, write
%%   \input{<filename>.pgf}
%%
%% Make sure the required packages are loaded in your preamble
%%   \usepackage{pgf}
%%
%% Also ensure that all the required font packages are loaded; for instance,
%% the lmodern package is sometimes necessary when using math font.
%%   \usepackage{lmodern}
%%
%% Figures using additional raster images can only be included by \input if
%% they are in the same directory as the main LaTeX file. For loading figures
%% from other directories you can use the `import` package
%%   \usepackage{import}
%%
%% and then include the figures with
%%   \import{<path to file>}{<filename>.pgf}
%%
%% Matplotlib used the following preamble
%%
\begingroup%
\makeatletter%
\begin{pgfpicture}%
\pgfpathrectangle{\pgfpointorigin}{\pgfqpoint{3.000000in}{6.000000in}}%
\pgfusepath{use as bounding box, clip}%
\begin{pgfscope}%
\pgfsetbuttcap%
\pgfsetmiterjoin%
\definecolor{currentfill}{rgb}{1.000000,1.000000,1.000000}%
\pgfsetfillcolor{currentfill}%
\pgfsetlinewidth{0.000000pt}%
\definecolor{currentstroke}{rgb}{1.000000,1.000000,1.000000}%
\pgfsetstrokecolor{currentstroke}%
\pgfsetdash{}{0pt}%
\pgfpathmoveto{\pgfqpoint{0.000000in}{0.000000in}}%
\pgfpathlineto{\pgfqpoint{3.000000in}{0.000000in}}%
\pgfpathlineto{\pgfqpoint{3.000000in}{6.000000in}}%
\pgfpathlineto{\pgfqpoint{0.000000in}{6.000000in}}%
\pgfpathlineto{\pgfqpoint{0.000000in}{0.000000in}}%
\pgfpathclose%
\pgfusepath{fill}%
\end{pgfscope}%
\begin{pgfscope}%
\pgfsetbuttcap%
\pgfsetmiterjoin%
\definecolor{currentfill}{rgb}{1.000000,1.000000,1.000000}%
\pgfsetfillcolor{currentfill}%
\pgfsetlinewidth{0.000000pt}%
\definecolor{currentstroke}{rgb}{0.000000,0.000000,0.000000}%
\pgfsetstrokecolor{currentstroke}%
\pgfsetstrokeopacity{0.000000}%
\pgfsetdash{}{0pt}%
\pgfpathmoveto{\pgfqpoint{0.495679in}{0.549691in}}%
\pgfpathlineto{\pgfqpoint{2.850000in}{0.549691in}}%
\pgfpathlineto{\pgfqpoint{2.850000in}{5.641667in}}%
\pgfpathlineto{\pgfqpoint{0.495679in}{5.641667in}}%
\pgfpathlineto{\pgfqpoint{0.495679in}{0.549691in}}%
\pgfpathclose%
\pgfusepath{fill}%
\end{pgfscope}%
\begin{pgfscope}%
\pgfsetbuttcap%
\pgfsetroundjoin%
\definecolor{currentfill}{rgb}{0.000000,0.000000,0.000000}%
\pgfsetfillcolor{currentfill}%
\pgfsetlinewidth{0.803000pt}%
\definecolor{currentstroke}{rgb}{0.000000,0.000000,0.000000}%
\pgfsetstrokecolor{currentstroke}%
\pgfsetdash{}{0pt}%
\pgfsys@defobject{currentmarker}{\pgfqpoint{0.000000in}{-0.048611in}}{\pgfqpoint{0.000000in}{0.000000in}}{%
\pgfpathmoveto{\pgfqpoint{0.000000in}{0.000000in}}%
\pgfpathlineto{\pgfqpoint{0.000000in}{-0.048611in}}%
\pgfusepath{stroke,fill}%
}%
\begin{pgfscope}%
\pgfsys@transformshift{0.731111in}{0.549691in}%
\pgfsys@useobject{currentmarker}{}%
\end{pgfscope}%
\end{pgfscope}%
\begin{pgfscope}%
\definecolor{textcolor}{rgb}{0.000000,0.000000,0.000000}%
\pgfsetstrokecolor{textcolor}%
\pgfsetfillcolor{textcolor}%
\pgftext[x=0.731111in,y=0.452469in,,top]{\color{textcolor}\rmfamily\fontsize{10.000000}{12.000000}\selectfont \(\displaystyle {\ensuremath{-}4}\)}%
\end{pgfscope}%
\begin{pgfscope}%
\pgfsetbuttcap%
\pgfsetroundjoin%
\definecolor{currentfill}{rgb}{0.000000,0.000000,0.000000}%
\pgfsetfillcolor{currentfill}%
\pgfsetlinewidth{0.803000pt}%
\definecolor{currentstroke}{rgb}{0.000000,0.000000,0.000000}%
\pgfsetstrokecolor{currentstroke}%
\pgfsetdash{}{0pt}%
\pgfsys@defobject{currentmarker}{\pgfqpoint{0.000000in}{-0.048611in}}{\pgfqpoint{0.000000in}{0.000000in}}{%
\pgfpathmoveto{\pgfqpoint{0.000000in}{0.000000in}}%
\pgfpathlineto{\pgfqpoint{0.000000in}{-0.048611in}}%
\pgfusepath{stroke,fill}%
}%
\begin{pgfscope}%
\pgfsys@transformshift{1.201975in}{0.549691in}%
\pgfsys@useobject{currentmarker}{}%
\end{pgfscope}%
\end{pgfscope}%
\begin{pgfscope}%
\definecolor{textcolor}{rgb}{0.000000,0.000000,0.000000}%
\pgfsetstrokecolor{textcolor}%
\pgfsetfillcolor{textcolor}%
\pgftext[x=1.201975in,y=0.452469in,,top]{\color{textcolor}\rmfamily\fontsize{10.000000}{12.000000}\selectfont \(\displaystyle {\ensuremath{-}2}\)}%
\end{pgfscope}%
\begin{pgfscope}%
\pgfsetbuttcap%
\pgfsetroundjoin%
\definecolor{currentfill}{rgb}{0.000000,0.000000,0.000000}%
\pgfsetfillcolor{currentfill}%
\pgfsetlinewidth{0.803000pt}%
\definecolor{currentstroke}{rgb}{0.000000,0.000000,0.000000}%
\pgfsetstrokecolor{currentstroke}%
\pgfsetdash{}{0pt}%
\pgfsys@defobject{currentmarker}{\pgfqpoint{0.000000in}{-0.048611in}}{\pgfqpoint{0.000000in}{0.000000in}}{%
\pgfpathmoveto{\pgfqpoint{0.000000in}{0.000000in}}%
\pgfpathlineto{\pgfqpoint{0.000000in}{-0.048611in}}%
\pgfusepath{stroke,fill}%
}%
\begin{pgfscope}%
\pgfsys@transformshift{1.672840in}{0.549691in}%
\pgfsys@useobject{currentmarker}{}%
\end{pgfscope}%
\end{pgfscope}%
\begin{pgfscope}%
\definecolor{textcolor}{rgb}{0.000000,0.000000,0.000000}%
\pgfsetstrokecolor{textcolor}%
\pgfsetfillcolor{textcolor}%
\pgftext[x=1.672840in,y=0.452469in,,top]{\color{textcolor}\rmfamily\fontsize{10.000000}{12.000000}\selectfont \(\displaystyle {0}\)}%
\end{pgfscope}%
\begin{pgfscope}%
\pgfsetbuttcap%
\pgfsetroundjoin%
\definecolor{currentfill}{rgb}{0.000000,0.000000,0.000000}%
\pgfsetfillcolor{currentfill}%
\pgfsetlinewidth{0.803000pt}%
\definecolor{currentstroke}{rgb}{0.000000,0.000000,0.000000}%
\pgfsetstrokecolor{currentstroke}%
\pgfsetdash{}{0pt}%
\pgfsys@defobject{currentmarker}{\pgfqpoint{0.000000in}{-0.048611in}}{\pgfqpoint{0.000000in}{0.000000in}}{%
\pgfpathmoveto{\pgfqpoint{0.000000in}{0.000000in}}%
\pgfpathlineto{\pgfqpoint{0.000000in}{-0.048611in}}%
\pgfusepath{stroke,fill}%
}%
\begin{pgfscope}%
\pgfsys@transformshift{2.143704in}{0.549691in}%
\pgfsys@useobject{currentmarker}{}%
\end{pgfscope}%
\end{pgfscope}%
\begin{pgfscope}%
\definecolor{textcolor}{rgb}{0.000000,0.000000,0.000000}%
\pgfsetstrokecolor{textcolor}%
\pgfsetfillcolor{textcolor}%
\pgftext[x=2.143704in,y=0.452469in,,top]{\color{textcolor}\rmfamily\fontsize{10.000000}{12.000000}\selectfont \(\displaystyle {2}\)}%
\end{pgfscope}%
\begin{pgfscope}%
\pgfsetbuttcap%
\pgfsetroundjoin%
\definecolor{currentfill}{rgb}{0.000000,0.000000,0.000000}%
\pgfsetfillcolor{currentfill}%
\pgfsetlinewidth{0.803000pt}%
\definecolor{currentstroke}{rgb}{0.000000,0.000000,0.000000}%
\pgfsetstrokecolor{currentstroke}%
\pgfsetdash{}{0pt}%
\pgfsys@defobject{currentmarker}{\pgfqpoint{0.000000in}{-0.048611in}}{\pgfqpoint{0.000000in}{0.000000in}}{%
\pgfpathmoveto{\pgfqpoint{0.000000in}{0.000000in}}%
\pgfpathlineto{\pgfqpoint{0.000000in}{-0.048611in}}%
\pgfusepath{stroke,fill}%
}%
\begin{pgfscope}%
\pgfsys@transformshift{2.614568in}{0.549691in}%
\pgfsys@useobject{currentmarker}{}%
\end{pgfscope}%
\end{pgfscope}%
\begin{pgfscope}%
\definecolor{textcolor}{rgb}{0.000000,0.000000,0.000000}%
\pgfsetstrokecolor{textcolor}%
\pgfsetfillcolor{textcolor}%
\pgftext[x=2.614568in,y=0.452469in,,top]{\color{textcolor}\rmfamily\fontsize{10.000000}{12.000000}\selectfont \(\displaystyle {4}\)}%
\end{pgfscope}%
\begin{pgfscope}%
\definecolor{textcolor}{rgb}{0.000000,0.000000,0.000000}%
\pgfsetstrokecolor{textcolor}%
\pgfsetfillcolor{textcolor}%
\pgftext[x=1.672840in,y=0.273457in,,top]{\color{textcolor}\rmfamily\fontsize{10.000000}{12.000000}\selectfont \(\displaystyle x\)}%
\end{pgfscope}%
\begin{pgfscope}%
\pgfsetbuttcap%
\pgfsetroundjoin%
\definecolor{currentfill}{rgb}{0.000000,0.000000,0.000000}%
\pgfsetfillcolor{currentfill}%
\pgfsetlinewidth{0.803000pt}%
\definecolor{currentstroke}{rgb}{0.000000,0.000000,0.000000}%
\pgfsetstrokecolor{currentstroke}%
\pgfsetdash{}{0pt}%
\pgfsys@defobject{currentmarker}{\pgfqpoint{-0.048611in}{0.000000in}}{\pgfqpoint{-0.000000in}{0.000000in}}{%
\pgfpathmoveto{\pgfqpoint{-0.000000in}{0.000000in}}%
\pgfpathlineto{\pgfqpoint{-0.048611in}{0.000000in}}%
\pgfusepath{stroke,fill}%
}%
\begin{pgfscope}%
\pgfsys@transformshift{0.495679in}{0.974022in}%
\pgfsys@useobject{currentmarker}{}%
\end{pgfscope}%
\end{pgfscope}%
\begin{pgfscope}%
\definecolor{textcolor}{rgb}{0.000000,0.000000,0.000000}%
\pgfsetstrokecolor{textcolor}%
\pgfsetfillcolor{textcolor}%
\pgftext[x=0.329012in, y=0.925797in, left, base]{\color{textcolor}\rmfamily\fontsize{10.000000}{12.000000}\selectfont \(\displaystyle {1}\)}%
\end{pgfscope}%
\begin{pgfscope}%
\pgfsetbuttcap%
\pgfsetroundjoin%
\definecolor{currentfill}{rgb}{0.000000,0.000000,0.000000}%
\pgfsetfillcolor{currentfill}%
\pgfsetlinewidth{0.803000pt}%
\definecolor{currentstroke}{rgb}{0.000000,0.000000,0.000000}%
\pgfsetstrokecolor{currentstroke}%
\pgfsetdash{}{0pt}%
\pgfsys@defobject{currentmarker}{\pgfqpoint{-0.048611in}{0.000000in}}{\pgfqpoint{-0.000000in}{0.000000in}}{%
\pgfpathmoveto{\pgfqpoint{-0.000000in}{0.000000in}}%
\pgfpathlineto{\pgfqpoint{-0.048611in}{0.000000in}}%
\pgfusepath{stroke,fill}%
}%
\begin{pgfscope}%
\pgfsys@transformshift{0.495679in}{1.822685in}%
\pgfsys@useobject{currentmarker}{}%
\end{pgfscope}%
\end{pgfscope}%
\begin{pgfscope}%
\definecolor{textcolor}{rgb}{0.000000,0.000000,0.000000}%
\pgfsetstrokecolor{textcolor}%
\pgfsetfillcolor{textcolor}%
\pgftext[x=0.329012in, y=1.774460in, left, base]{\color{textcolor}\rmfamily\fontsize{10.000000}{12.000000}\selectfont \(\displaystyle {2}\)}%
\end{pgfscope}%
\begin{pgfscope}%
\pgfsetbuttcap%
\pgfsetroundjoin%
\definecolor{currentfill}{rgb}{0.000000,0.000000,0.000000}%
\pgfsetfillcolor{currentfill}%
\pgfsetlinewidth{0.803000pt}%
\definecolor{currentstroke}{rgb}{0.000000,0.000000,0.000000}%
\pgfsetstrokecolor{currentstroke}%
\pgfsetdash{}{0pt}%
\pgfsys@defobject{currentmarker}{\pgfqpoint{-0.048611in}{0.000000in}}{\pgfqpoint{-0.000000in}{0.000000in}}{%
\pgfpathmoveto{\pgfqpoint{-0.000000in}{0.000000in}}%
\pgfpathlineto{\pgfqpoint{-0.048611in}{0.000000in}}%
\pgfusepath{stroke,fill}%
}%
\begin{pgfscope}%
\pgfsys@transformshift{0.495679in}{2.671348in}%
\pgfsys@useobject{currentmarker}{}%
\end{pgfscope}%
\end{pgfscope}%
\begin{pgfscope}%
\definecolor{textcolor}{rgb}{0.000000,0.000000,0.000000}%
\pgfsetstrokecolor{textcolor}%
\pgfsetfillcolor{textcolor}%
\pgftext[x=0.329012in, y=2.623122in, left, base]{\color{textcolor}\rmfamily\fontsize{10.000000}{12.000000}\selectfont \(\displaystyle {3}\)}%
\end{pgfscope}%
\begin{pgfscope}%
\pgfsetbuttcap%
\pgfsetroundjoin%
\definecolor{currentfill}{rgb}{0.000000,0.000000,0.000000}%
\pgfsetfillcolor{currentfill}%
\pgfsetlinewidth{0.803000pt}%
\definecolor{currentstroke}{rgb}{0.000000,0.000000,0.000000}%
\pgfsetstrokecolor{currentstroke}%
\pgfsetdash{}{0pt}%
\pgfsys@defobject{currentmarker}{\pgfqpoint{-0.048611in}{0.000000in}}{\pgfqpoint{-0.000000in}{0.000000in}}{%
\pgfpathmoveto{\pgfqpoint{-0.000000in}{0.000000in}}%
\pgfpathlineto{\pgfqpoint{-0.048611in}{0.000000in}}%
\pgfusepath{stroke,fill}%
}%
\begin{pgfscope}%
\pgfsys@transformshift{0.495679in}{3.520010in}%
\pgfsys@useobject{currentmarker}{}%
\end{pgfscope}%
\end{pgfscope}%
\begin{pgfscope}%
\definecolor{textcolor}{rgb}{0.000000,0.000000,0.000000}%
\pgfsetstrokecolor{textcolor}%
\pgfsetfillcolor{textcolor}%
\pgftext[x=0.329012in, y=3.471785in, left, base]{\color{textcolor}\rmfamily\fontsize{10.000000}{12.000000}\selectfont \(\displaystyle {4}\)}%
\end{pgfscope}%
\begin{pgfscope}%
\pgfsetbuttcap%
\pgfsetroundjoin%
\definecolor{currentfill}{rgb}{0.000000,0.000000,0.000000}%
\pgfsetfillcolor{currentfill}%
\pgfsetlinewidth{0.803000pt}%
\definecolor{currentstroke}{rgb}{0.000000,0.000000,0.000000}%
\pgfsetstrokecolor{currentstroke}%
\pgfsetdash{}{0pt}%
\pgfsys@defobject{currentmarker}{\pgfqpoint{-0.048611in}{0.000000in}}{\pgfqpoint{-0.000000in}{0.000000in}}{%
\pgfpathmoveto{\pgfqpoint{-0.000000in}{0.000000in}}%
\pgfpathlineto{\pgfqpoint{-0.048611in}{0.000000in}}%
\pgfusepath{stroke,fill}%
}%
\begin{pgfscope}%
\pgfsys@transformshift{0.495679in}{4.368673in}%
\pgfsys@useobject{currentmarker}{}%
\end{pgfscope}%
\end{pgfscope}%
\begin{pgfscope}%
\definecolor{textcolor}{rgb}{0.000000,0.000000,0.000000}%
\pgfsetstrokecolor{textcolor}%
\pgfsetfillcolor{textcolor}%
\pgftext[x=0.329012in, y=4.320448in, left, base]{\color{textcolor}\rmfamily\fontsize{10.000000}{12.000000}\selectfont \(\displaystyle {5}\)}%
\end{pgfscope}%
\begin{pgfscope}%
\pgfsetbuttcap%
\pgfsetroundjoin%
\definecolor{currentfill}{rgb}{0.000000,0.000000,0.000000}%
\pgfsetfillcolor{currentfill}%
\pgfsetlinewidth{0.803000pt}%
\definecolor{currentstroke}{rgb}{0.000000,0.000000,0.000000}%
\pgfsetstrokecolor{currentstroke}%
\pgfsetdash{}{0pt}%
\pgfsys@defobject{currentmarker}{\pgfqpoint{-0.048611in}{0.000000in}}{\pgfqpoint{-0.000000in}{0.000000in}}{%
\pgfpathmoveto{\pgfqpoint{-0.000000in}{0.000000in}}%
\pgfpathlineto{\pgfqpoint{-0.048611in}{0.000000in}}%
\pgfusepath{stroke,fill}%
}%
\begin{pgfscope}%
\pgfsys@transformshift{0.495679in}{5.217335in}%
\pgfsys@useobject{currentmarker}{}%
\end{pgfscope}%
\end{pgfscope}%
\begin{pgfscope}%
\definecolor{textcolor}{rgb}{0.000000,0.000000,0.000000}%
\pgfsetstrokecolor{textcolor}%
\pgfsetfillcolor{textcolor}%
\pgftext[x=0.329012in, y=5.169110in, left, base]{\color{textcolor}\rmfamily\fontsize{10.000000}{12.000000}\selectfont \(\displaystyle {6}\)}%
\end{pgfscope}%
\begin{pgfscope}%
\definecolor{textcolor}{rgb}{0.000000,0.000000,0.000000}%
\pgfsetstrokecolor{textcolor}%
\pgfsetfillcolor{textcolor}%
\pgftext[x=0.273457in,y=3.095679in,,bottom,rotate=90.000000]{\color{textcolor}\rmfamily\fontsize{10.000000}{12.000000}\selectfont \(\displaystyle n\)}%
\end{pgfscope}%
\begin{pgfscope}%
\pgfpathrectangle{\pgfqpoint{0.495679in}{0.549691in}}{\pgfqpoint{2.354321in}{5.091976in}}%
\pgfusepath{clip}%
\pgfsetrectcap%
\pgfsetroundjoin%
\pgfsetlinewidth{1.505625pt}%
\definecolor{currentstroke}{rgb}{0.121569,0.466667,0.705882}%
\pgfsetstrokecolor{currentstroke}%
\pgfsetdash{}{0pt}%
\pgfpathmoveto{\pgfqpoint{0.495679in}{0.974022in}}%
\pgfpathlineto{\pgfqpoint{0.695997in}{1.074269in}}%
\pgfpathlineto{\pgfqpoint{0.804404in}{1.125990in}}%
\pgfpathlineto{\pgfqpoint{0.896314in}{1.167395in}}%
\pgfpathlineto{\pgfqpoint{0.978798in}{1.202098in}}%
\pgfpathlineto{\pgfqpoint{1.054212in}{1.231420in}}%
\pgfpathlineto{\pgfqpoint{1.124912in}{1.256549in}}%
\pgfpathlineto{\pgfqpoint{1.193256in}{1.278455in}}%
\pgfpathlineto{\pgfqpoint{1.256886in}{1.296579in}}%
\pgfpathlineto{\pgfqpoint{1.320516in}{1.312379in}}%
\pgfpathlineto{\pgfqpoint{1.381790in}{1.325290in}}%
\pgfpathlineto{\pgfqpoint{1.440707in}{1.335493in}}%
\pgfpathlineto{\pgfqpoint{1.499624in}{1.343463in}}%
\pgfpathlineto{\pgfqpoint{1.558541in}{1.349150in}}%
\pgfpathlineto{\pgfqpoint{1.615101in}{1.352430in}}%
\pgfpathlineto{\pgfqpoint{1.671661in}{1.353555in}}%
\pgfpathlineto{\pgfqpoint{1.728222in}{1.352520in}}%
\pgfpathlineto{\pgfqpoint{1.784782in}{1.349329in}}%
\pgfpathlineto{\pgfqpoint{1.843699in}{1.343734in}}%
\pgfpathlineto{\pgfqpoint{1.902616in}{1.335855in}}%
\pgfpathlineto{\pgfqpoint{1.961533in}{1.325741in}}%
\pgfpathlineto{\pgfqpoint{2.022806in}{1.312918in}}%
\pgfpathlineto{\pgfqpoint{2.084080in}{1.297830in}}%
\pgfpathlineto{\pgfqpoint{2.147710in}{1.279874in}}%
\pgfpathlineto{\pgfqpoint{2.213697in}{1.258928in}}%
\pgfpathlineto{\pgfqpoint{2.282041in}{1.234908in}}%
\pgfpathlineto{\pgfqpoint{2.355098in}{1.206839in}}%
\pgfpathlineto{\pgfqpoint{2.432868in}{1.174537in}}%
\pgfpathlineto{\pgfqpoint{2.517708in}{1.136850in}}%
\pgfpathlineto{\pgfqpoint{2.614332in}{1.091418in}}%
\pgfpathlineto{\pgfqpoint{2.734523in}{1.032274in}}%
\pgfpathlineto{\pgfqpoint{2.850000in}{0.974022in}}%
\pgfpathlineto{\pgfqpoint{2.850000in}{0.974022in}}%
\pgfusepath{stroke}%
\end{pgfscope}%
\begin{pgfscope}%
\pgfpathrectangle{\pgfqpoint{0.495679in}{0.549691in}}{\pgfqpoint{2.354321in}{5.091976in}}%
\pgfusepath{clip}%
\pgfsetbuttcap%
\pgfsetroundjoin%
\pgfsetlinewidth{0.501875pt}%
\definecolor{currentstroke}{rgb}{0.121569,0.466667,0.705882}%
\pgfsetstrokecolor{currentstroke}%
\pgfsetdash{{1.850000pt}{0.800000pt}}{0.000000pt}%
\pgfpathmoveto{\pgfqpoint{0.495679in}{0.974022in}}%
\pgfpathlineto{\pgfqpoint{2.850000in}{0.974022in}}%
\pgfusepath{stroke}%
\end{pgfscope}%
\begin{pgfscope}%
\pgfpathrectangle{\pgfqpoint{0.495679in}{0.549691in}}{\pgfqpoint{2.354321in}{5.091976in}}%
\pgfusepath{clip}%
\pgfsetrectcap%
\pgfsetroundjoin%
\pgfsetlinewidth{1.505625pt}%
\definecolor{currentstroke}{rgb}{1.000000,0.498039,0.054902}%
\pgfsetstrokecolor{currentstroke}%
\pgfsetdash{}{0pt}%
\pgfpathmoveto{\pgfqpoint{0.495679in}{1.822685in}}%
\pgfpathlineto{\pgfqpoint{0.604086in}{1.714405in}}%
\pgfpathlineto{\pgfqpoint{0.663003in}{1.658780in}}%
\pgfpathlineto{\pgfqpoint{0.712493in}{1.615126in}}%
\pgfpathlineto{\pgfqpoint{0.754914in}{1.580559in}}%
\pgfpathlineto{\pgfqpoint{0.794977in}{1.550752in}}%
\pgfpathlineto{\pgfqpoint{0.832684in}{1.525529in}}%
\pgfpathlineto{\pgfqpoint{0.868034in}{1.504609in}}%
\pgfpathlineto{\pgfqpoint{0.901028in}{1.487633in}}%
\pgfpathlineto{\pgfqpoint{0.934021in}{1.473253in}}%
\pgfpathlineto{\pgfqpoint{0.964658in}{1.462322in}}%
\pgfpathlineto{\pgfqpoint{0.995295in}{1.453799in}}%
\pgfpathlineto{\pgfqpoint{1.023575in}{1.448118in}}%
\pgfpathlineto{\pgfqpoint{1.051855in}{1.444570in}}%
\pgfpathlineto{\pgfqpoint{1.080135in}{1.443175in}}%
\pgfpathlineto{\pgfqpoint{1.108415in}{1.443940in}}%
\pgfpathlineto{\pgfqpoint{1.136695in}{1.446862in}}%
\pgfpathlineto{\pgfqpoint{1.164976in}{1.451923in}}%
\pgfpathlineto{\pgfqpoint{1.195612in}{1.459788in}}%
\pgfpathlineto{\pgfqpoint{1.226249in}{1.470077in}}%
\pgfpathlineto{\pgfqpoint{1.256886in}{1.482722in}}%
\pgfpathlineto{\pgfqpoint{1.289879in}{1.498877in}}%
\pgfpathlineto{\pgfqpoint{1.322873in}{1.517541in}}%
\pgfpathlineto{\pgfqpoint{1.358223in}{1.540158in}}%
\pgfpathlineto{\pgfqpoint{1.395930in}{1.567047in}}%
\pgfpathlineto{\pgfqpoint{1.435993in}{1.598444in}}%
\pgfpathlineto{\pgfqpoint{1.480770in}{1.636547in}}%
\pgfpathlineto{\pgfqpoint{1.530261in}{1.681727in}}%
\pgfpathlineto{\pgfqpoint{1.589178in}{1.738646in}}%
\pgfpathlineto{\pgfqpoint{1.681088in}{1.831039in}}%
\pgfpathlineto{\pgfqpoint{1.782425in}{1.932108in}}%
\pgfpathlineto{\pgfqpoint{1.841342in}{1.987666in}}%
\pgfpathlineto{\pgfqpoint{1.890832in}{2.031242in}}%
\pgfpathlineto{\pgfqpoint{1.933252in}{2.065729in}}%
\pgfpathlineto{\pgfqpoint{1.973316in}{2.095450in}}%
\pgfpathlineto{\pgfqpoint{2.011023in}{2.120582in}}%
\pgfpathlineto{\pgfqpoint{2.046373in}{2.141410in}}%
\pgfpathlineto{\pgfqpoint{2.079366in}{2.158296in}}%
\pgfpathlineto{\pgfqpoint{2.112360in}{2.172581in}}%
\pgfpathlineto{\pgfqpoint{2.142997in}{2.183421in}}%
\pgfpathlineto{\pgfqpoint{2.173634in}{2.191850in}}%
\pgfpathlineto{\pgfqpoint{2.201914in}{2.197443in}}%
\pgfpathlineto{\pgfqpoint{2.230194in}{2.200901in}}%
\pgfpathlineto{\pgfqpoint{2.258474in}{2.202207in}}%
\pgfpathlineto{\pgfqpoint{2.286754in}{2.201351in}}%
\pgfpathlineto{\pgfqpoint{2.315034in}{2.198340in}}%
\pgfpathlineto{\pgfqpoint{2.343314in}{2.193190in}}%
\pgfpathlineto{\pgfqpoint{2.373951in}{2.185231in}}%
\pgfpathlineto{\pgfqpoint{2.404588in}{2.174850in}}%
\pgfpathlineto{\pgfqpoint{2.435225in}{2.162116in}}%
\pgfpathlineto{\pgfqpoint{2.468218in}{2.145869in}}%
\pgfpathlineto{\pgfqpoint{2.501212in}{2.127118in}}%
\pgfpathlineto{\pgfqpoint{2.536562in}{2.104414in}}%
\pgfpathlineto{\pgfqpoint{2.574269in}{2.077440in}}%
\pgfpathlineto{\pgfqpoint{2.614332in}{2.045962in}}%
\pgfpathlineto{\pgfqpoint{2.659109in}{2.007782in}}%
\pgfpathlineto{\pgfqpoint{2.708599in}{1.962534in}}%
\pgfpathlineto{\pgfqpoint{2.767516in}{1.905559in}}%
\pgfpathlineto{\pgfqpoint{2.850000in}{1.822685in}}%
\pgfpathlineto{\pgfqpoint{2.850000in}{1.822685in}}%
\pgfusepath{stroke}%
\end{pgfscope}%
\begin{pgfscope}%
\pgfpathrectangle{\pgfqpoint{0.495679in}{0.549691in}}{\pgfqpoint{2.354321in}{5.091976in}}%
\pgfusepath{clip}%
\pgfsetbuttcap%
\pgfsetroundjoin%
\pgfsetlinewidth{0.501875pt}%
\definecolor{currentstroke}{rgb}{0.121569,0.466667,0.705882}%
\pgfsetstrokecolor{currentstroke}%
\pgfsetdash{{1.850000pt}{0.800000pt}}{0.000000pt}%
\pgfpathmoveto{\pgfqpoint{0.495679in}{1.822685in}}%
\pgfpathlineto{\pgfqpoint{2.850000in}{1.822685in}}%
\pgfusepath{stroke}%
\end{pgfscope}%
\begin{pgfscope}%
\pgfpathrectangle{\pgfqpoint{0.495679in}{0.549691in}}{\pgfqpoint{2.354321in}{5.091976in}}%
\pgfusepath{clip}%
\pgfsetrectcap%
\pgfsetroundjoin%
\pgfsetlinewidth{1.505625pt}%
\definecolor{currentstroke}{rgb}{0.172549,0.627451,0.172549}%
\pgfsetstrokecolor{currentstroke}%
\pgfsetdash{}{0pt}%
\pgfpathmoveto{\pgfqpoint{0.495679in}{2.671348in}}%
\pgfpathlineto{\pgfqpoint{0.573450in}{2.555087in}}%
\pgfpathlineto{\pgfqpoint{0.615870in}{2.495702in}}%
\pgfpathlineto{\pgfqpoint{0.651220in}{2.450005in}}%
\pgfpathlineto{\pgfqpoint{0.681857in}{2.413950in}}%
\pgfpathlineto{\pgfqpoint{0.710137in}{2.384089in}}%
\pgfpathlineto{\pgfqpoint{0.736060in}{2.359939in}}%
\pgfpathlineto{\pgfqpoint{0.759627in}{2.340886in}}%
\pgfpathlineto{\pgfqpoint{0.783194in}{2.324772in}}%
\pgfpathlineto{\pgfqpoint{0.804404in}{2.312902in}}%
\pgfpathlineto{\pgfqpoint{0.825614in}{2.303614in}}%
\pgfpathlineto{\pgfqpoint{0.844467in}{2.297580in}}%
\pgfpathlineto{\pgfqpoint{0.863321in}{2.293675in}}%
\pgfpathlineto{\pgfqpoint{0.882174in}{2.291920in}}%
\pgfpathlineto{\pgfqpoint{0.901028in}{2.292325in}}%
\pgfpathlineto{\pgfqpoint{0.919881in}{2.294888in}}%
\pgfpathlineto{\pgfqpoint{0.938735in}{2.299595in}}%
\pgfpathlineto{\pgfqpoint{0.957588in}{2.306418in}}%
\pgfpathlineto{\pgfqpoint{0.978798in}{2.316575in}}%
\pgfpathlineto{\pgfqpoint{1.000008in}{2.329289in}}%
\pgfpathlineto{\pgfqpoint{1.021218in}{2.344467in}}%
\pgfpathlineto{\pgfqpoint{1.044785in}{2.364088in}}%
\pgfpathlineto{\pgfqpoint{1.070708in}{2.388821in}}%
\pgfpathlineto{\pgfqpoint{1.096632in}{2.416593in}}%
\pgfpathlineto{\pgfqpoint{1.124912in}{2.450005in}}%
\pgfpathlineto{\pgfqpoint{1.157906in}{2.492535in}}%
\pgfpathlineto{\pgfqpoint{1.195612in}{2.544910in}}%
\pgfpathlineto{\pgfqpoint{1.245103in}{2.617818in}}%
\pgfpathlineto{\pgfqpoint{1.377077in}{2.814519in}}%
\pgfpathlineto{\pgfqpoint{1.414783in}{2.865746in}}%
\pgfpathlineto{\pgfqpoint{1.447777in}{2.906982in}}%
\pgfpathlineto{\pgfqpoint{1.476057in}{2.939085in}}%
\pgfpathlineto{\pgfqpoint{1.501980in}{2.965510in}}%
\pgfpathlineto{\pgfqpoint{1.527904in}{2.988771in}}%
\pgfpathlineto{\pgfqpoint{1.551471in}{3.006959in}}%
\pgfpathlineto{\pgfqpoint{1.572681in}{3.020780in}}%
\pgfpathlineto{\pgfqpoint{1.593891in}{3.032083in}}%
\pgfpathlineto{\pgfqpoint{1.615101in}{3.040788in}}%
\pgfpathlineto{\pgfqpoint{1.633954in}{3.046292in}}%
\pgfpathlineto{\pgfqpoint{1.652808in}{3.049661in}}%
\pgfpathlineto{\pgfqpoint{1.671661in}{3.050877in}}%
\pgfpathlineto{\pgfqpoint{1.690515in}{3.049931in}}%
\pgfpathlineto{\pgfqpoint{1.709368in}{3.046830in}}%
\pgfpathlineto{\pgfqpoint{1.728222in}{3.041592in}}%
\pgfpathlineto{\pgfqpoint{1.747075in}{3.034245in}}%
\pgfpathlineto{\pgfqpoint{1.768285in}{3.023513in}}%
\pgfpathlineto{\pgfqpoint{1.789495in}{3.010243in}}%
\pgfpathlineto{\pgfqpoint{1.813062in}{2.992640in}}%
\pgfpathlineto{\pgfqpoint{1.836629in}{2.972179in}}%
\pgfpathlineto{\pgfqpoint{1.862552in}{2.946590in}}%
\pgfpathlineto{\pgfqpoint{1.890832in}{2.915307in}}%
\pgfpathlineto{\pgfqpoint{1.921469in}{2.877906in}}%
\pgfpathlineto{\pgfqpoint{1.956819in}{2.830934in}}%
\pgfpathlineto{\pgfqpoint{1.999239in}{2.770442in}}%
\pgfpathlineto{\pgfqpoint{2.060513in}{2.678508in}}%
\pgfpathlineto{\pgfqpoint{2.142997in}{2.555087in}}%
\pgfpathlineto{\pgfqpoint{2.185417in}{2.495702in}}%
\pgfpathlineto{\pgfqpoint{2.220767in}{2.450005in}}%
\pgfpathlineto{\pgfqpoint{2.251404in}{2.413950in}}%
\pgfpathlineto{\pgfqpoint{2.279684in}{2.384089in}}%
\pgfpathlineto{\pgfqpoint{2.305607in}{2.359939in}}%
\pgfpathlineto{\pgfqpoint{2.329174in}{2.340886in}}%
\pgfpathlineto{\pgfqpoint{2.352741in}{2.324772in}}%
\pgfpathlineto{\pgfqpoint{2.373951in}{2.312902in}}%
\pgfpathlineto{\pgfqpoint{2.395161in}{2.303614in}}%
\pgfpathlineto{\pgfqpoint{2.414015in}{2.297580in}}%
\pgfpathlineto{\pgfqpoint{2.432868in}{2.293675in}}%
\pgfpathlineto{\pgfqpoint{2.451722in}{2.291920in}}%
\pgfpathlineto{\pgfqpoint{2.470575in}{2.292325in}}%
\pgfpathlineto{\pgfqpoint{2.489428in}{2.294888in}}%
\pgfpathlineto{\pgfqpoint{2.508282in}{2.299595in}}%
\pgfpathlineto{\pgfqpoint{2.527135in}{2.306418in}}%
\pgfpathlineto{\pgfqpoint{2.548345in}{2.316575in}}%
\pgfpathlineto{\pgfqpoint{2.569555in}{2.329289in}}%
\pgfpathlineto{\pgfqpoint{2.590765in}{2.344467in}}%
\pgfpathlineto{\pgfqpoint{2.614332in}{2.364088in}}%
\pgfpathlineto{\pgfqpoint{2.640256in}{2.388821in}}%
\pgfpathlineto{\pgfqpoint{2.666179in}{2.416593in}}%
\pgfpathlineto{\pgfqpoint{2.694459in}{2.450005in}}%
\pgfpathlineto{\pgfqpoint{2.727453in}{2.492535in}}%
\pgfpathlineto{\pgfqpoint{2.765160in}{2.544910in}}%
\pgfpathlineto{\pgfqpoint{2.814650in}{2.617818in}}%
\pgfpathlineto{\pgfqpoint{2.850000in}{2.671348in}}%
\pgfpathlineto{\pgfqpoint{2.850000in}{2.671348in}}%
\pgfusepath{stroke}%
\end{pgfscope}%
\begin{pgfscope}%
\pgfpathrectangle{\pgfqpoint{0.495679in}{0.549691in}}{\pgfqpoint{2.354321in}{5.091976in}}%
\pgfusepath{clip}%
\pgfsetbuttcap%
\pgfsetroundjoin%
\pgfsetlinewidth{0.501875pt}%
\definecolor{currentstroke}{rgb}{0.121569,0.466667,0.705882}%
\pgfsetstrokecolor{currentstroke}%
\pgfsetdash{{1.850000pt}{0.800000pt}}{0.000000pt}%
\pgfpathmoveto{\pgfqpoint{0.495679in}{2.671348in}}%
\pgfpathlineto{\pgfqpoint{2.850000in}{2.671348in}}%
\pgfusepath{stroke}%
\end{pgfscope}%
\begin{pgfscope}%
\pgfpathrectangle{\pgfqpoint{0.495679in}{0.549691in}}{\pgfqpoint{2.354321in}{5.091976in}}%
\pgfusepath{clip}%
\pgfsetrectcap%
\pgfsetroundjoin%
\pgfsetlinewidth{1.505625pt}%
\definecolor{currentstroke}{rgb}{0.839216,0.152941,0.156863}%
\pgfsetstrokecolor{currentstroke}%
\pgfsetdash{}{0pt}%
\pgfpathmoveto{\pgfqpoint{0.495679in}{3.520010in}}%
\pgfpathlineto{\pgfqpoint{0.559309in}{3.646448in}}%
\pgfpathlineto{\pgfqpoint{0.592303in}{3.707187in}}%
\pgfpathlineto{\pgfqpoint{0.620583in}{3.754707in}}%
\pgfpathlineto{\pgfqpoint{0.644150in}{3.790273in}}%
\pgfpathlineto{\pgfqpoint{0.665360in}{3.818645in}}%
\pgfpathlineto{\pgfqpoint{0.686570in}{3.843194in}}%
\pgfpathlineto{\pgfqpoint{0.705423in}{3.861550in}}%
\pgfpathlineto{\pgfqpoint{0.721920in}{3.874782in}}%
\pgfpathlineto{\pgfqpoint{0.738417in}{3.885266in}}%
\pgfpathlineto{\pgfqpoint{0.754914in}{3.892919in}}%
\pgfpathlineto{\pgfqpoint{0.769054in}{3.897181in}}%
\pgfpathlineto{\pgfqpoint{0.783194in}{3.899295in}}%
\pgfpathlineto{\pgfqpoint{0.797334in}{3.899250in}}%
\pgfpathlineto{\pgfqpoint{0.811474in}{3.897046in}}%
\pgfpathlineto{\pgfqpoint{0.825614in}{3.892695in}}%
\pgfpathlineto{\pgfqpoint{0.839754in}{3.886222in}}%
\pgfpathlineto{\pgfqpoint{0.856251in}{3.876039in}}%
\pgfpathlineto{\pgfqpoint{0.872748in}{3.863096in}}%
\pgfpathlineto{\pgfqpoint{0.889244in}{3.847496in}}%
\pgfpathlineto{\pgfqpoint{0.908098in}{3.826567in}}%
\pgfpathlineto{\pgfqpoint{0.929308in}{3.799327in}}%
\pgfpathlineto{\pgfqpoint{0.950518in}{3.768511in}}%
\pgfpathlineto{\pgfqpoint{0.976441in}{3.726569in}}%
\pgfpathlineto{\pgfqpoint{1.004722in}{3.676340in}}%
\pgfpathlineto{\pgfqpoint{1.040072in}{3.608698in}}%
\pgfpathlineto{\pgfqpoint{1.101345in}{3.485446in}}%
\pgfpathlineto{\pgfqpoint{1.150836in}{3.387962in}}%
\pgfpathlineto{\pgfqpoint{1.183829in}{3.327665in}}%
\pgfpathlineto{\pgfqpoint{1.212109in}{3.280652in}}%
\pgfpathlineto{\pgfqpoint{1.235676in}{3.245591in}}%
\pgfpathlineto{\pgfqpoint{1.256886in}{3.217729in}}%
\pgfpathlineto{\pgfqpoint{1.275739in}{3.196202in}}%
\pgfpathlineto{\pgfqpoint{1.294593in}{3.177951in}}%
\pgfpathlineto{\pgfqpoint{1.311090in}{3.164816in}}%
\pgfpathlineto{\pgfqpoint{1.327586in}{3.154432in}}%
\pgfpathlineto{\pgfqpoint{1.344083in}{3.146881in}}%
\pgfpathlineto{\pgfqpoint{1.358223in}{3.142708in}}%
\pgfpathlineto{\pgfqpoint{1.372363in}{3.140684in}}%
\pgfpathlineto{\pgfqpoint{1.386503in}{3.140819in}}%
\pgfpathlineto{\pgfqpoint{1.400643in}{3.143113in}}%
\pgfpathlineto{\pgfqpoint{1.414783in}{3.147553in}}%
\pgfpathlineto{\pgfqpoint{1.428923in}{3.154113in}}%
\pgfpathlineto{\pgfqpoint{1.445420in}{3.164397in}}%
\pgfpathlineto{\pgfqpoint{1.461917in}{3.177436in}}%
\pgfpathlineto{\pgfqpoint{1.478414in}{3.193129in}}%
\pgfpathlineto{\pgfqpoint{1.497267in}{3.214158in}}%
\pgfpathlineto{\pgfqpoint{1.518477in}{3.241503in}}%
\pgfpathlineto{\pgfqpoint{1.542044in}{3.276051in}}%
\pgfpathlineto{\pgfqpoint{1.567967in}{3.318483in}}%
\pgfpathlineto{\pgfqpoint{1.596248in}{3.369137in}}%
\pgfpathlineto{\pgfqpoint{1.631598in}{3.437136in}}%
\pgfpathlineto{\pgfqpoint{1.765928in}{3.700924in}}%
\pgfpathlineto{\pgfqpoint{1.794208in}{3.749038in}}%
\pgfpathlineto{\pgfqpoint{1.817775in}{3.785197in}}%
\pgfpathlineto{\pgfqpoint{1.838985in}{3.814173in}}%
\pgfpathlineto{\pgfqpoint{1.860195in}{3.839382in}}%
\pgfpathlineto{\pgfqpoint{1.879049in}{3.858366in}}%
\pgfpathlineto{\pgfqpoint{1.895546in}{3.872175in}}%
\pgfpathlineto{\pgfqpoint{1.912042in}{3.883255in}}%
\pgfpathlineto{\pgfqpoint{1.928539in}{3.891521in}}%
\pgfpathlineto{\pgfqpoint{1.942679in}{3.896316in}}%
\pgfpathlineto{\pgfqpoint{1.956819in}{3.898969in}}%
\pgfpathlineto{\pgfqpoint{1.970959in}{3.899464in}}%
\pgfpathlineto{\pgfqpoint{1.985099in}{3.897799in}}%
\pgfpathlineto{\pgfqpoint{1.999239in}{3.893983in}}%
\pgfpathlineto{\pgfqpoint{2.013379in}{3.888038in}}%
\pgfpathlineto{\pgfqpoint{2.029876in}{3.878456in}}%
\pgfpathlineto{\pgfqpoint{2.046373in}{3.866097in}}%
\pgfpathlineto{\pgfqpoint{2.062870in}{3.851057in}}%
\pgfpathlineto{\pgfqpoint{2.081723in}{3.830735in}}%
\pgfpathlineto{\pgfqpoint{2.100577in}{3.807268in}}%
\pgfpathlineto{\pgfqpoint{2.121787in}{3.777408in}}%
\pgfpathlineto{\pgfqpoint{2.145353in}{3.740382in}}%
\pgfpathlineto{\pgfqpoint{2.173634in}{3.691410in}}%
\pgfpathlineto{\pgfqpoint{2.206627in}{3.629433in}}%
\pgfpathlineto{\pgfqpoint{2.253761in}{3.535522in}}%
\pgfpathlineto{\pgfqpoint{2.324461in}{3.394698in}}%
\pgfpathlineto{\pgfqpoint{2.357454in}{3.333873in}}%
\pgfpathlineto{\pgfqpoint{2.385735in}{3.286252in}}%
\pgfpathlineto{\pgfqpoint{2.409301in}{3.250587in}}%
\pgfpathlineto{\pgfqpoint{2.430511in}{3.222113in}}%
\pgfpathlineto{\pgfqpoint{2.451722in}{3.197454in}}%
\pgfpathlineto{\pgfqpoint{2.470575in}{3.178992in}}%
\pgfpathlineto{\pgfqpoint{2.487072in}{3.165664in}}%
\pgfpathlineto{\pgfqpoint{2.503568in}{3.155081in}}%
\pgfpathlineto{\pgfqpoint{2.520065in}{3.147325in}}%
\pgfpathlineto{\pgfqpoint{2.534205in}{3.142974in}}%
\pgfpathlineto{\pgfqpoint{2.548345in}{3.140770in}}%
\pgfpathlineto{\pgfqpoint{2.562485in}{3.140725in}}%
\pgfpathlineto{\pgfqpoint{2.576625in}{3.142839in}}%
\pgfpathlineto{\pgfqpoint{2.590765in}{3.147101in}}%
\pgfpathlineto{\pgfqpoint{2.604906in}{3.153486in}}%
\pgfpathlineto{\pgfqpoint{2.621402in}{3.163570in}}%
\pgfpathlineto{\pgfqpoint{2.637899in}{3.176415in}}%
\pgfpathlineto{\pgfqpoint{2.654396in}{3.191923in}}%
\pgfpathlineto{\pgfqpoint{2.673249in}{3.212751in}}%
\pgfpathlineto{\pgfqpoint{2.694459in}{3.239887in}}%
\pgfpathlineto{\pgfqpoint{2.715669in}{3.270609in}}%
\pgfpathlineto{\pgfqpoint{2.739236in}{3.308471in}}%
\pgfpathlineto{\pgfqpoint{2.767516in}{3.358262in}}%
\pgfpathlineto{\pgfqpoint{2.802866in}{3.425532in}}%
\pgfpathlineto{\pgfqpoint{2.850000in}{3.520010in}}%
\pgfpathlineto{\pgfqpoint{2.850000in}{3.520010in}}%
\pgfusepath{stroke}%
\end{pgfscope}%
\begin{pgfscope}%
\pgfpathrectangle{\pgfqpoint{0.495679in}{0.549691in}}{\pgfqpoint{2.354321in}{5.091976in}}%
\pgfusepath{clip}%
\pgfsetbuttcap%
\pgfsetroundjoin%
\pgfsetlinewidth{0.501875pt}%
\definecolor{currentstroke}{rgb}{0.121569,0.466667,0.705882}%
\pgfsetstrokecolor{currentstroke}%
\pgfsetdash{{1.850000pt}{0.800000pt}}{0.000000pt}%
\pgfpathmoveto{\pgfqpoint{0.495679in}{3.520010in}}%
\pgfpathlineto{\pgfqpoint{2.850000in}{3.520010in}}%
\pgfusepath{stroke}%
\end{pgfscope}%
\begin{pgfscope}%
\pgfpathrectangle{\pgfqpoint{0.495679in}{0.549691in}}{\pgfqpoint{2.354321in}{5.091976in}}%
\pgfusepath{clip}%
\pgfsetrectcap%
\pgfsetroundjoin%
\pgfsetlinewidth{1.505625pt}%
\definecolor{currentstroke}{rgb}{0.580392,0.403922,0.741176}%
\pgfsetstrokecolor{currentstroke}%
\pgfsetdash{}{0pt}%
\pgfpathmoveto{\pgfqpoint{0.495679in}{4.368673in}}%
\pgfpathlineto{\pgfqpoint{0.549883in}{4.502957in}}%
\pgfpathlineto{\pgfqpoint{0.578163in}{4.567157in}}%
\pgfpathlineto{\pgfqpoint{0.601730in}{4.615364in}}%
\pgfpathlineto{\pgfqpoint{0.622940in}{4.653578in}}%
\pgfpathlineto{\pgfqpoint{0.641793in}{4.682785in}}%
\pgfpathlineto{\pgfqpoint{0.658290in}{4.704284in}}%
\pgfpathlineto{\pgfqpoint{0.672430in}{4.719487in}}%
\pgfpathlineto{\pgfqpoint{0.686570in}{4.731570in}}%
\pgfpathlineto{\pgfqpoint{0.700710in}{4.740426in}}%
\pgfpathlineto{\pgfqpoint{0.712493in}{4.745282in}}%
\pgfpathlineto{\pgfqpoint{0.724277in}{4.747812in}}%
\pgfpathlineto{\pgfqpoint{0.736060in}{4.747999in}}%
\pgfpathlineto{\pgfqpoint{0.747844in}{4.745844in}}%
\pgfpathlineto{\pgfqpoint{0.759627in}{4.741358in}}%
\pgfpathlineto{\pgfqpoint{0.771410in}{4.734570in}}%
\pgfpathlineto{\pgfqpoint{0.785551in}{4.723445in}}%
\pgfpathlineto{\pgfqpoint{0.799691in}{4.709165in}}%
\pgfpathlineto{\pgfqpoint{0.813831in}{4.691857in}}%
\pgfpathlineto{\pgfqpoint{0.830327in}{4.668043in}}%
\pgfpathlineto{\pgfqpoint{0.849181in}{4.636410in}}%
\pgfpathlineto{\pgfqpoint{0.870391in}{4.595793in}}%
\pgfpathlineto{\pgfqpoint{0.893958in}{4.545376in}}%
\pgfpathlineto{\pgfqpoint{0.922238in}{4.479238in}}%
\pgfpathlineto{\pgfqpoint{0.964658in}{4.373447in}}%
\pgfpathlineto{\pgfqpoint{1.018862in}{4.238865in}}%
\pgfpathlineto{\pgfqpoint{1.047142in}{4.174274in}}%
\pgfpathlineto{\pgfqpoint{1.070708in}{4.125629in}}%
\pgfpathlineto{\pgfqpoint{1.091919in}{4.086944in}}%
\pgfpathlineto{\pgfqpoint{1.110772in}{4.057265in}}%
\pgfpathlineto{\pgfqpoint{1.127269in}{4.035318in}}%
\pgfpathlineto{\pgfqpoint{1.141409in}{4.019708in}}%
\pgfpathlineto{\pgfqpoint{1.155549in}{4.007202in}}%
\pgfpathlineto{\pgfqpoint{1.169689in}{3.997911in}}%
\pgfpathlineto{\pgfqpoint{1.181472in}{3.992685in}}%
\pgfpathlineto{\pgfqpoint{1.193256in}{3.989781in}}%
\pgfpathlineto{\pgfqpoint{1.205039in}{3.989219in}}%
\pgfpathlineto{\pgfqpoint{1.216822in}{3.991000in}}%
\pgfpathlineto{\pgfqpoint{1.228606in}{3.995114in}}%
\pgfpathlineto{\pgfqpoint{1.240389in}{4.001537in}}%
\pgfpathlineto{\pgfqpoint{1.252173in}{4.010227in}}%
\pgfpathlineto{\pgfqpoint{1.266313in}{4.023572in}}%
\pgfpathlineto{\pgfqpoint{1.280453in}{4.039987in}}%
\pgfpathlineto{\pgfqpoint{1.296950in}{4.062821in}}%
\pgfpathlineto{\pgfqpoint{1.313446in}{4.089356in}}%
\pgfpathlineto{\pgfqpoint{1.332300in}{4.123800in}}%
\pgfpathlineto{\pgfqpoint{1.353510in}{4.167146in}}%
\pgfpathlineto{\pgfqpoint{1.379433in}{4.225502in}}%
\pgfpathlineto{\pgfqpoint{1.412427in}{4.305708in}}%
\pgfpathlineto{\pgfqpoint{1.504337in}{4.532578in}}%
\pgfpathlineto{\pgfqpoint{1.530261in}{4.589045in}}%
\pgfpathlineto{\pgfqpoint{1.551471in}{4.630424in}}%
\pgfpathlineto{\pgfqpoint{1.570324in}{4.662835in}}%
\pgfpathlineto{\pgfqpoint{1.586821in}{4.687398in}}%
\pgfpathlineto{\pgfqpoint{1.603318in}{4.708104in}}%
\pgfpathlineto{\pgfqpoint{1.617458in}{4.722590in}}%
\pgfpathlineto{\pgfqpoint{1.631598in}{4.733928in}}%
\pgfpathlineto{\pgfqpoint{1.643381in}{4.740899in}}%
\pgfpathlineto{\pgfqpoint{1.655165in}{4.745570in}}%
\pgfpathlineto{\pgfqpoint{1.666948in}{4.747913in}}%
\pgfpathlineto{\pgfqpoint{1.678731in}{4.747913in}}%
\pgfpathlineto{\pgfqpoint{1.690515in}{4.745570in}}%
\pgfpathlineto{\pgfqpoint{1.702298in}{4.740899in}}%
\pgfpathlineto{\pgfqpoint{1.714081in}{4.733928in}}%
\pgfpathlineto{\pgfqpoint{1.728222in}{4.722590in}}%
\pgfpathlineto{\pgfqpoint{1.742362in}{4.708104in}}%
\pgfpathlineto{\pgfqpoint{1.756502in}{4.690599in}}%
\pgfpathlineto{\pgfqpoint{1.772998in}{4.666570in}}%
\pgfpathlineto{\pgfqpoint{1.791852in}{4.634713in}}%
\pgfpathlineto{\pgfqpoint{1.813062in}{4.593876in}}%
\pgfpathlineto{\pgfqpoint{1.836629in}{4.543260in}}%
\pgfpathlineto{\pgfqpoint{1.864909in}{4.476952in}}%
\pgfpathlineto{\pgfqpoint{1.907329in}{4.371060in}}%
\pgfpathlineto{\pgfqpoint{1.961533in}{4.236624in}}%
\pgfpathlineto{\pgfqpoint{1.989813in}{4.172228in}}%
\pgfpathlineto{\pgfqpoint{2.013379in}{4.123800in}}%
\pgfpathlineto{\pgfqpoint{2.034590in}{4.085350in}}%
\pgfpathlineto{\pgfqpoint{2.053443in}{4.055906in}}%
\pgfpathlineto{\pgfqpoint{2.069940in}{4.034183in}}%
\pgfpathlineto{\pgfqpoint{2.084080in}{4.018776in}}%
\pgfpathlineto{\pgfqpoint{2.098220in}{4.006482in}}%
\pgfpathlineto{\pgfqpoint{2.112360in}{3.997408in}}%
\pgfpathlineto{\pgfqpoint{2.124143in}{3.992367in}}%
\pgfpathlineto{\pgfqpoint{2.135927in}{3.989650in}}%
\pgfpathlineto{\pgfqpoint{2.147710in}{3.989275in}}%
\pgfpathlineto{\pgfqpoint{2.159493in}{3.991243in}}%
\pgfpathlineto{\pgfqpoint{2.171277in}{3.995544in}}%
\pgfpathlineto{\pgfqpoint{2.183060in}{4.002149in}}%
\pgfpathlineto{\pgfqpoint{2.194844in}{4.011018in}}%
\pgfpathlineto{\pgfqpoint{2.208984in}{4.024573in}}%
\pgfpathlineto{\pgfqpoint{2.223124in}{4.041187in}}%
\pgfpathlineto{\pgfqpoint{2.239621in}{4.064240in}}%
\pgfpathlineto{\pgfqpoint{2.256117in}{4.090978in}}%
\pgfpathlineto{\pgfqpoint{2.274971in}{4.125629in}}%
\pgfpathlineto{\pgfqpoint{2.296181in}{4.169173in}}%
\pgfpathlineto{\pgfqpoint{2.322104in}{4.227715in}}%
\pgfpathlineto{\pgfqpoint{2.355098in}{4.308063in}}%
\pgfpathlineto{\pgfqpoint{2.444651in}{4.529341in}}%
\pgfpathlineto{\pgfqpoint{2.470575in}{4.586120in}}%
\pgfpathlineto{\pgfqpoint{2.491785in}{4.627819in}}%
\pgfpathlineto{\pgfqpoint{2.510638in}{4.660560in}}%
\pgfpathlineto{\pgfqpoint{2.529492in}{4.688688in}}%
\pgfpathlineto{\pgfqpoint{2.545989in}{4.709165in}}%
\pgfpathlineto{\pgfqpoint{2.560129in}{4.723445in}}%
\pgfpathlineto{\pgfqpoint{2.574269in}{4.734570in}}%
\pgfpathlineto{\pgfqpoint{2.586052in}{4.741358in}}%
\pgfpathlineto{\pgfqpoint{2.597836in}{4.745844in}}%
\pgfpathlineto{\pgfqpoint{2.609619in}{4.747999in}}%
\pgfpathlineto{\pgfqpoint{2.621402in}{4.747812in}}%
\pgfpathlineto{\pgfqpoint{2.633186in}{4.745282in}}%
\pgfpathlineto{\pgfqpoint{2.644969in}{4.740426in}}%
\pgfpathlineto{\pgfqpoint{2.656752in}{4.733273in}}%
\pgfpathlineto{\pgfqpoint{2.670893in}{4.721721in}}%
\pgfpathlineto{\pgfqpoint{2.685033in}{4.707029in}}%
\pgfpathlineto{\pgfqpoint{2.699173in}{4.689328in}}%
\pgfpathlineto{\pgfqpoint{2.715669in}{4.665085in}}%
\pgfpathlineto{\pgfqpoint{2.734523in}{4.633005in}}%
\pgfpathlineto{\pgfqpoint{2.755733in}{4.591950in}}%
\pgfpathlineto{\pgfqpoint{2.779300in}{4.541137in}}%
\pgfpathlineto{\pgfqpoint{2.807580in}{4.474662in}}%
\pgfpathlineto{\pgfqpoint{2.850000in}{4.368673in}}%
\pgfpathlineto{\pgfqpoint{2.850000in}{4.368673in}}%
\pgfusepath{stroke}%
\end{pgfscope}%
\begin{pgfscope}%
\pgfpathrectangle{\pgfqpoint{0.495679in}{0.549691in}}{\pgfqpoint{2.354321in}{5.091976in}}%
\pgfusepath{clip}%
\pgfsetbuttcap%
\pgfsetroundjoin%
\pgfsetlinewidth{0.501875pt}%
\definecolor{currentstroke}{rgb}{0.121569,0.466667,0.705882}%
\pgfsetstrokecolor{currentstroke}%
\pgfsetdash{{1.850000pt}{0.800000pt}}{0.000000pt}%
\pgfpathmoveto{\pgfqpoint{0.495679in}{4.368673in}}%
\pgfpathlineto{\pgfqpoint{2.850000in}{4.368673in}}%
\pgfusepath{stroke}%
\end{pgfscope}%
\begin{pgfscope}%
\pgfpathrectangle{\pgfqpoint{0.495679in}{0.549691in}}{\pgfqpoint{2.354321in}{5.091976in}}%
\pgfusepath{clip}%
\pgfsetrectcap%
\pgfsetroundjoin%
\pgfsetlinewidth{1.505625pt}%
\definecolor{currentstroke}{rgb}{0.549020,0.337255,0.294118}%
\pgfsetstrokecolor{currentstroke}%
\pgfsetdash{}{0pt}%
\pgfpathmoveto{\pgfqpoint{0.495679in}{5.217335in}}%
\pgfpathlineto{\pgfqpoint{0.542813in}{5.077487in}}%
\pgfpathlineto{\pgfqpoint{0.568736in}{5.007782in}}%
\pgfpathlineto{\pgfqpoint{0.589946in}{4.957318in}}%
\pgfpathlineto{\pgfqpoint{0.606443in}{4.923173in}}%
\pgfpathlineto{\pgfqpoint{0.622940in}{4.894152in}}%
\pgfpathlineto{\pgfqpoint{0.637080in}{4.873741in}}%
\pgfpathlineto{\pgfqpoint{0.648863in}{4.860082in}}%
\pgfpathlineto{\pgfqpoint{0.660647in}{4.849601in}}%
\pgfpathlineto{\pgfqpoint{0.672430in}{4.842391in}}%
\pgfpathlineto{\pgfqpoint{0.681857in}{4.839022in}}%
\pgfpathlineto{\pgfqpoint{0.691283in}{4.837806in}}%
\pgfpathlineto{\pgfqpoint{0.700710in}{4.838752in}}%
\pgfpathlineto{\pgfqpoint{0.710137in}{4.841853in}}%
\pgfpathlineto{\pgfqpoint{0.719564in}{4.847091in}}%
\pgfpathlineto{\pgfqpoint{0.731347in}{4.856600in}}%
\pgfpathlineto{\pgfqpoint{0.743130in}{4.869316in}}%
\pgfpathlineto{\pgfqpoint{0.754914in}{4.885128in}}%
\pgfpathlineto{\pgfqpoint{0.769054in}{4.907988in}}%
\pgfpathlineto{\pgfqpoint{0.785551in}{4.939640in}}%
\pgfpathlineto{\pgfqpoint{0.802047in}{4.976130in}}%
\pgfpathlineto{\pgfqpoint{0.823257in}{5.029121in}}%
\pgfpathlineto{\pgfqpoint{0.846824in}{5.094279in}}%
\pgfpathlineto{\pgfqpoint{0.882174in}{5.199439in}}%
\pgfpathlineto{\pgfqpoint{0.936378in}{5.360506in}}%
\pgfpathlineto{\pgfqpoint{0.959945in}{5.423894in}}%
\pgfpathlineto{\pgfqpoint{0.981155in}{5.474733in}}%
\pgfpathlineto{\pgfqpoint{0.997651in}{5.509222in}}%
\pgfpathlineto{\pgfqpoint{1.014148in}{5.538627in}}%
\pgfpathlineto{\pgfqpoint{1.028288in}{5.559394in}}%
\pgfpathlineto{\pgfqpoint{1.040072in}{5.573364in}}%
\pgfpathlineto{\pgfqpoint{1.051855in}{5.584167in}}%
\pgfpathlineto{\pgfqpoint{1.063638in}{5.591708in}}%
\pgfpathlineto{\pgfqpoint{1.073065in}{5.595346in}}%
\pgfpathlineto{\pgfqpoint{1.082492in}{5.596831in}}%
\pgfpathlineto{\pgfqpoint{1.091919in}{5.596155in}}%
\pgfpathlineto{\pgfqpoint{1.101345in}{5.593323in}}%
\pgfpathlineto{\pgfqpoint{1.110772in}{5.588350in}}%
\pgfpathlineto{\pgfqpoint{1.122555in}{5.579168in}}%
\pgfpathlineto{\pgfqpoint{1.134339in}{5.566768in}}%
\pgfpathlineto{\pgfqpoint{1.146122in}{5.551260in}}%
\pgfpathlineto{\pgfqpoint{1.160262in}{5.528744in}}%
\pgfpathlineto{\pgfqpoint{1.176759in}{5.497459in}}%
\pgfpathlineto{\pgfqpoint{1.193256in}{5.461295in}}%
\pgfpathlineto{\pgfqpoint{1.212109in}{5.414801in}}%
\pgfpathlineto{\pgfqpoint{1.235676in}{5.350502in}}%
\pgfpathlineto{\pgfqpoint{1.268669in}{5.253088in}}%
\pgfpathlineto{\pgfqpoint{1.329943in}{5.070855in}}%
\pgfpathlineto{\pgfqpoint{1.353510in}{5.007782in}}%
\pgfpathlineto{\pgfqpoint{1.374720in}{4.957318in}}%
\pgfpathlineto{\pgfqpoint{1.391217in}{4.923173in}}%
\pgfpathlineto{\pgfqpoint{1.407713in}{4.894152in}}%
\pgfpathlineto{\pgfqpoint{1.421853in}{4.873741in}}%
\pgfpathlineto{\pgfqpoint{1.433637in}{4.860082in}}%
\pgfpathlineto{\pgfqpoint{1.445420in}{4.849601in}}%
\pgfpathlineto{\pgfqpoint{1.457204in}{4.842391in}}%
\pgfpathlineto{\pgfqpoint{1.466630in}{4.839022in}}%
\pgfpathlineto{\pgfqpoint{1.476057in}{4.837806in}}%
\pgfpathlineto{\pgfqpoint{1.485484in}{4.838752in}}%
\pgfpathlineto{\pgfqpoint{1.494910in}{4.841853in}}%
\pgfpathlineto{\pgfqpoint{1.504337in}{4.847091in}}%
\pgfpathlineto{\pgfqpoint{1.516121in}{4.856600in}}%
\pgfpathlineto{\pgfqpoint{1.527904in}{4.869316in}}%
\pgfpathlineto{\pgfqpoint{1.539687in}{4.885128in}}%
\pgfpathlineto{\pgfqpoint{1.553827in}{4.907988in}}%
\pgfpathlineto{\pgfqpoint{1.570324in}{4.939640in}}%
\pgfpathlineto{\pgfqpoint{1.586821in}{4.976130in}}%
\pgfpathlineto{\pgfqpoint{1.608031in}{5.029121in}}%
\pgfpathlineto{\pgfqpoint{1.631598in}{5.094279in}}%
\pgfpathlineto{\pgfqpoint{1.666948in}{5.199439in}}%
\pgfpathlineto{\pgfqpoint{1.721151in}{5.360506in}}%
\pgfpathlineto{\pgfqpoint{1.744718in}{5.423894in}}%
\pgfpathlineto{\pgfqpoint{1.765928in}{5.474733in}}%
\pgfpathlineto{\pgfqpoint{1.782425in}{5.509222in}}%
\pgfpathlineto{\pgfqpoint{1.798922in}{5.538627in}}%
\pgfpathlineto{\pgfqpoint{1.813062in}{5.559394in}}%
\pgfpathlineto{\pgfqpoint{1.824845in}{5.573364in}}%
\pgfpathlineto{\pgfqpoint{1.836629in}{5.584167in}}%
\pgfpathlineto{\pgfqpoint{1.848412in}{5.591708in}}%
\pgfpathlineto{\pgfqpoint{1.857839in}{5.595346in}}%
\pgfpathlineto{\pgfqpoint{1.867265in}{5.596831in}}%
\pgfpathlineto{\pgfqpoint{1.876692in}{5.596155in}}%
\pgfpathlineto{\pgfqpoint{1.886119in}{5.593323in}}%
\pgfpathlineto{\pgfqpoint{1.895546in}{5.588350in}}%
\pgfpathlineto{\pgfqpoint{1.907329in}{5.579168in}}%
\pgfpathlineto{\pgfqpoint{1.919112in}{5.566768in}}%
\pgfpathlineto{\pgfqpoint{1.930896in}{5.551260in}}%
\pgfpathlineto{\pgfqpoint{1.945036in}{5.528744in}}%
\pgfpathlineto{\pgfqpoint{1.961533in}{5.497459in}}%
\pgfpathlineto{\pgfqpoint{1.978029in}{5.461295in}}%
\pgfpathlineto{\pgfqpoint{1.996883in}{5.414801in}}%
\pgfpathlineto{\pgfqpoint{2.020450in}{5.350502in}}%
\pgfpathlineto{\pgfqpoint{2.053443in}{5.253088in}}%
\pgfpathlineto{\pgfqpoint{2.114717in}{5.070855in}}%
\pgfpathlineto{\pgfqpoint{2.138283in}{5.007782in}}%
\pgfpathlineto{\pgfqpoint{2.159493in}{4.957318in}}%
\pgfpathlineto{\pgfqpoint{2.175990in}{4.923173in}}%
\pgfpathlineto{\pgfqpoint{2.192487in}{4.894152in}}%
\pgfpathlineto{\pgfqpoint{2.206627in}{4.873741in}}%
\pgfpathlineto{\pgfqpoint{2.218410in}{4.860082in}}%
\pgfpathlineto{\pgfqpoint{2.230194in}{4.849601in}}%
\pgfpathlineto{\pgfqpoint{2.241977in}{4.842391in}}%
\pgfpathlineto{\pgfqpoint{2.251404in}{4.839022in}}%
\pgfpathlineto{\pgfqpoint{2.260831in}{4.837806in}}%
\pgfpathlineto{\pgfqpoint{2.270257in}{4.838752in}}%
\pgfpathlineto{\pgfqpoint{2.279684in}{4.841853in}}%
\pgfpathlineto{\pgfqpoint{2.289111in}{4.847091in}}%
\pgfpathlineto{\pgfqpoint{2.300894in}{4.856600in}}%
\pgfpathlineto{\pgfqpoint{2.312678in}{4.869316in}}%
\pgfpathlineto{\pgfqpoint{2.324461in}{4.885128in}}%
\pgfpathlineto{\pgfqpoint{2.338601in}{4.907988in}}%
\pgfpathlineto{\pgfqpoint{2.355098in}{4.939640in}}%
\pgfpathlineto{\pgfqpoint{2.371594in}{4.976130in}}%
\pgfpathlineto{\pgfqpoint{2.392805in}{5.029121in}}%
\pgfpathlineto{\pgfqpoint{2.416371in}{5.094279in}}%
\pgfpathlineto{\pgfqpoint{2.451722in}{5.199439in}}%
\pgfpathlineto{\pgfqpoint{2.505925in}{5.360506in}}%
\pgfpathlineto{\pgfqpoint{2.529492in}{5.423894in}}%
\pgfpathlineto{\pgfqpoint{2.550702in}{5.474733in}}%
\pgfpathlineto{\pgfqpoint{2.567199in}{5.509222in}}%
\pgfpathlineto{\pgfqpoint{2.583695in}{5.538627in}}%
\pgfpathlineto{\pgfqpoint{2.597836in}{5.559394in}}%
\pgfpathlineto{\pgfqpoint{2.609619in}{5.573364in}}%
\pgfpathlineto{\pgfqpoint{2.621402in}{5.584167in}}%
\pgfpathlineto{\pgfqpoint{2.633186in}{5.591708in}}%
\pgfpathlineto{\pgfqpoint{2.642612in}{5.595346in}}%
\pgfpathlineto{\pgfqpoint{2.652039in}{5.596831in}}%
\pgfpathlineto{\pgfqpoint{2.661466in}{5.596155in}}%
\pgfpathlineto{\pgfqpoint{2.670893in}{5.593323in}}%
\pgfpathlineto{\pgfqpoint{2.680319in}{5.588350in}}%
\pgfpathlineto{\pgfqpoint{2.692103in}{5.579168in}}%
\pgfpathlineto{\pgfqpoint{2.703886in}{5.566768in}}%
\pgfpathlineto{\pgfqpoint{2.715669in}{5.551260in}}%
\pgfpathlineto{\pgfqpoint{2.729809in}{5.528744in}}%
\pgfpathlineto{\pgfqpoint{2.746306in}{5.497459in}}%
\pgfpathlineto{\pgfqpoint{2.762803in}{5.461295in}}%
\pgfpathlineto{\pgfqpoint{2.781656in}{5.414801in}}%
\pgfpathlineto{\pgfqpoint{2.805223in}{5.350502in}}%
\pgfpathlineto{\pgfqpoint{2.838217in}{5.253088in}}%
\pgfpathlineto{\pgfqpoint{2.850000in}{5.217335in}}%
\pgfpathlineto{\pgfqpoint{2.850000in}{5.217335in}}%
\pgfusepath{stroke}%
\end{pgfscope}%
\begin{pgfscope}%
\pgfpathrectangle{\pgfqpoint{0.495679in}{0.549691in}}{\pgfqpoint{2.354321in}{5.091976in}}%
\pgfusepath{clip}%
\pgfsetbuttcap%
\pgfsetroundjoin%
\pgfsetlinewidth{0.501875pt}%
\definecolor{currentstroke}{rgb}{0.121569,0.466667,0.705882}%
\pgfsetstrokecolor{currentstroke}%
\pgfsetdash{{1.850000pt}{0.800000pt}}{0.000000pt}%
\pgfpathmoveto{\pgfqpoint{0.495679in}{5.217335in}}%
\pgfpathlineto{\pgfqpoint{2.850000in}{5.217335in}}%
\pgfusepath{stroke}%
\end{pgfscope}%
\begin{pgfscope}%
\pgfpathrectangle{\pgfqpoint{0.495679in}{0.549691in}}{\pgfqpoint{2.354321in}{5.091976in}}%
\pgfusepath{clip}%
\pgfsetbuttcap%
\pgfsetroundjoin%
\pgfsetlinewidth{0.501875pt}%
\definecolor{currentstroke}{rgb}{0.121569,0.466667,0.705882}%
\pgfsetstrokecolor{currentstroke}%
\pgfsetdash{{1.850000pt}{0.800000pt}}{0.000000pt}%
\pgfpathmoveto{\pgfqpoint{1.672840in}{0.549691in}}%
\pgfpathlineto{\pgfqpoint{1.672840in}{5.641667in}}%
\pgfusepath{stroke}%
\end{pgfscope}%
\begin{pgfscope}%
\pgfsetrectcap%
\pgfsetmiterjoin%
\pgfsetlinewidth{0.803000pt}%
\definecolor{currentstroke}{rgb}{0.000000,0.000000,0.000000}%
\pgfsetstrokecolor{currentstroke}%
\pgfsetdash{}{0pt}%
\pgfpathmoveto{\pgfqpoint{0.495679in}{0.549691in}}%
\pgfpathlineto{\pgfqpoint{0.495679in}{5.641667in}}%
\pgfusepath{stroke}%
\end{pgfscope}%
\begin{pgfscope}%
\pgfsetrectcap%
\pgfsetmiterjoin%
\pgfsetlinewidth{0.803000pt}%
\definecolor{currentstroke}{rgb}{0.000000,0.000000,0.000000}%
\pgfsetstrokecolor{currentstroke}%
\pgfsetdash{}{0pt}%
\pgfpathmoveto{\pgfqpoint{2.850000in}{0.549691in}}%
\pgfpathlineto{\pgfqpoint{2.850000in}{5.641667in}}%
\pgfusepath{stroke}%
\end{pgfscope}%
\begin{pgfscope}%
\pgfsetrectcap%
\pgfsetmiterjoin%
\pgfsetlinewidth{0.803000pt}%
\definecolor{currentstroke}{rgb}{0.000000,0.000000,0.000000}%
\pgfsetstrokecolor{currentstroke}%
\pgfsetdash{}{0pt}%
\pgfpathmoveto{\pgfqpoint{0.495679in}{0.549691in}}%
\pgfpathlineto{\pgfqpoint{2.850000in}{0.549691in}}%
\pgfusepath{stroke}%
\end{pgfscope}%
\begin{pgfscope}%
\definecolor{textcolor}{rgb}{0.000000,0.000000,0.000000}%
\pgfsetstrokecolor{textcolor}%
\pgfsetfillcolor{textcolor}%
\pgftext[x=1.672840in,y=5.725000in,,base]{\color{textcolor}\rmfamily\fontsize{12.000000}{14.400000}\selectfont \(\displaystyle \Phi_n(x)\)}%
\end{pgfscope}%
\end{pgfpicture}%
\makeatother%
\endgroup%

  \end{center}
  \caption{Wave functions for the analytical solution of the one-dimensional, time-independent Schr\"{o}dinger's describing one electron in an infinitely-deep potential well of width $10a$}
  \label{analytical-plot}
\end{figure}

For the energy eigenvalues for each of these wavefunctions, we'll need to relate the energy eigenvalue term $E_n$ back to the constant term k that was derived. We can do this by substituting the wave function derived and plug it into the Schr\"{o}dinger equation that is valid for within the well (where potential energy is equal to 0: $v(x) = 0$) as follows:

\begin{align*}
  -\frac{1}{2}\Phi^{''}_n(x) &= {E_n}\Phi_n(x) \\
  -\frac{1}{2}\left[ \frac{2}{\sqrt{20a}}\sin\left(\frac{\pi nx}{10a}\right) \right]\frac{d^2\Phi_n(x)}{dx^2} &= E_n\frac{2}{\sqrt{20a}}\sin\left(\frac{\pi nx}{10a}\right) \\
  -\frac{1}{2}\left[ \cancel{\frac{2}{\sqrt{20a}}} \left(-\frac{n^2\pi^2}{100a^2}\right) \cancel{\sin\left(\frac{\pi nx}{10a}\right)} \right] &= E_n\cancel{ \frac{2}{\sqrt{20a}}\sin\left(\frac{\pi nx}{10a}\right) } \\
  \frac{n^2\pi^2}{200a^2} &= E_n \\
  E_o &= \frac{\pi^2}{200a^2} \\
  \Aboxed{E_n &= n^2E_o & }
\end{align*}
We started off with the even $n$ wavefunction, but the odd $n$ wavefunction results in the same energy eigenvalues. Therefore our result is valid for both even and odd $n$. Our first six eigenvalues are as follows:

\begin{table}
\begin{center}
\begin{tabular}{l|llllll}\hline
$n$    & $1$    & $2$     & $3$     & $4$      & $5$      & $6$      \\\hline
$E_n$  & $E_o$  & $4E_o$  & $9E_o$  & $16E_o$  & $25E_o$  & $36E_o$ \\\hline
\end{tabular}
\end{center}
  \caption{The first six energy levels calculated for this problem through the analytical method. This will be compared to the following problem's results.}
  \label{analytical-energies}
\end{table}

\subsection{Numerical Solution 1: Single Electron}

In order to solve the single electron problem numerically, we must discretize the following equation:

\begin{align*}
  -\frac{1}{2}\Phi^{''}_n(x) + v(x)\Phi_n(x) &= {E_n}\Phi_n(x)
  & v(x_i) = \begin{cases}
          0       \quad \, & |x| < 5a \\
          \infty  \quad \, & |x| > 5a \\
     \end{cases}
\end{align*}

To accomplish this, we can replace the second derivative of the wave function in the one-dimensional, time-independent Schr\"{o}dinger's with a second-order centered difference approximation which has the following form:

\begin{align*}
  \frac{d^2f(x)}{dx^2} &\approx \frac{f(x + \Delta x) - 2f(x) + f(x - \Delta x)}{\Delta x^2}
\end{align*}

Schr\"{o}dinger's equation then transforms into a discretized version as follows:

\begin{align*}
  -\frac{1}{2}\left[\frac{\Phi_n(x + \Delta x) - 2\Phi_n(x) + \Phi_n(x - \Delta x)}{\Delta x^2}\right] + v(x)\Phi_n(x) &= {E_n}\Phi_n(x)
\end{align*}

This solution relies on a step size $\Delta x$ with the second derivative approximation requiring a previous value ($x - \Delta x$) and the next value ($x + \Delta x$) along with the current value $x$. By taking into account the entire well, sliced into $N$ segments of size $\Delta x$, each segment will represent a different set of the discretized Schr\"{o}dinger's equation to solve, which together forms a linear set of equations which can be solved with the help of linear algebra. For a discretization of $N=10$, we have the following matrices involved:

The kinetic energy matrix:

\begin{align*}
\bm{T_{N=10}} = -\frac{1}{2\Delta x^2}
\begin{bmatrix}
-2 & 1 & 0 & 0 & 0 & 0 & 0 & 0 & 0 & 0\\
 1 &-2 & 1 & 0 & 0 & 0 & 0 & 0 & 0 & 0\\
 0 & 1 &-2 & 1 & 0 & 0 & 0 & 0 & 0 & 0\\
 0 & 0 & 1 &-2 & 1 & 0 & 0 & 0 & 0 & 0\\
 0 & 0 & 0 & 1 &-2 & 1 & 0 & 0 & 0 & 0\\
 0 & 0 & 0 & 0 & 1 &-2 & 1 & 0 & 0 & 0\\
 0 & 0 & 0 & 0 & 0 & 1 &-2 & 1 & 0 & 0\\
 0 & 0 & 0 & 0 & 0 & 0 & 1 &-2 & 1 & 0\\
 0 & 0 & 0 & 0 & 0 & 0 & 0 & 1 &-2 & 1\\
 0 & 0 & 0 & 0 & 0 & 0 & 0 & 0 & 1 &-2\\
\end{bmatrix}
\end{align*}

And the potential energy matrix:

\begin{align*}
\bm{V_{N=10}} =
\begin{bmatrix}
 v(x) & 0 & 0 & 0 & 0 & 0 & 0 & 0 & 0 & 0\\
 0 & v(x) & 0 & 0 & 0 & 0 & 0 & 0 & 0 & 0\\
 0 & 0 & v(x) & 0 & 0 & 0 & 0 & 0 & 0 & 0\\
 0 & 0 & 0 & v(x) & 0 & 0 & 0 & 0 & 0 & 0\\
 0 & 0 & 0 & 0 & v(x) & 0 & 0 & 0 & 0 & 0\\
 0 & 0 & 0 & 0 & 0 & v(x) & 0 & 0 & 0 & 0\\
 0 & 0 & 0 & 0 & 0 & 0 & v(x) & 0 & 0 & 0\\
 0 & 0 & 0 & 0 & 0 & 0 & 0 & v(x) & 0 & 0\\
 0 & 0 & 0 & 0 & 0 & 0 & 0 & 0 & v(x) & 0\\
 0 & 0 & 0 & 0 & 0 & 0 & 0 & 0 & 0 & v(x)\\
\end{bmatrix}
\end{align*}

where

\begin{align*}
  v(x) &= \begin{cases}
          0       \quad \, & |x| < 5a \\
          \infty  \quad \, & |x| > 5a \\
     \end{cases}
\end{align*}

Which results in the following system to solve for:

\begin{align*}
\bm{T}\ket{\Phi} + \bm{V}\ket{\Phi} = \bm{E}\ket{\Phi}
\end{align*}

or simply:

\begin{align*}
\bm{H}\ket{\Phi} = \bm{E}\ket{\Phi}
\end{align*}

where $\ket{\Phi}$ is the column vector representing each step of the wave function's discretization:

\begin{align*}
\ket{\Phi} =
\begin{bmatrix}
 \Phi_1 \\
 \Phi_2 \\ 
 \Phi_3 \\ 
 \Phi_4 \\ 
 \Phi_5 \\ 
 \Phi_6 \\ 
 \Phi_7 \\ 
 \Phi_8 \\ 
 \Phi_9 \\ 
 \Phi_{10} \\ 
\end{bmatrix}
\end{align*}

and $\bm{H}$ is the Hamiltonian matrix composed of $\bm{T} + \bm{V}$. By diagonalizing $\bm{H}$, we will obtain the eigenvectors as well as the eigenvalues of the system. The eigenvalues will provide the energies and the eigenvectors will provide the associated functions for the discretization chosen for the problem.

When simulating using $N=10$, we obtain the following wavefunctions and energies:

\begin{figure}[H]
  \begin{center}
    %% Creator: Matplotlib, PGF backend
%%
%% To include the figure in your LaTeX document, write
%%   \input{<filename>.pgf}
%%
%% Make sure the required packages are loaded in your preamble
%%   \usepackage{pgf}
%%
%% Also ensure that all the required font packages are loaded; for instance,
%% the lmodern package is sometimes necessary when using math font.
%%   \usepackage{lmodern}
%%
%% Figures using additional raster images can only be included by \input if
%% they are in the same directory as the main LaTeX file. For loading figures
%% from other directories you can use the `import` package
%%   \usepackage{import}
%%
%% and then include the figures with
%%   \import{<path to file>}{<filename>.pgf}
%%
%% Matplotlib used the following preamble
%%
\begingroup%
\makeatletter%
\begin{pgfpicture}%
\pgfpathrectangle{\pgfpointorigin}{\pgfqpoint{3.000000in}{6.000000in}}%
\pgfusepath{use as bounding box, clip}%
\begin{pgfscope}%
\pgfsetbuttcap%
\pgfsetmiterjoin%
\definecolor{currentfill}{rgb}{1.000000,1.000000,1.000000}%
\pgfsetfillcolor{currentfill}%
\pgfsetlinewidth{0.000000pt}%
\definecolor{currentstroke}{rgb}{1.000000,1.000000,1.000000}%
\pgfsetstrokecolor{currentstroke}%
\pgfsetdash{}{0pt}%
\pgfpathmoveto{\pgfqpoint{0.000000in}{0.000000in}}%
\pgfpathlineto{\pgfqpoint{3.000000in}{0.000000in}}%
\pgfpathlineto{\pgfqpoint{3.000000in}{6.000000in}}%
\pgfpathlineto{\pgfqpoint{0.000000in}{6.000000in}}%
\pgfpathlineto{\pgfqpoint{0.000000in}{0.000000in}}%
\pgfpathclose%
\pgfusepath{fill}%
\end{pgfscope}%
\begin{pgfscope}%
\pgfsetbuttcap%
\pgfsetmiterjoin%
\definecolor{currentfill}{rgb}{1.000000,1.000000,1.000000}%
\pgfsetfillcolor{currentfill}%
\pgfsetlinewidth{0.000000pt}%
\definecolor{currentstroke}{rgb}{0.000000,0.000000,0.000000}%
\pgfsetstrokecolor{currentstroke}%
\pgfsetstrokeopacity{0.000000}%
\pgfsetdash{}{0pt}%
\pgfpathmoveto{\pgfqpoint{0.495679in}{0.549691in}}%
\pgfpathlineto{\pgfqpoint{2.850000in}{0.549691in}}%
\pgfpathlineto{\pgfqpoint{2.850000in}{5.641667in}}%
\pgfpathlineto{\pgfqpoint{0.495679in}{5.641667in}}%
\pgfpathlineto{\pgfqpoint{0.495679in}{0.549691in}}%
\pgfpathclose%
\pgfusepath{fill}%
\end{pgfscope}%
\begin{pgfscope}%
\pgfsetbuttcap%
\pgfsetroundjoin%
\definecolor{currentfill}{rgb}{0.000000,0.000000,0.000000}%
\pgfsetfillcolor{currentfill}%
\pgfsetlinewidth{0.803000pt}%
\definecolor{currentstroke}{rgb}{0.000000,0.000000,0.000000}%
\pgfsetstrokecolor{currentstroke}%
\pgfsetdash{}{0pt}%
\pgfsys@defobject{currentmarker}{\pgfqpoint{0.000000in}{-0.048611in}}{\pgfqpoint{0.000000in}{0.000000in}}{%
\pgfpathmoveto{\pgfqpoint{0.000000in}{0.000000in}}%
\pgfpathlineto{\pgfqpoint{0.000000in}{-0.048611in}}%
\pgfusepath{stroke,fill}%
}%
\begin{pgfscope}%
\pgfsys@transformshift{0.731111in}{0.549691in}%
\pgfsys@useobject{currentmarker}{}%
\end{pgfscope}%
\end{pgfscope}%
\begin{pgfscope}%
\definecolor{textcolor}{rgb}{0.000000,0.000000,0.000000}%
\pgfsetstrokecolor{textcolor}%
\pgfsetfillcolor{textcolor}%
\pgftext[x=0.731111in,y=0.452469in,,top]{\color{textcolor}\rmfamily\fontsize{10.000000}{12.000000}\selectfont \(\displaystyle {\ensuremath{-}4}\)}%
\end{pgfscope}%
\begin{pgfscope}%
\pgfsetbuttcap%
\pgfsetroundjoin%
\definecolor{currentfill}{rgb}{0.000000,0.000000,0.000000}%
\pgfsetfillcolor{currentfill}%
\pgfsetlinewidth{0.803000pt}%
\definecolor{currentstroke}{rgb}{0.000000,0.000000,0.000000}%
\pgfsetstrokecolor{currentstroke}%
\pgfsetdash{}{0pt}%
\pgfsys@defobject{currentmarker}{\pgfqpoint{0.000000in}{-0.048611in}}{\pgfqpoint{0.000000in}{0.000000in}}{%
\pgfpathmoveto{\pgfqpoint{0.000000in}{0.000000in}}%
\pgfpathlineto{\pgfqpoint{0.000000in}{-0.048611in}}%
\pgfusepath{stroke,fill}%
}%
\begin{pgfscope}%
\pgfsys@transformshift{1.201975in}{0.549691in}%
\pgfsys@useobject{currentmarker}{}%
\end{pgfscope}%
\end{pgfscope}%
\begin{pgfscope}%
\definecolor{textcolor}{rgb}{0.000000,0.000000,0.000000}%
\pgfsetstrokecolor{textcolor}%
\pgfsetfillcolor{textcolor}%
\pgftext[x=1.201975in,y=0.452469in,,top]{\color{textcolor}\rmfamily\fontsize{10.000000}{12.000000}\selectfont \(\displaystyle {\ensuremath{-}2}\)}%
\end{pgfscope}%
\begin{pgfscope}%
\pgfsetbuttcap%
\pgfsetroundjoin%
\definecolor{currentfill}{rgb}{0.000000,0.000000,0.000000}%
\pgfsetfillcolor{currentfill}%
\pgfsetlinewidth{0.803000pt}%
\definecolor{currentstroke}{rgb}{0.000000,0.000000,0.000000}%
\pgfsetstrokecolor{currentstroke}%
\pgfsetdash{}{0pt}%
\pgfsys@defobject{currentmarker}{\pgfqpoint{0.000000in}{-0.048611in}}{\pgfqpoint{0.000000in}{0.000000in}}{%
\pgfpathmoveto{\pgfqpoint{0.000000in}{0.000000in}}%
\pgfpathlineto{\pgfqpoint{0.000000in}{-0.048611in}}%
\pgfusepath{stroke,fill}%
}%
\begin{pgfscope}%
\pgfsys@transformshift{1.672840in}{0.549691in}%
\pgfsys@useobject{currentmarker}{}%
\end{pgfscope}%
\end{pgfscope}%
\begin{pgfscope}%
\definecolor{textcolor}{rgb}{0.000000,0.000000,0.000000}%
\pgfsetstrokecolor{textcolor}%
\pgfsetfillcolor{textcolor}%
\pgftext[x=1.672840in,y=0.452469in,,top]{\color{textcolor}\rmfamily\fontsize{10.000000}{12.000000}\selectfont \(\displaystyle {0}\)}%
\end{pgfscope}%
\begin{pgfscope}%
\pgfsetbuttcap%
\pgfsetroundjoin%
\definecolor{currentfill}{rgb}{0.000000,0.000000,0.000000}%
\pgfsetfillcolor{currentfill}%
\pgfsetlinewidth{0.803000pt}%
\definecolor{currentstroke}{rgb}{0.000000,0.000000,0.000000}%
\pgfsetstrokecolor{currentstroke}%
\pgfsetdash{}{0pt}%
\pgfsys@defobject{currentmarker}{\pgfqpoint{0.000000in}{-0.048611in}}{\pgfqpoint{0.000000in}{0.000000in}}{%
\pgfpathmoveto{\pgfqpoint{0.000000in}{0.000000in}}%
\pgfpathlineto{\pgfqpoint{0.000000in}{-0.048611in}}%
\pgfusepath{stroke,fill}%
}%
\begin{pgfscope}%
\pgfsys@transformshift{2.143704in}{0.549691in}%
\pgfsys@useobject{currentmarker}{}%
\end{pgfscope}%
\end{pgfscope}%
\begin{pgfscope}%
\definecolor{textcolor}{rgb}{0.000000,0.000000,0.000000}%
\pgfsetstrokecolor{textcolor}%
\pgfsetfillcolor{textcolor}%
\pgftext[x=2.143704in,y=0.452469in,,top]{\color{textcolor}\rmfamily\fontsize{10.000000}{12.000000}\selectfont \(\displaystyle {2}\)}%
\end{pgfscope}%
\begin{pgfscope}%
\pgfsetbuttcap%
\pgfsetroundjoin%
\definecolor{currentfill}{rgb}{0.000000,0.000000,0.000000}%
\pgfsetfillcolor{currentfill}%
\pgfsetlinewidth{0.803000pt}%
\definecolor{currentstroke}{rgb}{0.000000,0.000000,0.000000}%
\pgfsetstrokecolor{currentstroke}%
\pgfsetdash{}{0pt}%
\pgfsys@defobject{currentmarker}{\pgfqpoint{0.000000in}{-0.048611in}}{\pgfqpoint{0.000000in}{0.000000in}}{%
\pgfpathmoveto{\pgfqpoint{0.000000in}{0.000000in}}%
\pgfpathlineto{\pgfqpoint{0.000000in}{-0.048611in}}%
\pgfusepath{stroke,fill}%
}%
\begin{pgfscope}%
\pgfsys@transformshift{2.614568in}{0.549691in}%
\pgfsys@useobject{currentmarker}{}%
\end{pgfscope}%
\end{pgfscope}%
\begin{pgfscope}%
\definecolor{textcolor}{rgb}{0.000000,0.000000,0.000000}%
\pgfsetstrokecolor{textcolor}%
\pgfsetfillcolor{textcolor}%
\pgftext[x=2.614568in,y=0.452469in,,top]{\color{textcolor}\rmfamily\fontsize{10.000000}{12.000000}\selectfont \(\displaystyle {4}\)}%
\end{pgfscope}%
\begin{pgfscope}%
\definecolor{textcolor}{rgb}{0.000000,0.000000,0.000000}%
\pgfsetstrokecolor{textcolor}%
\pgfsetfillcolor{textcolor}%
\pgftext[x=1.672840in,y=0.273457in,,top]{\color{textcolor}\rmfamily\fontsize{10.000000}{12.000000}\selectfont \(\displaystyle x\)}%
\end{pgfscope}%
\begin{pgfscope}%
\pgfsetbuttcap%
\pgfsetroundjoin%
\definecolor{currentfill}{rgb}{0.000000,0.000000,0.000000}%
\pgfsetfillcolor{currentfill}%
\pgfsetlinewidth{0.803000pt}%
\definecolor{currentstroke}{rgb}{0.000000,0.000000,0.000000}%
\pgfsetstrokecolor{currentstroke}%
\pgfsetdash{}{0pt}%
\pgfsys@defobject{currentmarker}{\pgfqpoint{-0.048611in}{0.000000in}}{\pgfqpoint{-0.000000in}{0.000000in}}{%
\pgfpathmoveto{\pgfqpoint{-0.000000in}{0.000000in}}%
\pgfpathlineto{\pgfqpoint{-0.048611in}{0.000000in}}%
\pgfusepath{stroke,fill}%
}%
\begin{pgfscope}%
\pgfsys@transformshift{0.495679in}{0.974022in}%
\pgfsys@useobject{currentmarker}{}%
\end{pgfscope}%
\end{pgfscope}%
\begin{pgfscope}%
\definecolor{textcolor}{rgb}{0.000000,0.000000,0.000000}%
\pgfsetstrokecolor{textcolor}%
\pgfsetfillcolor{textcolor}%
\pgftext[x=0.329012in, y=0.925797in, left, base]{\color{textcolor}\rmfamily\fontsize{10.000000}{12.000000}\selectfont \(\displaystyle {1}\)}%
\end{pgfscope}%
\begin{pgfscope}%
\pgfsetbuttcap%
\pgfsetroundjoin%
\definecolor{currentfill}{rgb}{0.000000,0.000000,0.000000}%
\pgfsetfillcolor{currentfill}%
\pgfsetlinewidth{0.803000pt}%
\definecolor{currentstroke}{rgb}{0.000000,0.000000,0.000000}%
\pgfsetstrokecolor{currentstroke}%
\pgfsetdash{}{0pt}%
\pgfsys@defobject{currentmarker}{\pgfqpoint{-0.048611in}{0.000000in}}{\pgfqpoint{-0.000000in}{0.000000in}}{%
\pgfpathmoveto{\pgfqpoint{-0.000000in}{0.000000in}}%
\pgfpathlineto{\pgfqpoint{-0.048611in}{0.000000in}}%
\pgfusepath{stroke,fill}%
}%
\begin{pgfscope}%
\pgfsys@transformshift{0.495679in}{1.822685in}%
\pgfsys@useobject{currentmarker}{}%
\end{pgfscope}%
\end{pgfscope}%
\begin{pgfscope}%
\definecolor{textcolor}{rgb}{0.000000,0.000000,0.000000}%
\pgfsetstrokecolor{textcolor}%
\pgfsetfillcolor{textcolor}%
\pgftext[x=0.329012in, y=1.774460in, left, base]{\color{textcolor}\rmfamily\fontsize{10.000000}{12.000000}\selectfont \(\displaystyle {2}\)}%
\end{pgfscope}%
\begin{pgfscope}%
\pgfsetbuttcap%
\pgfsetroundjoin%
\definecolor{currentfill}{rgb}{0.000000,0.000000,0.000000}%
\pgfsetfillcolor{currentfill}%
\pgfsetlinewidth{0.803000pt}%
\definecolor{currentstroke}{rgb}{0.000000,0.000000,0.000000}%
\pgfsetstrokecolor{currentstroke}%
\pgfsetdash{}{0pt}%
\pgfsys@defobject{currentmarker}{\pgfqpoint{-0.048611in}{0.000000in}}{\pgfqpoint{-0.000000in}{0.000000in}}{%
\pgfpathmoveto{\pgfqpoint{-0.000000in}{0.000000in}}%
\pgfpathlineto{\pgfqpoint{-0.048611in}{0.000000in}}%
\pgfusepath{stroke,fill}%
}%
\begin{pgfscope}%
\pgfsys@transformshift{0.495679in}{2.671348in}%
\pgfsys@useobject{currentmarker}{}%
\end{pgfscope}%
\end{pgfscope}%
\begin{pgfscope}%
\definecolor{textcolor}{rgb}{0.000000,0.000000,0.000000}%
\pgfsetstrokecolor{textcolor}%
\pgfsetfillcolor{textcolor}%
\pgftext[x=0.329012in, y=2.623122in, left, base]{\color{textcolor}\rmfamily\fontsize{10.000000}{12.000000}\selectfont \(\displaystyle {3}\)}%
\end{pgfscope}%
\begin{pgfscope}%
\pgfsetbuttcap%
\pgfsetroundjoin%
\definecolor{currentfill}{rgb}{0.000000,0.000000,0.000000}%
\pgfsetfillcolor{currentfill}%
\pgfsetlinewidth{0.803000pt}%
\definecolor{currentstroke}{rgb}{0.000000,0.000000,0.000000}%
\pgfsetstrokecolor{currentstroke}%
\pgfsetdash{}{0pt}%
\pgfsys@defobject{currentmarker}{\pgfqpoint{-0.048611in}{0.000000in}}{\pgfqpoint{-0.000000in}{0.000000in}}{%
\pgfpathmoveto{\pgfqpoint{-0.000000in}{0.000000in}}%
\pgfpathlineto{\pgfqpoint{-0.048611in}{0.000000in}}%
\pgfusepath{stroke,fill}%
}%
\begin{pgfscope}%
\pgfsys@transformshift{0.495679in}{3.520010in}%
\pgfsys@useobject{currentmarker}{}%
\end{pgfscope}%
\end{pgfscope}%
\begin{pgfscope}%
\definecolor{textcolor}{rgb}{0.000000,0.000000,0.000000}%
\pgfsetstrokecolor{textcolor}%
\pgfsetfillcolor{textcolor}%
\pgftext[x=0.329012in, y=3.471785in, left, base]{\color{textcolor}\rmfamily\fontsize{10.000000}{12.000000}\selectfont \(\displaystyle {4}\)}%
\end{pgfscope}%
\begin{pgfscope}%
\pgfsetbuttcap%
\pgfsetroundjoin%
\definecolor{currentfill}{rgb}{0.000000,0.000000,0.000000}%
\pgfsetfillcolor{currentfill}%
\pgfsetlinewidth{0.803000pt}%
\definecolor{currentstroke}{rgb}{0.000000,0.000000,0.000000}%
\pgfsetstrokecolor{currentstroke}%
\pgfsetdash{}{0pt}%
\pgfsys@defobject{currentmarker}{\pgfqpoint{-0.048611in}{0.000000in}}{\pgfqpoint{-0.000000in}{0.000000in}}{%
\pgfpathmoveto{\pgfqpoint{-0.000000in}{0.000000in}}%
\pgfpathlineto{\pgfqpoint{-0.048611in}{0.000000in}}%
\pgfusepath{stroke,fill}%
}%
\begin{pgfscope}%
\pgfsys@transformshift{0.495679in}{4.368673in}%
\pgfsys@useobject{currentmarker}{}%
\end{pgfscope}%
\end{pgfscope}%
\begin{pgfscope}%
\definecolor{textcolor}{rgb}{0.000000,0.000000,0.000000}%
\pgfsetstrokecolor{textcolor}%
\pgfsetfillcolor{textcolor}%
\pgftext[x=0.329012in, y=4.320448in, left, base]{\color{textcolor}\rmfamily\fontsize{10.000000}{12.000000}\selectfont \(\displaystyle {5}\)}%
\end{pgfscope}%
\begin{pgfscope}%
\pgfsetbuttcap%
\pgfsetroundjoin%
\definecolor{currentfill}{rgb}{0.000000,0.000000,0.000000}%
\pgfsetfillcolor{currentfill}%
\pgfsetlinewidth{0.803000pt}%
\definecolor{currentstroke}{rgb}{0.000000,0.000000,0.000000}%
\pgfsetstrokecolor{currentstroke}%
\pgfsetdash{}{0pt}%
\pgfsys@defobject{currentmarker}{\pgfqpoint{-0.048611in}{0.000000in}}{\pgfqpoint{-0.000000in}{0.000000in}}{%
\pgfpathmoveto{\pgfqpoint{-0.000000in}{0.000000in}}%
\pgfpathlineto{\pgfqpoint{-0.048611in}{0.000000in}}%
\pgfusepath{stroke,fill}%
}%
\begin{pgfscope}%
\pgfsys@transformshift{0.495679in}{5.217335in}%
\pgfsys@useobject{currentmarker}{}%
\end{pgfscope}%
\end{pgfscope}%
\begin{pgfscope}%
\definecolor{textcolor}{rgb}{0.000000,0.000000,0.000000}%
\pgfsetstrokecolor{textcolor}%
\pgfsetfillcolor{textcolor}%
\pgftext[x=0.329012in, y=5.169110in, left, base]{\color{textcolor}\rmfamily\fontsize{10.000000}{12.000000}\selectfont \(\displaystyle {6}\)}%
\end{pgfscope}%
\begin{pgfscope}%
\definecolor{textcolor}{rgb}{0.000000,0.000000,0.000000}%
\pgfsetstrokecolor{textcolor}%
\pgfsetfillcolor{textcolor}%
\pgftext[x=0.273457in,y=3.095679in,,bottom,rotate=90.000000]{\color{textcolor}\rmfamily\fontsize{10.000000}{12.000000}\selectfont \(\displaystyle n\)}%
\end{pgfscope}%
\begin{pgfscope}%
\pgfpathrectangle{\pgfqpoint{0.495679in}{0.549691in}}{\pgfqpoint{2.354321in}{5.091976in}}%
\pgfusepath{clip}%
\pgfsetbuttcap%
\pgfsetroundjoin%
\pgfsetlinewidth{0.501875pt}%
\definecolor{currentstroke}{rgb}{0.121569,0.466667,0.705882}%
\pgfsetstrokecolor{currentstroke}%
\pgfsetdash{{1.850000pt}{0.800000pt}}{0.000000pt}%
\pgfpathmoveto{\pgfqpoint{1.672840in}{0.549691in}}%
\pgfpathlineto{\pgfqpoint{1.672840in}{5.641667in}}%
\pgfusepath{stroke}%
\end{pgfscope}%
\begin{pgfscope}%
\pgfpathrectangle{\pgfqpoint{0.495679in}{0.549691in}}{\pgfqpoint{2.354321in}{5.091976in}}%
\pgfusepath{clip}%
\pgfsetrectcap%
\pgfsetroundjoin%
\pgfsetlinewidth{1.505625pt}%
\definecolor{currentstroke}{rgb}{0.121569,0.466667,0.705882}%
\pgfsetstrokecolor{currentstroke}%
\pgfsetdash{}{0pt}%
\pgfpathmoveto{\pgfqpoint{0.495679in}{0.872072in}}%
\pgfpathlineto{\pgfqpoint{0.757270in}{0.778380in}}%
\pgfpathlineto{\pgfqpoint{1.018862in}{0.700539in}}%
\pgfpathlineto{\pgfqpoint{1.280453in}{0.644853in}}%
\pgfpathlineto{\pgfqpoint{1.542044in}{0.615835in}}%
\pgfpathlineto{\pgfqpoint{1.803635in}{0.615835in}}%
\pgfpathlineto{\pgfqpoint{2.065226in}{0.644853in}}%
\pgfpathlineto{\pgfqpoint{2.326818in}{0.700539in}}%
\pgfpathlineto{\pgfqpoint{2.588409in}{0.778380in}}%
\pgfpathlineto{\pgfqpoint{2.850000in}{0.872072in}}%
\pgfusepath{stroke}%
\end{pgfscope}%
\begin{pgfscope}%
\pgfpathrectangle{\pgfqpoint{0.495679in}{0.549691in}}{\pgfqpoint{2.354321in}{5.091976in}}%
\pgfusepath{clip}%
\pgfsetbuttcap%
\pgfsetroundjoin%
\pgfsetlinewidth{0.501875pt}%
\definecolor{currentstroke}{rgb}{0.121569,0.466667,0.705882}%
\pgfsetstrokecolor{currentstroke}%
\pgfsetdash{{1.850000pt}{0.800000pt}}{0.000000pt}%
\pgfpathmoveto{\pgfqpoint{0.495679in}{0.974022in}}%
\pgfpathlineto{\pgfqpoint{2.850000in}{0.974022in}}%
\pgfusepath{stroke}%
\end{pgfscope}%
\begin{pgfscope}%
\pgfpathrectangle{\pgfqpoint{0.495679in}{0.549691in}}{\pgfqpoint{2.354321in}{5.091976in}}%
\pgfusepath{clip}%
\pgfsetrectcap%
\pgfsetroundjoin%
\pgfsetlinewidth{1.505625pt}%
\definecolor{currentstroke}{rgb}{1.000000,0.498039,0.054902}%
\pgfsetstrokecolor{currentstroke}%
\pgfsetdash{}{0pt}%
\pgfpathmoveto{\pgfqpoint{0.495679in}{2.018327in}}%
\pgfpathlineto{\pgfqpoint{0.757270in}{2.151854in}}%
\pgfpathlineto{\pgfqpoint{1.018862in}{2.180873in}}%
\pgfpathlineto{\pgfqpoint{1.280453in}{2.096169in}}%
\pgfpathlineto{\pgfqpoint{1.542044in}{1.924636in}}%
\pgfpathlineto{\pgfqpoint{1.803635in}{1.720734in}}%
\pgfpathlineto{\pgfqpoint{2.065226in}{1.549201in}}%
\pgfpathlineto{\pgfqpoint{2.326818in}{1.464497in}}%
\pgfpathlineto{\pgfqpoint{2.588409in}{1.493516in}}%
\pgfpathlineto{\pgfqpoint{2.850000in}{1.627043in}}%
\pgfusepath{stroke}%
\end{pgfscope}%
\begin{pgfscope}%
\pgfpathrectangle{\pgfqpoint{0.495679in}{0.549691in}}{\pgfqpoint{2.354321in}{5.091976in}}%
\pgfusepath{clip}%
\pgfsetbuttcap%
\pgfsetroundjoin%
\pgfsetlinewidth{0.501875pt}%
\definecolor{currentstroke}{rgb}{0.121569,0.466667,0.705882}%
\pgfsetstrokecolor{currentstroke}%
\pgfsetdash{{1.850000pt}{0.800000pt}}{0.000000pt}%
\pgfpathmoveto{\pgfqpoint{0.495679in}{1.822685in}}%
\pgfpathlineto{\pgfqpoint{2.850000in}{1.822685in}}%
\pgfusepath{stroke}%
\end{pgfscope}%
\begin{pgfscope}%
\pgfpathrectangle{\pgfqpoint{0.495679in}{0.549691in}}{\pgfqpoint{2.354321in}{5.091976in}}%
\pgfusepath{clip}%
\pgfsetrectcap%
\pgfsetroundjoin%
\pgfsetlinewidth{1.505625pt}%
\definecolor{currentstroke}{rgb}{0.172549,0.627451,0.172549}%
\pgfsetstrokecolor{currentstroke}%
\pgfsetdash{}{0pt}%
\pgfpathmoveto{\pgfqpoint{0.495679in}{2.944831in}}%
\pgfpathlineto{\pgfqpoint{0.757270in}{3.029535in}}%
\pgfpathlineto{\pgfqpoint{1.018862in}{2.866990in}}%
\pgfpathlineto{\pgfqpoint{1.280453in}{2.569397in}}%
\pgfpathlineto{\pgfqpoint{1.542044in}{2.342178in}}%
\pgfpathlineto{\pgfqpoint{1.803635in}{2.342178in}}%
\pgfpathlineto{\pgfqpoint{2.065226in}{2.569397in}}%
\pgfpathlineto{\pgfqpoint{2.326818in}{2.866990in}}%
\pgfpathlineto{\pgfqpoint{2.588409in}{3.029535in}}%
\pgfpathlineto{\pgfqpoint{2.850000in}{2.944831in}}%
\pgfusepath{stroke}%
\end{pgfscope}%
\begin{pgfscope}%
\pgfpathrectangle{\pgfqpoint{0.495679in}{0.549691in}}{\pgfqpoint{2.354321in}{5.091976in}}%
\pgfusepath{clip}%
\pgfsetbuttcap%
\pgfsetroundjoin%
\pgfsetlinewidth{0.501875pt}%
\definecolor{currentstroke}{rgb}{0.121569,0.466667,0.705882}%
\pgfsetstrokecolor{currentstroke}%
\pgfsetdash{{1.850000pt}{0.800000pt}}{0.000000pt}%
\pgfpathmoveto{\pgfqpoint{0.495679in}{2.671348in}}%
\pgfpathlineto{\pgfqpoint{2.850000in}{2.671348in}}%
\pgfusepath{stroke}%
\end{pgfscope}%
\begin{pgfscope}%
\pgfpathrectangle{\pgfqpoint{0.495679in}{0.549691in}}{\pgfqpoint{2.354321in}{5.091976in}}%
\pgfusepath{clip}%
\pgfsetrectcap%
\pgfsetroundjoin%
\pgfsetlinewidth{1.505625pt}%
\definecolor{currentstroke}{rgb}{0.839216,0.152941,0.156863}%
\pgfsetstrokecolor{currentstroke}%
\pgfsetdash{}{0pt}%
\pgfpathmoveto{\pgfqpoint{0.495679in}{3.849180in}}%
\pgfpathlineto{\pgfqpoint{0.757270in}{3.793494in}}%
\pgfpathlineto{\pgfqpoint{1.018862in}{3.418059in}}%
\pgfpathlineto{\pgfqpoint{1.280453in}{3.161823in}}%
\pgfpathlineto{\pgfqpoint{1.542044in}{3.324368in}}%
\pgfpathlineto{\pgfqpoint{1.803635in}{3.715652in}}%
\pgfpathlineto{\pgfqpoint{2.065226in}{3.878198in}}%
\pgfpathlineto{\pgfqpoint{2.326818in}{3.621961in}}%
\pgfpathlineto{\pgfqpoint{2.588409in}{3.246526in}}%
\pgfpathlineto{\pgfqpoint{2.850000in}{3.190841in}}%
\pgfusepath{stroke}%
\end{pgfscope}%
\begin{pgfscope}%
\pgfpathrectangle{\pgfqpoint{0.495679in}{0.549691in}}{\pgfqpoint{2.354321in}{5.091976in}}%
\pgfusepath{clip}%
\pgfsetbuttcap%
\pgfsetroundjoin%
\pgfsetlinewidth{0.501875pt}%
\definecolor{currentstroke}{rgb}{0.121569,0.466667,0.705882}%
\pgfsetstrokecolor{currentstroke}%
\pgfsetdash{{1.850000pt}{0.800000pt}}{0.000000pt}%
\pgfpathmoveto{\pgfqpoint{0.495679in}{3.520010in}}%
\pgfpathlineto{\pgfqpoint{2.850000in}{3.520010in}}%
\pgfusepath{stroke}%
\end{pgfscope}%
\begin{pgfscope}%
\pgfpathrectangle{\pgfqpoint{0.495679in}{0.549691in}}{\pgfqpoint{2.354321in}{5.091976in}}%
\pgfusepath{clip}%
\pgfsetrectcap%
\pgfsetroundjoin%
\pgfsetlinewidth{1.505625pt}%
\definecolor{currentstroke}{rgb}{0.580392,0.403922,0.741176}%
\pgfsetstrokecolor{currentstroke}%
\pgfsetdash{}{0pt}%
\pgfpathmoveto{\pgfqpoint{0.495679in}{4.010485in}}%
\pgfpathlineto{\pgfqpoint{0.757270in}{4.266722in}}%
\pgfpathlineto{\pgfqpoint{1.018862in}{4.697842in}}%
\pgfpathlineto{\pgfqpoint{1.280453in}{4.564315in}}%
\pgfpathlineto{\pgfqpoint{1.542044in}{4.095189in}}%
\pgfpathlineto{\pgfqpoint{1.803635in}{4.095189in}}%
\pgfpathlineto{\pgfqpoint{2.065226in}{4.564315in}}%
\pgfpathlineto{\pgfqpoint{2.326818in}{4.697842in}}%
\pgfpathlineto{\pgfqpoint{2.588409in}{4.266722in}}%
\pgfpathlineto{\pgfqpoint{2.850000in}{4.010485in}}%
\pgfusepath{stroke}%
\end{pgfscope}%
\begin{pgfscope}%
\pgfpathrectangle{\pgfqpoint{0.495679in}{0.549691in}}{\pgfqpoint{2.354321in}{5.091976in}}%
\pgfusepath{clip}%
\pgfsetbuttcap%
\pgfsetroundjoin%
\pgfsetlinewidth{0.501875pt}%
\definecolor{currentstroke}{rgb}{0.121569,0.466667,0.705882}%
\pgfsetstrokecolor{currentstroke}%
\pgfsetdash{{1.850000pt}{0.800000pt}}{0.000000pt}%
\pgfpathmoveto{\pgfqpoint{0.495679in}{4.368673in}}%
\pgfpathlineto{\pgfqpoint{2.850000in}{4.368673in}}%
\pgfusepath{stroke}%
\end{pgfscope}%
\begin{pgfscope}%
\pgfpathrectangle{\pgfqpoint{0.495679in}{0.549691in}}{\pgfqpoint{2.354321in}{5.091976in}}%
\pgfusepath{clip}%
\pgfsetrectcap%
\pgfsetroundjoin%
\pgfsetlinewidth{1.505625pt}%
\definecolor{currentstroke}{rgb}{0.549020,0.337255,0.294118}%
\pgfsetstrokecolor{currentstroke}%
\pgfsetdash{}{0pt}%
\pgfpathmoveto{\pgfqpoint{0.495679in}{4.859148in}}%
\pgfpathlineto{\pgfqpoint{0.757270in}{5.319286in}}%
\pgfpathlineto{\pgfqpoint{1.018862in}{5.546505in}}%
\pgfpathlineto{\pgfqpoint{1.280453in}{5.021693in}}%
\pgfpathlineto{\pgfqpoint{1.542044in}{4.943852in}}%
\pgfpathlineto{\pgfqpoint{1.803635in}{5.490819in}}%
\pgfpathlineto{\pgfqpoint{2.065226in}{5.412978in}}%
\pgfpathlineto{\pgfqpoint{2.326818in}{4.888166in}}%
\pgfpathlineto{\pgfqpoint{2.588409in}{5.115385in}}%
\pgfpathlineto{\pgfqpoint{2.850000in}{5.575523in}}%
\pgfusepath{stroke}%
\end{pgfscope}%
\begin{pgfscope}%
\pgfpathrectangle{\pgfqpoint{0.495679in}{0.549691in}}{\pgfqpoint{2.354321in}{5.091976in}}%
\pgfusepath{clip}%
\pgfsetbuttcap%
\pgfsetroundjoin%
\pgfsetlinewidth{0.501875pt}%
\definecolor{currentstroke}{rgb}{0.121569,0.466667,0.705882}%
\pgfsetstrokecolor{currentstroke}%
\pgfsetdash{{1.850000pt}{0.800000pt}}{0.000000pt}%
\pgfpathmoveto{\pgfqpoint{0.495679in}{5.217335in}}%
\pgfpathlineto{\pgfqpoint{2.850000in}{5.217335in}}%
\pgfusepath{stroke}%
\end{pgfscope}%
\begin{pgfscope}%
\pgfsetrectcap%
\pgfsetmiterjoin%
\pgfsetlinewidth{0.803000pt}%
\definecolor{currentstroke}{rgb}{0.000000,0.000000,0.000000}%
\pgfsetstrokecolor{currentstroke}%
\pgfsetdash{}{0pt}%
\pgfpathmoveto{\pgfqpoint{0.495679in}{0.549691in}}%
\pgfpathlineto{\pgfqpoint{0.495679in}{5.641667in}}%
\pgfusepath{stroke}%
\end{pgfscope}%
\begin{pgfscope}%
\pgfsetrectcap%
\pgfsetmiterjoin%
\pgfsetlinewidth{0.803000pt}%
\definecolor{currentstroke}{rgb}{0.000000,0.000000,0.000000}%
\pgfsetstrokecolor{currentstroke}%
\pgfsetdash{}{0pt}%
\pgfpathmoveto{\pgfqpoint{2.850000in}{0.549691in}}%
\pgfpathlineto{\pgfqpoint{2.850000in}{5.641667in}}%
\pgfusepath{stroke}%
\end{pgfscope}%
\begin{pgfscope}%
\pgfsetrectcap%
\pgfsetmiterjoin%
\pgfsetlinewidth{0.803000pt}%
\definecolor{currentstroke}{rgb}{0.000000,0.000000,0.000000}%
\pgfsetstrokecolor{currentstroke}%
\pgfsetdash{}{0pt}%
\pgfpathmoveto{\pgfqpoint{0.495679in}{0.549691in}}%
\pgfpathlineto{\pgfqpoint{2.850000in}{0.549691in}}%
\pgfusepath{stroke}%
\end{pgfscope}%
\begin{pgfscope}%
\definecolor{textcolor}{rgb}{0.000000,0.000000,0.000000}%
\pgfsetstrokecolor{textcolor}%
\pgfsetfillcolor{textcolor}%
\pgftext[x=1.672840in,y=5.725000in,,base]{\color{textcolor}\rmfamily\fontsize{12.000000}{14.400000}\selectfont \(\displaystyle \Phi_n(x)\)}%
\end{pgfscope}%
\end{pgfpicture}%
\makeatother%
\endgroup%

  \end{center}
  \caption{Wave functions for the numerical solution of the one-dimensional, time-independent Schr\"{o}dinger's describing one electron in an infinitely-deep potential well of width $10a$ and discretization of $N=10$}
  \label{numerical-plot}
\end{figure}

\begin{table}
\begin{center}
\begin{tabular}{l|llllll}\hline
$n$    & $1$    & $2$     & $3$     & $4$      & $5$      & $6$      \\\hline
$E_n$  & $1$  & $3.91898595$  & $8.52047896$  & $14.43169344$  & $21.17373795$  & $28.20041213$ \\\hline
\end{tabular}
\end{center}
  \caption{The first six energy levels calculated for this problem. They roughly match the coefficients that were obtained and listed in Table \ref{analytical-energies}}
  \label{numerical-energies-1}
\end{table}

\subsection{Numerical Solution 2: Two Non-Interacting Electrons}

When an additional, non-interacting electron is introduced, the Hamiltonian changes as follows:

\begin{align*}
  \hat{H} &= -\frac{1}{2}\nabla^{2}_1 -\frac{1}{2}\nabla^{2}_2 + v(x_1) + v(x_2)\\
\end{align*}

A second electron is introduced into the system, meanining that the wave function will also be composed of two coordinates $x_1$ and $x_2$: $\Phi(x_1, x_2)$ The second order differentiation will now have to occur for the wave function with respect to $x_1$ as well as a second time with respect to $x_2$. Therefore, using the work outlined previously, the version of the two, non-interacting electron Schr\"{o}dinger equation is equal to:

\begin{align*}
  -\frac{1}{2}\frac{d^2\Phi(x_1, x_2)}{dx_1^2} + -\frac{1}{2}\frac{d^2\Phi(x_1, x_2)}{dx_2^2} + v(x_1)\Phi(x_1, x_2) + v(x_2)\Phi(x_1, x_2) &= {E}\Phi(x_1, x_2)
  \\ v(x_1) = v(x_2) = v(x) =
  \begin{cases}
          0       \quad \, & |x| < 5a \\
          \infty  \quad \, & |x| > 5a \\
  \end{cases}
\end{align*}

Which discretizes as follows:

\begin{align*}
  -\frac{1}{2}\left[\frac{\Phi(x_1 + \Delta x, x_2) - 2\Phi(x_1, x_2) + \Phi(x_1 - \Delta x, x_2)}{\Delta x^2}\right] + -\frac{1}{2}\left[\frac{\Phi(x1, x_2 + \Delta x) - 2\Phi(x1, x_2) + \Phi(x1, x_2 - \Delta x)}{\Delta x^2}\right] \\ + 2v(x)\Phi(x_1, x_2) = {E}\Phi(x_1, x_2) \\
  -\frac{1}{2}\frac{\Phi(x_1 + \Delta x, x_2) + \Phi(x1, x_2 + \Delta x) - 4\Phi(x_1, x_2) + \Phi(x_1 - \Delta x, x_2) + \Phi(x1, x_2 - \Delta x)}{\Delta x^2} + 2v(x)\Phi(x_1, x_2) = {E}\Phi(x_1, x_2)
\end{align*}

Numerically solving this problem is more complicated than the previous case because now we are solving for two variables: $x_1$ and $x_2$. This bears resemblance to solving the one-dimensional problem in two dimensions. To solve this problem, we construct the wave equation column vector as follows:

\begin{align*}
\ket{\Phi} =
\begin{bmatrix}
\Phi_{0,0}    \\
\Phi_{1,0}    \\
\vdots        \\
\Phi_{N-1,0}  \\ 
\Phi_{N,0}    \\
\Phi_{0,1}    \\
\Phi_{1,1}    \\
\vdots        \\
\Phi_{N-1,1}  \\ 
\Phi_{N,1}    \\
\vdots        \\
\Phi_{0,N}    \\
\Phi_{1,N}    \\
\vdots        \\
\Phi_{N-1,N}  \\
\Phi_{N,N}    \\ 
\end{bmatrix}
\end{align*}

The problem must now take into account the independent positions of the two electrons in the system. This means that we now have $N \times N$ systems of equations to solve for. Using the equation above, the kinetic energy matrix will take the following general form:

\begin{align*}
\left[\scalemath{0.6}{\begin{array}{c c c c c | c c c c c | c c c c c | c | c c c c c | c c c c c | c c c c c}
   -4   &    1   &        &        &        &    1   &        &        &        &        &        &        &        &        &        &        &        &        &        &        &        &        &        &        &        &        &        &        &        &        &        \\
    1   &   -4   &        &        &        &        &    1   &        &        &        &        &        &        &        &        &        &        &        &        &        &        &        &        &        &        &        &        &        &        &        &        \\
        &        & \ddots &        &        &        &        & \ddots &        &        &        &        &        &        &        &        &        &        &        &        &        &        &        &        &        &        &        &        &        &        &        \\
        &        &        &   -4   &    1   &        &        &        &    1   &        &        &        &        &        &        &        &        &        &        &        &        &        &        &        &        &        &        &        &        &        &        \\
        &        &        &    1   &   -4   &        &        &        &        &    1   &        &        &        &        &        &        &        &        &        &        &        &        &        &        &        &        &        &        &        &        &        \\
\hline
    1   &        &        &        &        &   -4   &    1   &        &        &        &    1   &        &        &        &        &        &        &        &        &        &        &        &        &        &        &        &        &        &        &        &        \\
        &    1   &        &        &        &    1   &   -4   &        &        &        &        &    1   &        &        &        &        &        &        &        &        &        &        &        &        &        &        &        &        &        &        &        \\
        &        & \ddots &        &        &        &        & \ddots &        &        &        &        & \ddots &        &        &        &        &        &        &        &        &        &        &        &        &        &        &        &        &        &        \\
        &        &        &    1   &        &        &        &        &   -4   &    1   &        &        &        &    1   &        &        &        &        &        &        &        &        &        &        &        &        &        &        &        &        &        \\
        &        &        &        &    1   &        &        &        &    1   &   -4   &        &        &        &        &    1   &        &        &        &        &        &        &        &        &        &        &        &        &        &        &        &        \\
\hline
        &        &        &        &        &    1   &        &        &        &        &   -4   &    1   &        &        &        &        &        &        &        &        &        &        &        &        &        &        &        &        &        &        &        \\
        &        &        &        &        &        &    1   &        &        &        &    1   &   -4   &        &        &        &        &        &        &        &        &        &        &        &        &        &        &        &        &        &        &        \\
        &        &        &        &        &        &        & \ddots &        &        &        &        & \ddots &        &        &        &        &        &        &        &        &        &        &        &        &        &        &        &        &        &        \\
        &        &        &        &        &        &        &        &    1   &        &        &        &        &   -4   &    1   &        &        &        &        &        &        &        &        &        &        &        &        &        &        &        &        \\
        &        &        &        &        &        &        &        &        &    1   &        &        &        &    1   &   -4   &        &        &        &        &        &        &        &        &        &        &        &        &        &        &        &        \\
\hline
        &        &        &        &        &        &        &        &        &        &        &        &        &        &        & \ddots &        &        &        &        &        &        &        &        &        &        &        &        &        &        &        \\
\hline
        &        &        &        &        &        &        &        &        &        &        &        &        &        &        &        &   -4   &    1   &        &        &        &    1   &        &        &        &        &        &        &        &        &        \\
        &        &        &        &        &        &        &        &        &        &        &        &        &        &        &        &    1   &   -4   &        &        &        &        &    1   &        &        &        &        &        &        &        &        \\
        &        &        &        &        &        &        &        &        &        &        &        &        &        &        &        &        &        & \ddots &        &        &        &        & \ddots &        &        &        &        &        &        &        \\
        &        &        &        &        &        &        &        &        &        &        &        &        &        &        &        &        &        &        &   -4   &    1   &        &        &        &    1   &        &        &        &        &        &        \\
        &        &        &        &        &        &        &        &        &        &        &        &        &        &        &        &        &        &        &    1   &   -4   &        &        &        &        &    1   &        &        &        &        &        \\
\hline
        &        &        &        &        &        &        &        &        &        &        &        &        &        &        &        &    1   &        &        &        &        &   -4   &    1   &        &        &        &    1   &        &        &        &        \\
        &        &        &        &        &        &        &        &        &        &        &        &        &        &        &        &        &    1   &        &        &        &    1   &   -4   &        &        &        &        &    1   &        &        &        \\
        &        &        &        &        &        &        &        &        &        &        &        &        &        &        &        &        &        & \ddots &        &        &        &        & \ddots &        &        &        &        & \ddots &        &        \\
        &        &        &        &        &        &        &        &        &        &        &        &        &        &        &        &        &        &        &    1   &        &        &        &        &   -4   &    1   &        &        &        &    1   &        \\
        &        &        &        &        &        &        &        &        &        &        &        &        &        &        &        &        &        &        &        &    1   &        &        &        &    1   &   -4   &        &        &        &        &    1   \\
\hline
        &        &        &        &        &        &        &        &        &        &        &        &        &        &        &        &        &        &        &        &        &    1   &        &        &        &        &   -4   &    1   &        &        &        \\
        &        &        &        &        &        &        &        &        &        &        &        &        &        &        &        &        &        &        &        &        &        &    1   &        &        &        &    1   &   -4   &        &        &        \\
        &        &        &        &        &        &        &        &        &        &        &        &        &        &        &        &        &        &        &        &        &        &        & \ddots &        &        &        &        & \ddots &        &        \\
        &        &        &        &        &        &        &        &        &        &        &        &        &        &        &        &        &        &        &        &        &        &        &        &    1   &        &        &        &        &   -4   &    1   \\
        &        &        &        &        &        &        &        &        &        &        &        &        &        &        &        &        &        &        &        &        &        &        &        &        &    1   &        &        &        &    1   &   -4   \\
\end{array}}\right]
\end{align*}

This general form is also known as the sparse matrix for the Laplacian differential operator. Obtaining the eigenvalues and the eigenvectors of the system will provide us with the solutions for the equation. Unfortunately, since we now have two electrons with two different possibilities for their current energy levels, it isn't obvious how to discern between the two. For the following numerical answers, the first six energy levels are calculated from lowest to highest, but the specific values for n for each of the electrons is not evident.

\begin{figure}[H]
  \begin{center}
    \input{numerical-solution-2-wave-function-plot.pgf}
  \end{center}
  \caption{Wave functions for the numerical solution of the one-dimensional, time-independent Schr\"{o}dinger's describing two non-interacting electrons in an infinitely-deep potential well of width $10a$ and discretization of $N=25$. Each plot represents two electrons, with the x and y axis corresponding to the position of each electron in the well: $\Phi_n{x_1,x_2}$.}
  \label{numerical-plot-2}
\end{figure}

\begin{table}
\begin{center}
\begin{tabular}{l|llllll}\hline
$n$    & $1$    & $2$     & $3$     & $4$      & $5$      & $6$      \\\hline
$E_n$  & $1$  & $3.97946748$  & $5.26546441$  & $5.27606611$  & $8.17405936$  & $8.27969253$ \\\hline
\end{tabular}
\end{center}
  \caption{The first six energy levels calculated for this problem, normalized to the first level. All of the solution energies were computed and these are the first six to come out of the sorting process. The associated plots are related to these values through their $n$ values, although the $n$ value doesn't correlate exactly to the actual $n$ for each electron.}
  \label{numerical-energies-2}
\end{table}

\subsection{Numerical Solution 3: Two Electrons using Coulomb Interaction}

For this last problem, we will now introduce the Coulomb repulsion between the two electrons in the system. This adds a third term to the Hamiltonian as follows:

\begin{align*}
  \hat{H} &= -\frac{1}{2}\nabla^{2}_1 -\frac{1}{2}\nabla^{2}_2 + \frac{1}{\left|\left|\bf{x_1} - \bf{x_2}\right|\right|} + v(x_1) + v(x_2)\\
\end{align*}

This is straightforward to implement. The magnitude of the distance between the two electrons is computed as follows:

\begin{align*}
  \frac{1}{\left|\left|\bf{x_1} - \bf{x_2}\right|\right|} &= \frac{1}{\sqrt{(x_2 - x_1)^2}}
\end{align*}

In the numerical solution, this is implemented as a matrix with a diagonal, where every element of the diagonal computes the Coulomb repulsion for the given $x_1$ and $x_2$ coordinates of each electron. For each row of the matrix, we know what discretization slice is assigned to it.

\begin{figure}[H]
  \begin{center}
    %% Creator: Matplotlib, PGF backend
%%
%% To include the figure in your LaTeX document, write
%%   \input{<filename>.pgf}
%%
%% Make sure the required packages are loaded in your preamble
%%   \usepackage{pgf}
%%
%% Also ensure that all the required font packages are loaded; for instance,
%% the lmodern package is sometimes necessary when using math font.
%%   \usepackage{lmodern}
%%
%% Figures using additional raster images can only be included by \input if
%% they are in the same directory as the main LaTeX file. For loading figures
%% from other directories you can use the `import` package
%%   \usepackage{import}
%%
%% and then include the figures with
%%   \import{<path to file>}{<filename>.pgf}
%%
%% Matplotlib used the following preamble
%%
\begingroup%
\makeatletter%
\begin{pgfpicture}%
\pgfpathrectangle{\pgfpointorigin}{\pgfqpoint{6.000000in}{4.000000in}}%
\pgfusepath{use as bounding box, clip}%
\begin{pgfscope}%
\pgfsetbuttcap%
\pgfsetmiterjoin%
\definecolor{currentfill}{rgb}{1.000000,1.000000,1.000000}%
\pgfsetfillcolor{currentfill}%
\pgfsetlinewidth{0.000000pt}%
\definecolor{currentstroke}{rgb}{1.000000,1.000000,1.000000}%
\pgfsetstrokecolor{currentstroke}%
\pgfsetdash{}{0pt}%
\pgfpathmoveto{\pgfqpoint{0.000000in}{0.000000in}}%
\pgfpathlineto{\pgfqpoint{6.000000in}{0.000000in}}%
\pgfpathlineto{\pgfqpoint{6.000000in}{4.000000in}}%
\pgfpathlineto{\pgfqpoint{0.000000in}{4.000000in}}%
\pgfpathlineto{\pgfqpoint{0.000000in}{0.000000in}}%
\pgfpathclose%
\pgfusepath{fill}%
\end{pgfscope}%
\begin{pgfscope}%
\pgfsetbuttcap%
\pgfsetmiterjoin%
\definecolor{currentfill}{rgb}{1.000000,1.000000,1.000000}%
\pgfsetfillcolor{currentfill}%
\pgfsetlinewidth{0.000000pt}%
\definecolor{currentstroke}{rgb}{0.000000,0.000000,0.000000}%
\pgfsetstrokecolor{currentstroke}%
\pgfsetstrokeopacity{0.000000}%
\pgfsetdash{}{0pt}%
\pgfpathmoveto{\pgfqpoint{0.993989in}{2.197519in}}%
\pgfpathlineto{\pgfqpoint{2.146906in}{2.197519in}}%
\pgfpathlineto{\pgfqpoint{2.146906in}{3.350436in}}%
\pgfpathlineto{\pgfqpoint{0.993989in}{3.350436in}}%
\pgfpathlineto{\pgfqpoint{0.993989in}{2.197519in}}%
\pgfpathclose%
\pgfusepath{fill}%
\end{pgfscope}%
\begin{pgfscope}%
\pgfpathrectangle{\pgfqpoint{0.993989in}{2.197519in}}{\pgfqpoint{1.152917in}{1.152917in}}%
\pgfusepath{clip}%
\pgfsys@transformcm{1.160000}{0.000000}{0.000000}{1.160000}{0.993989in}{2.197519in}%
\pgftext[left,bottom]{\includegraphics[interpolate=false,width=1.000000in,height=1.000000in]{numerical-solution-3-wave-function-plot-img0.png}}%
\end{pgfscope}%
\begin{pgfscope}%
\pgfsetbuttcap%
\pgfsetroundjoin%
\definecolor{currentfill}{rgb}{0.000000,0.000000,0.000000}%
\pgfsetfillcolor{currentfill}%
\pgfsetlinewidth{0.803000pt}%
\definecolor{currentstroke}{rgb}{0.000000,0.000000,0.000000}%
\pgfsetstrokecolor{currentstroke}%
\pgfsetdash{}{0pt}%
\pgfsys@defobject{currentmarker}{\pgfqpoint{0.000000in}{-0.048611in}}{\pgfqpoint{0.000000in}{0.000000in}}{%
\pgfpathmoveto{\pgfqpoint{0.000000in}{0.000000in}}%
\pgfpathlineto{\pgfqpoint{0.000000in}{-0.048611in}}%
\pgfusepath{stroke,fill}%
}%
\begin{pgfscope}%
\pgfsys@transformshift{0.993989in}{2.197519in}%
\pgfsys@useobject{currentmarker}{}%
\end{pgfscope}%
\end{pgfscope}%
\begin{pgfscope}%
\pgfsetbuttcap%
\pgfsetroundjoin%
\definecolor{currentfill}{rgb}{0.000000,0.000000,0.000000}%
\pgfsetfillcolor{currentfill}%
\pgfsetlinewidth{0.803000pt}%
\definecolor{currentstroke}{rgb}{0.000000,0.000000,0.000000}%
\pgfsetstrokecolor{currentstroke}%
\pgfsetdash{}{0pt}%
\pgfsys@defobject{currentmarker}{\pgfqpoint{0.000000in}{-0.048611in}}{\pgfqpoint{0.000000in}{0.000000in}}{%
\pgfpathmoveto{\pgfqpoint{0.000000in}{0.000000in}}%
\pgfpathlineto{\pgfqpoint{0.000000in}{-0.048611in}}%
\pgfusepath{stroke,fill}%
}%
\begin{pgfscope}%
\pgfsys@transformshift{1.570448in}{2.197519in}%
\pgfsys@useobject{currentmarker}{}%
\end{pgfscope}%
\end{pgfscope}%
\begin{pgfscope}%
\pgfsetbuttcap%
\pgfsetroundjoin%
\definecolor{currentfill}{rgb}{0.000000,0.000000,0.000000}%
\pgfsetfillcolor{currentfill}%
\pgfsetlinewidth{0.803000pt}%
\definecolor{currentstroke}{rgb}{0.000000,0.000000,0.000000}%
\pgfsetstrokecolor{currentstroke}%
\pgfsetdash{}{0pt}%
\pgfsys@defobject{currentmarker}{\pgfqpoint{0.000000in}{-0.048611in}}{\pgfqpoint{0.000000in}{0.000000in}}{%
\pgfpathmoveto{\pgfqpoint{0.000000in}{0.000000in}}%
\pgfpathlineto{\pgfqpoint{0.000000in}{-0.048611in}}%
\pgfusepath{stroke,fill}%
}%
\begin{pgfscope}%
\pgfsys@transformshift{2.146906in}{2.197519in}%
\pgfsys@useobject{currentmarker}{}%
\end{pgfscope}%
\end{pgfscope}%
\begin{pgfscope}%
\pgfsetbuttcap%
\pgfsetroundjoin%
\definecolor{currentfill}{rgb}{0.000000,0.000000,0.000000}%
\pgfsetfillcolor{currentfill}%
\pgfsetlinewidth{0.803000pt}%
\definecolor{currentstroke}{rgb}{0.000000,0.000000,0.000000}%
\pgfsetstrokecolor{currentstroke}%
\pgfsetdash{}{0pt}%
\pgfsys@defobject{currentmarker}{\pgfqpoint{-0.048611in}{0.000000in}}{\pgfqpoint{-0.000000in}{0.000000in}}{%
\pgfpathmoveto{\pgfqpoint{-0.000000in}{0.000000in}}%
\pgfpathlineto{\pgfqpoint{-0.048611in}{0.000000in}}%
\pgfusepath{stroke,fill}%
}%
\begin{pgfscope}%
\pgfsys@transformshift{0.993989in}{2.197519in}%
\pgfsys@useobject{currentmarker}{}%
\end{pgfscope}%
\end{pgfscope}%
\begin{pgfscope}%
\definecolor{textcolor}{rgb}{0.000000,0.000000,0.000000}%
\pgfsetstrokecolor{textcolor}%
\pgfsetfillcolor{textcolor}%
\pgftext[x=0.611272in, y=2.149294in, left, base]{\color{textcolor}\rmfamily\fontsize{10.000000}{12.000000}\selectfont \(\displaystyle {\ensuremath{-}5.0}\)}%
\end{pgfscope}%
\begin{pgfscope}%
\pgfsetbuttcap%
\pgfsetroundjoin%
\definecolor{currentfill}{rgb}{0.000000,0.000000,0.000000}%
\pgfsetfillcolor{currentfill}%
\pgfsetlinewidth{0.803000pt}%
\definecolor{currentstroke}{rgb}{0.000000,0.000000,0.000000}%
\pgfsetstrokecolor{currentstroke}%
\pgfsetdash{}{0pt}%
\pgfsys@defobject{currentmarker}{\pgfqpoint{-0.048611in}{0.000000in}}{\pgfqpoint{-0.000000in}{0.000000in}}{%
\pgfpathmoveto{\pgfqpoint{-0.000000in}{0.000000in}}%
\pgfpathlineto{\pgfqpoint{-0.048611in}{0.000000in}}%
\pgfusepath{stroke,fill}%
}%
\begin{pgfscope}%
\pgfsys@transformshift{0.993989in}{2.485748in}%
\pgfsys@useobject{currentmarker}{}%
\end{pgfscope}%
\end{pgfscope}%
\begin{pgfscope}%
\definecolor{textcolor}{rgb}{0.000000,0.000000,0.000000}%
\pgfsetstrokecolor{textcolor}%
\pgfsetfillcolor{textcolor}%
\pgftext[x=0.611272in, y=2.437523in, left, base]{\color{textcolor}\rmfamily\fontsize{10.000000}{12.000000}\selectfont \(\displaystyle {\ensuremath{-}2.5}\)}%
\end{pgfscope}%
\begin{pgfscope}%
\pgfsetbuttcap%
\pgfsetroundjoin%
\definecolor{currentfill}{rgb}{0.000000,0.000000,0.000000}%
\pgfsetfillcolor{currentfill}%
\pgfsetlinewidth{0.803000pt}%
\definecolor{currentstroke}{rgb}{0.000000,0.000000,0.000000}%
\pgfsetstrokecolor{currentstroke}%
\pgfsetdash{}{0pt}%
\pgfsys@defobject{currentmarker}{\pgfqpoint{-0.048611in}{0.000000in}}{\pgfqpoint{-0.000000in}{0.000000in}}{%
\pgfpathmoveto{\pgfqpoint{-0.000000in}{0.000000in}}%
\pgfpathlineto{\pgfqpoint{-0.048611in}{0.000000in}}%
\pgfusepath{stroke,fill}%
}%
\begin{pgfscope}%
\pgfsys@transformshift{0.993989in}{2.773978in}%
\pgfsys@useobject{currentmarker}{}%
\end{pgfscope}%
\end{pgfscope}%
\begin{pgfscope}%
\definecolor{textcolor}{rgb}{0.000000,0.000000,0.000000}%
\pgfsetstrokecolor{textcolor}%
\pgfsetfillcolor{textcolor}%
\pgftext[x=0.719297in, y=2.725752in, left, base]{\color{textcolor}\rmfamily\fontsize{10.000000}{12.000000}\selectfont \(\displaystyle {0.0}\)}%
\end{pgfscope}%
\begin{pgfscope}%
\pgfsetbuttcap%
\pgfsetroundjoin%
\definecolor{currentfill}{rgb}{0.000000,0.000000,0.000000}%
\pgfsetfillcolor{currentfill}%
\pgfsetlinewidth{0.803000pt}%
\definecolor{currentstroke}{rgb}{0.000000,0.000000,0.000000}%
\pgfsetstrokecolor{currentstroke}%
\pgfsetdash{}{0pt}%
\pgfsys@defobject{currentmarker}{\pgfqpoint{-0.048611in}{0.000000in}}{\pgfqpoint{-0.000000in}{0.000000in}}{%
\pgfpathmoveto{\pgfqpoint{-0.000000in}{0.000000in}}%
\pgfpathlineto{\pgfqpoint{-0.048611in}{0.000000in}}%
\pgfusepath{stroke,fill}%
}%
\begin{pgfscope}%
\pgfsys@transformshift{0.993989in}{3.062207in}%
\pgfsys@useobject{currentmarker}{}%
\end{pgfscope}%
\end{pgfscope}%
\begin{pgfscope}%
\definecolor{textcolor}{rgb}{0.000000,0.000000,0.000000}%
\pgfsetstrokecolor{textcolor}%
\pgfsetfillcolor{textcolor}%
\pgftext[x=0.719297in, y=3.013982in, left, base]{\color{textcolor}\rmfamily\fontsize{10.000000}{12.000000}\selectfont \(\displaystyle {2.5}\)}%
\end{pgfscope}%
\begin{pgfscope}%
\pgfsetbuttcap%
\pgfsetroundjoin%
\definecolor{currentfill}{rgb}{0.000000,0.000000,0.000000}%
\pgfsetfillcolor{currentfill}%
\pgfsetlinewidth{0.803000pt}%
\definecolor{currentstroke}{rgb}{0.000000,0.000000,0.000000}%
\pgfsetstrokecolor{currentstroke}%
\pgfsetdash{}{0pt}%
\pgfsys@defobject{currentmarker}{\pgfqpoint{-0.048611in}{0.000000in}}{\pgfqpoint{-0.000000in}{0.000000in}}{%
\pgfpathmoveto{\pgfqpoint{-0.000000in}{0.000000in}}%
\pgfpathlineto{\pgfqpoint{-0.048611in}{0.000000in}}%
\pgfusepath{stroke,fill}%
}%
\begin{pgfscope}%
\pgfsys@transformshift{0.993989in}{3.350436in}%
\pgfsys@useobject{currentmarker}{}%
\end{pgfscope}%
\end{pgfscope}%
\begin{pgfscope}%
\definecolor{textcolor}{rgb}{0.000000,0.000000,0.000000}%
\pgfsetstrokecolor{textcolor}%
\pgfsetfillcolor{textcolor}%
\pgftext[x=0.719297in, y=3.302211in, left, base]{\color{textcolor}\rmfamily\fontsize{10.000000}{12.000000}\selectfont \(\displaystyle {5.0}\)}%
\end{pgfscope}%
\begin{pgfscope}%
\pgfsetrectcap%
\pgfsetmiterjoin%
\pgfsetlinewidth{0.803000pt}%
\definecolor{currentstroke}{rgb}{0.000000,0.000000,0.000000}%
\pgfsetstrokecolor{currentstroke}%
\pgfsetdash{}{0pt}%
\pgfpathmoveto{\pgfqpoint{0.993989in}{2.197519in}}%
\pgfpathlineto{\pgfqpoint{0.993989in}{3.350436in}}%
\pgfusepath{stroke}%
\end{pgfscope}%
\begin{pgfscope}%
\pgfsetrectcap%
\pgfsetmiterjoin%
\pgfsetlinewidth{0.803000pt}%
\definecolor{currentstroke}{rgb}{0.000000,0.000000,0.000000}%
\pgfsetstrokecolor{currentstroke}%
\pgfsetdash{}{0pt}%
\pgfpathmoveto{\pgfqpoint{2.146906in}{2.197519in}}%
\pgfpathlineto{\pgfqpoint{2.146906in}{3.350436in}}%
\pgfusepath{stroke}%
\end{pgfscope}%
\begin{pgfscope}%
\pgfsetrectcap%
\pgfsetmiterjoin%
\pgfsetlinewidth{0.803000pt}%
\definecolor{currentstroke}{rgb}{0.000000,0.000000,0.000000}%
\pgfsetstrokecolor{currentstroke}%
\pgfsetdash{}{0pt}%
\pgfpathmoveto{\pgfqpoint{0.993989in}{2.197519in}}%
\pgfpathlineto{\pgfqpoint{2.146906in}{2.197519in}}%
\pgfusepath{stroke}%
\end{pgfscope}%
\begin{pgfscope}%
\pgfsetrectcap%
\pgfsetmiterjoin%
\pgfsetlinewidth{0.803000pt}%
\definecolor{currentstroke}{rgb}{0.000000,0.000000,0.000000}%
\pgfsetstrokecolor{currentstroke}%
\pgfsetdash{}{0pt}%
\pgfpathmoveto{\pgfqpoint{0.993989in}{3.350436in}}%
\pgfpathlineto{\pgfqpoint{2.146906in}{3.350436in}}%
\pgfusepath{stroke}%
\end{pgfscope}%
\begin{pgfscope}%
\definecolor{textcolor}{rgb}{0.000000,0.000000,0.000000}%
\pgfsetstrokecolor{textcolor}%
\pgfsetfillcolor{textcolor}%
\pgftext[x=1.570448in,y=3.433769in,,base]{\color{textcolor}\rmfamily\fontsize{12.000000}{14.400000}\selectfont n=1}%
\end{pgfscope}%
\begin{pgfscope}%
\pgfsetbuttcap%
\pgfsetmiterjoin%
\definecolor{currentfill}{rgb}{1.000000,1.000000,1.000000}%
\pgfsetfillcolor{currentfill}%
\pgfsetlinewidth{0.000000pt}%
\definecolor{currentstroke}{rgb}{0.000000,0.000000,0.000000}%
\pgfsetstrokecolor{currentstroke}%
\pgfsetstrokeopacity{0.000000}%
\pgfsetdash{}{0pt}%
\pgfpathmoveto{\pgfqpoint{2.746613in}{2.197519in}}%
\pgfpathlineto{\pgfqpoint{3.899530in}{2.197519in}}%
\pgfpathlineto{\pgfqpoint{3.899530in}{3.350436in}}%
\pgfpathlineto{\pgfqpoint{2.746613in}{3.350436in}}%
\pgfpathlineto{\pgfqpoint{2.746613in}{2.197519in}}%
\pgfpathclose%
\pgfusepath{fill}%
\end{pgfscope}%
\begin{pgfscope}%
\pgfpathrectangle{\pgfqpoint{2.746613in}{2.197519in}}{\pgfqpoint{1.152917in}{1.152917in}}%
\pgfusepath{clip}%
\pgfsys@transformcm{1.160000}{0.000000}{0.000000}{1.160000}{2.746613in}{2.197519in}%
\pgftext[left,bottom]{\includegraphics[interpolate=false,width=1.000000in,height=1.000000in]{numerical-solution-3-wave-function-plot-img1.png}}%
\end{pgfscope}%
\begin{pgfscope}%
\pgfsetbuttcap%
\pgfsetroundjoin%
\definecolor{currentfill}{rgb}{0.000000,0.000000,0.000000}%
\pgfsetfillcolor{currentfill}%
\pgfsetlinewidth{0.803000pt}%
\definecolor{currentstroke}{rgb}{0.000000,0.000000,0.000000}%
\pgfsetstrokecolor{currentstroke}%
\pgfsetdash{}{0pt}%
\pgfsys@defobject{currentmarker}{\pgfqpoint{0.000000in}{-0.048611in}}{\pgfqpoint{0.000000in}{0.000000in}}{%
\pgfpathmoveto{\pgfqpoint{0.000000in}{0.000000in}}%
\pgfpathlineto{\pgfqpoint{0.000000in}{-0.048611in}}%
\pgfusepath{stroke,fill}%
}%
\begin{pgfscope}%
\pgfsys@transformshift{2.746613in}{2.197519in}%
\pgfsys@useobject{currentmarker}{}%
\end{pgfscope}%
\end{pgfscope}%
\begin{pgfscope}%
\pgfsetbuttcap%
\pgfsetroundjoin%
\definecolor{currentfill}{rgb}{0.000000,0.000000,0.000000}%
\pgfsetfillcolor{currentfill}%
\pgfsetlinewidth{0.803000pt}%
\definecolor{currentstroke}{rgb}{0.000000,0.000000,0.000000}%
\pgfsetstrokecolor{currentstroke}%
\pgfsetdash{}{0pt}%
\pgfsys@defobject{currentmarker}{\pgfqpoint{0.000000in}{-0.048611in}}{\pgfqpoint{0.000000in}{0.000000in}}{%
\pgfpathmoveto{\pgfqpoint{0.000000in}{0.000000in}}%
\pgfpathlineto{\pgfqpoint{0.000000in}{-0.048611in}}%
\pgfusepath{stroke,fill}%
}%
\begin{pgfscope}%
\pgfsys@transformshift{3.323071in}{2.197519in}%
\pgfsys@useobject{currentmarker}{}%
\end{pgfscope}%
\end{pgfscope}%
\begin{pgfscope}%
\pgfsetbuttcap%
\pgfsetroundjoin%
\definecolor{currentfill}{rgb}{0.000000,0.000000,0.000000}%
\pgfsetfillcolor{currentfill}%
\pgfsetlinewidth{0.803000pt}%
\definecolor{currentstroke}{rgb}{0.000000,0.000000,0.000000}%
\pgfsetstrokecolor{currentstroke}%
\pgfsetdash{}{0pt}%
\pgfsys@defobject{currentmarker}{\pgfqpoint{0.000000in}{-0.048611in}}{\pgfqpoint{0.000000in}{0.000000in}}{%
\pgfpathmoveto{\pgfqpoint{0.000000in}{0.000000in}}%
\pgfpathlineto{\pgfqpoint{0.000000in}{-0.048611in}}%
\pgfusepath{stroke,fill}%
}%
\begin{pgfscope}%
\pgfsys@transformshift{3.899530in}{2.197519in}%
\pgfsys@useobject{currentmarker}{}%
\end{pgfscope}%
\end{pgfscope}%
\begin{pgfscope}%
\pgfsetbuttcap%
\pgfsetroundjoin%
\definecolor{currentfill}{rgb}{0.000000,0.000000,0.000000}%
\pgfsetfillcolor{currentfill}%
\pgfsetlinewidth{0.803000pt}%
\definecolor{currentstroke}{rgb}{0.000000,0.000000,0.000000}%
\pgfsetstrokecolor{currentstroke}%
\pgfsetdash{}{0pt}%
\pgfsys@defobject{currentmarker}{\pgfqpoint{-0.048611in}{0.000000in}}{\pgfqpoint{-0.000000in}{0.000000in}}{%
\pgfpathmoveto{\pgfqpoint{-0.000000in}{0.000000in}}%
\pgfpathlineto{\pgfqpoint{-0.048611in}{0.000000in}}%
\pgfusepath{stroke,fill}%
}%
\begin{pgfscope}%
\pgfsys@transformshift{2.746613in}{2.197519in}%
\pgfsys@useobject{currentmarker}{}%
\end{pgfscope}%
\end{pgfscope}%
\begin{pgfscope}%
\pgfsetbuttcap%
\pgfsetroundjoin%
\definecolor{currentfill}{rgb}{0.000000,0.000000,0.000000}%
\pgfsetfillcolor{currentfill}%
\pgfsetlinewidth{0.803000pt}%
\definecolor{currentstroke}{rgb}{0.000000,0.000000,0.000000}%
\pgfsetstrokecolor{currentstroke}%
\pgfsetdash{}{0pt}%
\pgfsys@defobject{currentmarker}{\pgfqpoint{-0.048611in}{0.000000in}}{\pgfqpoint{-0.000000in}{0.000000in}}{%
\pgfpathmoveto{\pgfqpoint{-0.000000in}{0.000000in}}%
\pgfpathlineto{\pgfqpoint{-0.048611in}{0.000000in}}%
\pgfusepath{stroke,fill}%
}%
\begin{pgfscope}%
\pgfsys@transformshift{2.746613in}{2.485748in}%
\pgfsys@useobject{currentmarker}{}%
\end{pgfscope}%
\end{pgfscope}%
\begin{pgfscope}%
\pgfsetbuttcap%
\pgfsetroundjoin%
\definecolor{currentfill}{rgb}{0.000000,0.000000,0.000000}%
\pgfsetfillcolor{currentfill}%
\pgfsetlinewidth{0.803000pt}%
\definecolor{currentstroke}{rgb}{0.000000,0.000000,0.000000}%
\pgfsetstrokecolor{currentstroke}%
\pgfsetdash{}{0pt}%
\pgfsys@defobject{currentmarker}{\pgfqpoint{-0.048611in}{0.000000in}}{\pgfqpoint{-0.000000in}{0.000000in}}{%
\pgfpathmoveto{\pgfqpoint{-0.000000in}{0.000000in}}%
\pgfpathlineto{\pgfqpoint{-0.048611in}{0.000000in}}%
\pgfusepath{stroke,fill}%
}%
\begin{pgfscope}%
\pgfsys@transformshift{2.746613in}{2.773978in}%
\pgfsys@useobject{currentmarker}{}%
\end{pgfscope}%
\end{pgfscope}%
\begin{pgfscope}%
\pgfsetbuttcap%
\pgfsetroundjoin%
\definecolor{currentfill}{rgb}{0.000000,0.000000,0.000000}%
\pgfsetfillcolor{currentfill}%
\pgfsetlinewidth{0.803000pt}%
\definecolor{currentstroke}{rgb}{0.000000,0.000000,0.000000}%
\pgfsetstrokecolor{currentstroke}%
\pgfsetdash{}{0pt}%
\pgfsys@defobject{currentmarker}{\pgfqpoint{-0.048611in}{0.000000in}}{\pgfqpoint{-0.000000in}{0.000000in}}{%
\pgfpathmoveto{\pgfqpoint{-0.000000in}{0.000000in}}%
\pgfpathlineto{\pgfqpoint{-0.048611in}{0.000000in}}%
\pgfusepath{stroke,fill}%
}%
\begin{pgfscope}%
\pgfsys@transformshift{2.746613in}{3.062207in}%
\pgfsys@useobject{currentmarker}{}%
\end{pgfscope}%
\end{pgfscope}%
\begin{pgfscope}%
\pgfsetbuttcap%
\pgfsetroundjoin%
\definecolor{currentfill}{rgb}{0.000000,0.000000,0.000000}%
\pgfsetfillcolor{currentfill}%
\pgfsetlinewidth{0.803000pt}%
\definecolor{currentstroke}{rgb}{0.000000,0.000000,0.000000}%
\pgfsetstrokecolor{currentstroke}%
\pgfsetdash{}{0pt}%
\pgfsys@defobject{currentmarker}{\pgfqpoint{-0.048611in}{0.000000in}}{\pgfqpoint{-0.000000in}{0.000000in}}{%
\pgfpathmoveto{\pgfqpoint{-0.000000in}{0.000000in}}%
\pgfpathlineto{\pgfqpoint{-0.048611in}{0.000000in}}%
\pgfusepath{stroke,fill}%
}%
\begin{pgfscope}%
\pgfsys@transformshift{2.746613in}{3.350436in}%
\pgfsys@useobject{currentmarker}{}%
\end{pgfscope}%
\end{pgfscope}%
\begin{pgfscope}%
\pgfsetrectcap%
\pgfsetmiterjoin%
\pgfsetlinewidth{0.803000pt}%
\definecolor{currentstroke}{rgb}{0.000000,0.000000,0.000000}%
\pgfsetstrokecolor{currentstroke}%
\pgfsetdash{}{0pt}%
\pgfpathmoveto{\pgfqpoint{2.746613in}{2.197519in}}%
\pgfpathlineto{\pgfqpoint{2.746613in}{3.350436in}}%
\pgfusepath{stroke}%
\end{pgfscope}%
\begin{pgfscope}%
\pgfsetrectcap%
\pgfsetmiterjoin%
\pgfsetlinewidth{0.803000pt}%
\definecolor{currentstroke}{rgb}{0.000000,0.000000,0.000000}%
\pgfsetstrokecolor{currentstroke}%
\pgfsetdash{}{0pt}%
\pgfpathmoveto{\pgfqpoint{3.899530in}{2.197519in}}%
\pgfpathlineto{\pgfqpoint{3.899530in}{3.350436in}}%
\pgfusepath{stroke}%
\end{pgfscope}%
\begin{pgfscope}%
\pgfsetrectcap%
\pgfsetmiterjoin%
\pgfsetlinewidth{0.803000pt}%
\definecolor{currentstroke}{rgb}{0.000000,0.000000,0.000000}%
\pgfsetstrokecolor{currentstroke}%
\pgfsetdash{}{0pt}%
\pgfpathmoveto{\pgfqpoint{2.746613in}{2.197519in}}%
\pgfpathlineto{\pgfqpoint{3.899530in}{2.197519in}}%
\pgfusepath{stroke}%
\end{pgfscope}%
\begin{pgfscope}%
\pgfsetrectcap%
\pgfsetmiterjoin%
\pgfsetlinewidth{0.803000pt}%
\definecolor{currentstroke}{rgb}{0.000000,0.000000,0.000000}%
\pgfsetstrokecolor{currentstroke}%
\pgfsetdash{}{0pt}%
\pgfpathmoveto{\pgfqpoint{2.746613in}{3.350436in}}%
\pgfpathlineto{\pgfqpoint{3.899530in}{3.350436in}}%
\pgfusepath{stroke}%
\end{pgfscope}%
\begin{pgfscope}%
\definecolor{textcolor}{rgb}{0.000000,0.000000,0.000000}%
\pgfsetstrokecolor{textcolor}%
\pgfsetfillcolor{textcolor}%
\pgftext[x=3.323071in,y=3.433769in,,base]{\color{textcolor}\rmfamily\fontsize{12.000000}{14.400000}\selectfont n=2}%
\end{pgfscope}%
\begin{pgfscope}%
\pgfsetbuttcap%
\pgfsetmiterjoin%
\definecolor{currentfill}{rgb}{1.000000,1.000000,1.000000}%
\pgfsetfillcolor{currentfill}%
\pgfsetlinewidth{0.000000pt}%
\definecolor{currentstroke}{rgb}{0.000000,0.000000,0.000000}%
\pgfsetstrokecolor{currentstroke}%
\pgfsetstrokeopacity{0.000000}%
\pgfsetdash{}{0pt}%
\pgfpathmoveto{\pgfqpoint{4.499236in}{2.197519in}}%
\pgfpathlineto{\pgfqpoint{5.652153in}{2.197519in}}%
\pgfpathlineto{\pgfqpoint{5.652153in}{3.350436in}}%
\pgfpathlineto{\pgfqpoint{4.499236in}{3.350436in}}%
\pgfpathlineto{\pgfqpoint{4.499236in}{2.197519in}}%
\pgfpathclose%
\pgfusepath{fill}%
\end{pgfscope}%
\begin{pgfscope}%
\pgfpathrectangle{\pgfqpoint{4.499236in}{2.197519in}}{\pgfqpoint{1.152917in}{1.152917in}}%
\pgfusepath{clip}%
\pgfsys@transformcm{1.160000}{0.000000}{0.000000}{1.160000}{4.499236in}{2.197519in}%
\pgftext[left,bottom]{\includegraphics[interpolate=false,width=1.000000in,height=1.000000in]{numerical-solution-3-wave-function-plot-img2.png}}%
\end{pgfscope}%
\begin{pgfscope}%
\pgfsetbuttcap%
\pgfsetroundjoin%
\definecolor{currentfill}{rgb}{0.000000,0.000000,0.000000}%
\pgfsetfillcolor{currentfill}%
\pgfsetlinewidth{0.803000pt}%
\definecolor{currentstroke}{rgb}{0.000000,0.000000,0.000000}%
\pgfsetstrokecolor{currentstroke}%
\pgfsetdash{}{0pt}%
\pgfsys@defobject{currentmarker}{\pgfqpoint{0.000000in}{-0.048611in}}{\pgfqpoint{0.000000in}{0.000000in}}{%
\pgfpathmoveto{\pgfqpoint{0.000000in}{0.000000in}}%
\pgfpathlineto{\pgfqpoint{0.000000in}{-0.048611in}}%
\pgfusepath{stroke,fill}%
}%
\begin{pgfscope}%
\pgfsys@transformshift{4.499236in}{2.197519in}%
\pgfsys@useobject{currentmarker}{}%
\end{pgfscope}%
\end{pgfscope}%
\begin{pgfscope}%
\pgfsetbuttcap%
\pgfsetroundjoin%
\definecolor{currentfill}{rgb}{0.000000,0.000000,0.000000}%
\pgfsetfillcolor{currentfill}%
\pgfsetlinewidth{0.803000pt}%
\definecolor{currentstroke}{rgb}{0.000000,0.000000,0.000000}%
\pgfsetstrokecolor{currentstroke}%
\pgfsetdash{}{0pt}%
\pgfsys@defobject{currentmarker}{\pgfqpoint{0.000000in}{-0.048611in}}{\pgfqpoint{0.000000in}{0.000000in}}{%
\pgfpathmoveto{\pgfqpoint{0.000000in}{0.000000in}}%
\pgfpathlineto{\pgfqpoint{0.000000in}{-0.048611in}}%
\pgfusepath{stroke,fill}%
}%
\begin{pgfscope}%
\pgfsys@transformshift{5.075695in}{2.197519in}%
\pgfsys@useobject{currentmarker}{}%
\end{pgfscope}%
\end{pgfscope}%
\begin{pgfscope}%
\pgfsetbuttcap%
\pgfsetroundjoin%
\definecolor{currentfill}{rgb}{0.000000,0.000000,0.000000}%
\pgfsetfillcolor{currentfill}%
\pgfsetlinewidth{0.803000pt}%
\definecolor{currentstroke}{rgb}{0.000000,0.000000,0.000000}%
\pgfsetstrokecolor{currentstroke}%
\pgfsetdash{}{0pt}%
\pgfsys@defobject{currentmarker}{\pgfqpoint{0.000000in}{-0.048611in}}{\pgfqpoint{0.000000in}{0.000000in}}{%
\pgfpathmoveto{\pgfqpoint{0.000000in}{0.000000in}}%
\pgfpathlineto{\pgfqpoint{0.000000in}{-0.048611in}}%
\pgfusepath{stroke,fill}%
}%
\begin{pgfscope}%
\pgfsys@transformshift{5.652153in}{2.197519in}%
\pgfsys@useobject{currentmarker}{}%
\end{pgfscope}%
\end{pgfscope}%
\begin{pgfscope}%
\pgfsetbuttcap%
\pgfsetroundjoin%
\definecolor{currentfill}{rgb}{0.000000,0.000000,0.000000}%
\pgfsetfillcolor{currentfill}%
\pgfsetlinewidth{0.803000pt}%
\definecolor{currentstroke}{rgb}{0.000000,0.000000,0.000000}%
\pgfsetstrokecolor{currentstroke}%
\pgfsetdash{}{0pt}%
\pgfsys@defobject{currentmarker}{\pgfqpoint{-0.048611in}{0.000000in}}{\pgfqpoint{-0.000000in}{0.000000in}}{%
\pgfpathmoveto{\pgfqpoint{-0.000000in}{0.000000in}}%
\pgfpathlineto{\pgfqpoint{-0.048611in}{0.000000in}}%
\pgfusepath{stroke,fill}%
}%
\begin{pgfscope}%
\pgfsys@transformshift{4.499236in}{2.197519in}%
\pgfsys@useobject{currentmarker}{}%
\end{pgfscope}%
\end{pgfscope}%
\begin{pgfscope}%
\pgfsetbuttcap%
\pgfsetroundjoin%
\definecolor{currentfill}{rgb}{0.000000,0.000000,0.000000}%
\pgfsetfillcolor{currentfill}%
\pgfsetlinewidth{0.803000pt}%
\definecolor{currentstroke}{rgb}{0.000000,0.000000,0.000000}%
\pgfsetstrokecolor{currentstroke}%
\pgfsetdash{}{0pt}%
\pgfsys@defobject{currentmarker}{\pgfqpoint{-0.048611in}{0.000000in}}{\pgfqpoint{-0.000000in}{0.000000in}}{%
\pgfpathmoveto{\pgfqpoint{-0.000000in}{0.000000in}}%
\pgfpathlineto{\pgfqpoint{-0.048611in}{0.000000in}}%
\pgfusepath{stroke,fill}%
}%
\begin{pgfscope}%
\pgfsys@transformshift{4.499236in}{2.485748in}%
\pgfsys@useobject{currentmarker}{}%
\end{pgfscope}%
\end{pgfscope}%
\begin{pgfscope}%
\pgfsetbuttcap%
\pgfsetroundjoin%
\definecolor{currentfill}{rgb}{0.000000,0.000000,0.000000}%
\pgfsetfillcolor{currentfill}%
\pgfsetlinewidth{0.803000pt}%
\definecolor{currentstroke}{rgb}{0.000000,0.000000,0.000000}%
\pgfsetstrokecolor{currentstroke}%
\pgfsetdash{}{0pt}%
\pgfsys@defobject{currentmarker}{\pgfqpoint{-0.048611in}{0.000000in}}{\pgfqpoint{-0.000000in}{0.000000in}}{%
\pgfpathmoveto{\pgfqpoint{-0.000000in}{0.000000in}}%
\pgfpathlineto{\pgfqpoint{-0.048611in}{0.000000in}}%
\pgfusepath{stroke,fill}%
}%
\begin{pgfscope}%
\pgfsys@transformshift{4.499236in}{2.773978in}%
\pgfsys@useobject{currentmarker}{}%
\end{pgfscope}%
\end{pgfscope}%
\begin{pgfscope}%
\pgfsetbuttcap%
\pgfsetroundjoin%
\definecolor{currentfill}{rgb}{0.000000,0.000000,0.000000}%
\pgfsetfillcolor{currentfill}%
\pgfsetlinewidth{0.803000pt}%
\definecolor{currentstroke}{rgb}{0.000000,0.000000,0.000000}%
\pgfsetstrokecolor{currentstroke}%
\pgfsetdash{}{0pt}%
\pgfsys@defobject{currentmarker}{\pgfqpoint{-0.048611in}{0.000000in}}{\pgfqpoint{-0.000000in}{0.000000in}}{%
\pgfpathmoveto{\pgfqpoint{-0.000000in}{0.000000in}}%
\pgfpathlineto{\pgfqpoint{-0.048611in}{0.000000in}}%
\pgfusepath{stroke,fill}%
}%
\begin{pgfscope}%
\pgfsys@transformshift{4.499236in}{3.062207in}%
\pgfsys@useobject{currentmarker}{}%
\end{pgfscope}%
\end{pgfscope}%
\begin{pgfscope}%
\pgfsetbuttcap%
\pgfsetroundjoin%
\definecolor{currentfill}{rgb}{0.000000,0.000000,0.000000}%
\pgfsetfillcolor{currentfill}%
\pgfsetlinewidth{0.803000pt}%
\definecolor{currentstroke}{rgb}{0.000000,0.000000,0.000000}%
\pgfsetstrokecolor{currentstroke}%
\pgfsetdash{}{0pt}%
\pgfsys@defobject{currentmarker}{\pgfqpoint{-0.048611in}{0.000000in}}{\pgfqpoint{-0.000000in}{0.000000in}}{%
\pgfpathmoveto{\pgfqpoint{-0.000000in}{0.000000in}}%
\pgfpathlineto{\pgfqpoint{-0.048611in}{0.000000in}}%
\pgfusepath{stroke,fill}%
}%
\begin{pgfscope}%
\pgfsys@transformshift{4.499236in}{3.350436in}%
\pgfsys@useobject{currentmarker}{}%
\end{pgfscope}%
\end{pgfscope}%
\begin{pgfscope}%
\pgfsetrectcap%
\pgfsetmiterjoin%
\pgfsetlinewidth{0.803000pt}%
\definecolor{currentstroke}{rgb}{0.000000,0.000000,0.000000}%
\pgfsetstrokecolor{currentstroke}%
\pgfsetdash{}{0pt}%
\pgfpathmoveto{\pgfqpoint{4.499236in}{2.197519in}}%
\pgfpathlineto{\pgfqpoint{4.499236in}{3.350436in}}%
\pgfusepath{stroke}%
\end{pgfscope}%
\begin{pgfscope}%
\pgfsetrectcap%
\pgfsetmiterjoin%
\pgfsetlinewidth{0.803000pt}%
\definecolor{currentstroke}{rgb}{0.000000,0.000000,0.000000}%
\pgfsetstrokecolor{currentstroke}%
\pgfsetdash{}{0pt}%
\pgfpathmoveto{\pgfqpoint{5.652153in}{2.197519in}}%
\pgfpathlineto{\pgfqpoint{5.652153in}{3.350436in}}%
\pgfusepath{stroke}%
\end{pgfscope}%
\begin{pgfscope}%
\pgfsetrectcap%
\pgfsetmiterjoin%
\pgfsetlinewidth{0.803000pt}%
\definecolor{currentstroke}{rgb}{0.000000,0.000000,0.000000}%
\pgfsetstrokecolor{currentstroke}%
\pgfsetdash{}{0pt}%
\pgfpathmoveto{\pgfqpoint{4.499236in}{2.197519in}}%
\pgfpathlineto{\pgfqpoint{5.652153in}{2.197519in}}%
\pgfusepath{stroke}%
\end{pgfscope}%
\begin{pgfscope}%
\pgfsetrectcap%
\pgfsetmiterjoin%
\pgfsetlinewidth{0.803000pt}%
\definecolor{currentstroke}{rgb}{0.000000,0.000000,0.000000}%
\pgfsetstrokecolor{currentstroke}%
\pgfsetdash{}{0pt}%
\pgfpathmoveto{\pgfqpoint{4.499236in}{3.350436in}}%
\pgfpathlineto{\pgfqpoint{5.652153in}{3.350436in}}%
\pgfusepath{stroke}%
\end{pgfscope}%
\begin{pgfscope}%
\definecolor{textcolor}{rgb}{0.000000,0.000000,0.000000}%
\pgfsetstrokecolor{textcolor}%
\pgfsetfillcolor{textcolor}%
\pgftext[x=5.075695in,y=3.433769in,,base]{\color{textcolor}\rmfamily\fontsize{12.000000}{14.400000}\selectfont n=3}%
\end{pgfscope}%
\begin{pgfscope}%
\pgfsetbuttcap%
\pgfsetmiterjoin%
\definecolor{currentfill}{rgb}{1.000000,1.000000,1.000000}%
\pgfsetfillcolor{currentfill}%
\pgfsetlinewidth{0.000000pt}%
\definecolor{currentstroke}{rgb}{0.000000,0.000000,0.000000}%
\pgfsetstrokecolor{currentstroke}%
\pgfsetstrokeopacity{0.000000}%
\pgfsetdash{}{0pt}%
\pgfpathmoveto{\pgfqpoint{0.993989in}{0.652650in}}%
\pgfpathlineto{\pgfqpoint{2.146906in}{0.652650in}}%
\pgfpathlineto{\pgfqpoint{2.146906in}{1.805567in}}%
\pgfpathlineto{\pgfqpoint{0.993989in}{1.805567in}}%
\pgfpathlineto{\pgfqpoint{0.993989in}{0.652650in}}%
\pgfpathclose%
\pgfusepath{fill}%
\end{pgfscope}%
\begin{pgfscope}%
\pgfpathrectangle{\pgfqpoint{0.993989in}{0.652650in}}{\pgfqpoint{1.152917in}{1.152917in}}%
\pgfusepath{clip}%
\pgfsys@transformcm{1.160000}{0.000000}{0.000000}{1.160000}{0.993989in}{0.652650in}%
\pgftext[left,bottom]{\includegraphics[interpolate=false,width=1.000000in,height=1.000000in]{numerical-solution-3-wave-function-plot-img3.png}}%
\end{pgfscope}%
\begin{pgfscope}%
\pgfsetbuttcap%
\pgfsetroundjoin%
\definecolor{currentfill}{rgb}{0.000000,0.000000,0.000000}%
\pgfsetfillcolor{currentfill}%
\pgfsetlinewidth{0.803000pt}%
\definecolor{currentstroke}{rgb}{0.000000,0.000000,0.000000}%
\pgfsetstrokecolor{currentstroke}%
\pgfsetdash{}{0pt}%
\pgfsys@defobject{currentmarker}{\pgfqpoint{0.000000in}{-0.048611in}}{\pgfqpoint{0.000000in}{0.000000in}}{%
\pgfpathmoveto{\pgfqpoint{0.000000in}{0.000000in}}%
\pgfpathlineto{\pgfqpoint{0.000000in}{-0.048611in}}%
\pgfusepath{stroke,fill}%
}%
\begin{pgfscope}%
\pgfsys@transformshift{0.993989in}{0.652650in}%
\pgfsys@useobject{currentmarker}{}%
\end{pgfscope}%
\end{pgfscope}%
\begin{pgfscope}%
\definecolor{textcolor}{rgb}{0.000000,0.000000,0.000000}%
\pgfsetstrokecolor{textcolor}%
\pgfsetfillcolor{textcolor}%
\pgftext[x=0.993989in,y=0.555428in,,top]{\color{textcolor}\rmfamily\fontsize{10.000000}{12.000000}\selectfont \(\displaystyle {\ensuremath{-}5}\)}%
\end{pgfscope}%
\begin{pgfscope}%
\pgfsetbuttcap%
\pgfsetroundjoin%
\definecolor{currentfill}{rgb}{0.000000,0.000000,0.000000}%
\pgfsetfillcolor{currentfill}%
\pgfsetlinewidth{0.803000pt}%
\definecolor{currentstroke}{rgb}{0.000000,0.000000,0.000000}%
\pgfsetstrokecolor{currentstroke}%
\pgfsetdash{}{0pt}%
\pgfsys@defobject{currentmarker}{\pgfqpoint{0.000000in}{-0.048611in}}{\pgfqpoint{0.000000in}{0.000000in}}{%
\pgfpathmoveto{\pgfqpoint{0.000000in}{0.000000in}}%
\pgfpathlineto{\pgfqpoint{0.000000in}{-0.048611in}}%
\pgfusepath{stroke,fill}%
}%
\begin{pgfscope}%
\pgfsys@transformshift{1.570448in}{0.652650in}%
\pgfsys@useobject{currentmarker}{}%
\end{pgfscope}%
\end{pgfscope}%
\begin{pgfscope}%
\definecolor{textcolor}{rgb}{0.000000,0.000000,0.000000}%
\pgfsetstrokecolor{textcolor}%
\pgfsetfillcolor{textcolor}%
\pgftext[x=1.570448in,y=0.555428in,,top]{\color{textcolor}\rmfamily\fontsize{10.000000}{12.000000}\selectfont \(\displaystyle {0}\)}%
\end{pgfscope}%
\begin{pgfscope}%
\pgfsetbuttcap%
\pgfsetroundjoin%
\definecolor{currentfill}{rgb}{0.000000,0.000000,0.000000}%
\pgfsetfillcolor{currentfill}%
\pgfsetlinewidth{0.803000pt}%
\definecolor{currentstroke}{rgb}{0.000000,0.000000,0.000000}%
\pgfsetstrokecolor{currentstroke}%
\pgfsetdash{}{0pt}%
\pgfsys@defobject{currentmarker}{\pgfqpoint{0.000000in}{-0.048611in}}{\pgfqpoint{0.000000in}{0.000000in}}{%
\pgfpathmoveto{\pgfqpoint{0.000000in}{0.000000in}}%
\pgfpathlineto{\pgfqpoint{0.000000in}{-0.048611in}}%
\pgfusepath{stroke,fill}%
}%
\begin{pgfscope}%
\pgfsys@transformshift{2.146906in}{0.652650in}%
\pgfsys@useobject{currentmarker}{}%
\end{pgfscope}%
\end{pgfscope}%
\begin{pgfscope}%
\definecolor{textcolor}{rgb}{0.000000,0.000000,0.000000}%
\pgfsetstrokecolor{textcolor}%
\pgfsetfillcolor{textcolor}%
\pgftext[x=2.146906in,y=0.555428in,,top]{\color{textcolor}\rmfamily\fontsize{10.000000}{12.000000}\selectfont \(\displaystyle {5}\)}%
\end{pgfscope}%
\begin{pgfscope}%
\pgfsetbuttcap%
\pgfsetroundjoin%
\definecolor{currentfill}{rgb}{0.000000,0.000000,0.000000}%
\pgfsetfillcolor{currentfill}%
\pgfsetlinewidth{0.803000pt}%
\definecolor{currentstroke}{rgb}{0.000000,0.000000,0.000000}%
\pgfsetstrokecolor{currentstroke}%
\pgfsetdash{}{0pt}%
\pgfsys@defobject{currentmarker}{\pgfqpoint{-0.048611in}{0.000000in}}{\pgfqpoint{-0.000000in}{0.000000in}}{%
\pgfpathmoveto{\pgfqpoint{-0.000000in}{0.000000in}}%
\pgfpathlineto{\pgfqpoint{-0.048611in}{0.000000in}}%
\pgfusepath{stroke,fill}%
}%
\begin{pgfscope}%
\pgfsys@transformshift{0.993989in}{0.652650in}%
\pgfsys@useobject{currentmarker}{}%
\end{pgfscope}%
\end{pgfscope}%
\begin{pgfscope}%
\definecolor{textcolor}{rgb}{0.000000,0.000000,0.000000}%
\pgfsetstrokecolor{textcolor}%
\pgfsetfillcolor{textcolor}%
\pgftext[x=0.611272in, y=0.604425in, left, base]{\color{textcolor}\rmfamily\fontsize{10.000000}{12.000000}\selectfont \(\displaystyle {\ensuremath{-}5.0}\)}%
\end{pgfscope}%
\begin{pgfscope}%
\pgfsetbuttcap%
\pgfsetroundjoin%
\definecolor{currentfill}{rgb}{0.000000,0.000000,0.000000}%
\pgfsetfillcolor{currentfill}%
\pgfsetlinewidth{0.803000pt}%
\definecolor{currentstroke}{rgb}{0.000000,0.000000,0.000000}%
\pgfsetstrokecolor{currentstroke}%
\pgfsetdash{}{0pt}%
\pgfsys@defobject{currentmarker}{\pgfqpoint{-0.048611in}{0.000000in}}{\pgfqpoint{-0.000000in}{0.000000in}}{%
\pgfpathmoveto{\pgfqpoint{-0.000000in}{0.000000in}}%
\pgfpathlineto{\pgfqpoint{-0.048611in}{0.000000in}}%
\pgfusepath{stroke,fill}%
}%
\begin{pgfscope}%
\pgfsys@transformshift{0.993989in}{0.940879in}%
\pgfsys@useobject{currentmarker}{}%
\end{pgfscope}%
\end{pgfscope}%
\begin{pgfscope}%
\definecolor{textcolor}{rgb}{0.000000,0.000000,0.000000}%
\pgfsetstrokecolor{textcolor}%
\pgfsetfillcolor{textcolor}%
\pgftext[x=0.611272in, y=0.892654in, left, base]{\color{textcolor}\rmfamily\fontsize{10.000000}{12.000000}\selectfont \(\displaystyle {\ensuremath{-}2.5}\)}%
\end{pgfscope}%
\begin{pgfscope}%
\pgfsetbuttcap%
\pgfsetroundjoin%
\definecolor{currentfill}{rgb}{0.000000,0.000000,0.000000}%
\pgfsetfillcolor{currentfill}%
\pgfsetlinewidth{0.803000pt}%
\definecolor{currentstroke}{rgb}{0.000000,0.000000,0.000000}%
\pgfsetstrokecolor{currentstroke}%
\pgfsetdash{}{0pt}%
\pgfsys@defobject{currentmarker}{\pgfqpoint{-0.048611in}{0.000000in}}{\pgfqpoint{-0.000000in}{0.000000in}}{%
\pgfpathmoveto{\pgfqpoint{-0.000000in}{0.000000in}}%
\pgfpathlineto{\pgfqpoint{-0.048611in}{0.000000in}}%
\pgfusepath{stroke,fill}%
}%
\begin{pgfscope}%
\pgfsys@transformshift{0.993989in}{1.229108in}%
\pgfsys@useobject{currentmarker}{}%
\end{pgfscope}%
\end{pgfscope}%
\begin{pgfscope}%
\definecolor{textcolor}{rgb}{0.000000,0.000000,0.000000}%
\pgfsetstrokecolor{textcolor}%
\pgfsetfillcolor{textcolor}%
\pgftext[x=0.719297in, y=1.180883in, left, base]{\color{textcolor}\rmfamily\fontsize{10.000000}{12.000000}\selectfont \(\displaystyle {0.0}\)}%
\end{pgfscope}%
\begin{pgfscope}%
\pgfsetbuttcap%
\pgfsetroundjoin%
\definecolor{currentfill}{rgb}{0.000000,0.000000,0.000000}%
\pgfsetfillcolor{currentfill}%
\pgfsetlinewidth{0.803000pt}%
\definecolor{currentstroke}{rgb}{0.000000,0.000000,0.000000}%
\pgfsetstrokecolor{currentstroke}%
\pgfsetdash{}{0pt}%
\pgfsys@defobject{currentmarker}{\pgfqpoint{-0.048611in}{0.000000in}}{\pgfqpoint{-0.000000in}{0.000000in}}{%
\pgfpathmoveto{\pgfqpoint{-0.000000in}{0.000000in}}%
\pgfpathlineto{\pgfqpoint{-0.048611in}{0.000000in}}%
\pgfusepath{stroke,fill}%
}%
\begin{pgfscope}%
\pgfsys@transformshift{0.993989in}{1.517338in}%
\pgfsys@useobject{currentmarker}{}%
\end{pgfscope}%
\end{pgfscope}%
\begin{pgfscope}%
\definecolor{textcolor}{rgb}{0.000000,0.000000,0.000000}%
\pgfsetstrokecolor{textcolor}%
\pgfsetfillcolor{textcolor}%
\pgftext[x=0.719297in, y=1.469112in, left, base]{\color{textcolor}\rmfamily\fontsize{10.000000}{12.000000}\selectfont \(\displaystyle {2.5}\)}%
\end{pgfscope}%
\begin{pgfscope}%
\pgfsetbuttcap%
\pgfsetroundjoin%
\definecolor{currentfill}{rgb}{0.000000,0.000000,0.000000}%
\pgfsetfillcolor{currentfill}%
\pgfsetlinewidth{0.803000pt}%
\definecolor{currentstroke}{rgb}{0.000000,0.000000,0.000000}%
\pgfsetstrokecolor{currentstroke}%
\pgfsetdash{}{0pt}%
\pgfsys@defobject{currentmarker}{\pgfqpoint{-0.048611in}{0.000000in}}{\pgfqpoint{-0.000000in}{0.000000in}}{%
\pgfpathmoveto{\pgfqpoint{-0.000000in}{0.000000in}}%
\pgfpathlineto{\pgfqpoint{-0.048611in}{0.000000in}}%
\pgfusepath{stroke,fill}%
}%
\begin{pgfscope}%
\pgfsys@transformshift{0.993989in}{1.805567in}%
\pgfsys@useobject{currentmarker}{}%
\end{pgfscope}%
\end{pgfscope}%
\begin{pgfscope}%
\definecolor{textcolor}{rgb}{0.000000,0.000000,0.000000}%
\pgfsetstrokecolor{textcolor}%
\pgfsetfillcolor{textcolor}%
\pgftext[x=0.719297in, y=1.757342in, left, base]{\color{textcolor}\rmfamily\fontsize{10.000000}{12.000000}\selectfont \(\displaystyle {5.0}\)}%
\end{pgfscope}%
\begin{pgfscope}%
\pgfsetrectcap%
\pgfsetmiterjoin%
\pgfsetlinewidth{0.803000pt}%
\definecolor{currentstroke}{rgb}{0.000000,0.000000,0.000000}%
\pgfsetstrokecolor{currentstroke}%
\pgfsetdash{}{0pt}%
\pgfpathmoveto{\pgfqpoint{0.993989in}{0.652650in}}%
\pgfpathlineto{\pgfqpoint{0.993989in}{1.805567in}}%
\pgfusepath{stroke}%
\end{pgfscope}%
\begin{pgfscope}%
\pgfsetrectcap%
\pgfsetmiterjoin%
\pgfsetlinewidth{0.803000pt}%
\definecolor{currentstroke}{rgb}{0.000000,0.000000,0.000000}%
\pgfsetstrokecolor{currentstroke}%
\pgfsetdash{}{0pt}%
\pgfpathmoveto{\pgfqpoint{2.146906in}{0.652650in}}%
\pgfpathlineto{\pgfqpoint{2.146906in}{1.805567in}}%
\pgfusepath{stroke}%
\end{pgfscope}%
\begin{pgfscope}%
\pgfsetrectcap%
\pgfsetmiterjoin%
\pgfsetlinewidth{0.803000pt}%
\definecolor{currentstroke}{rgb}{0.000000,0.000000,0.000000}%
\pgfsetstrokecolor{currentstroke}%
\pgfsetdash{}{0pt}%
\pgfpathmoveto{\pgfqpoint{0.993989in}{0.652650in}}%
\pgfpathlineto{\pgfqpoint{2.146906in}{0.652650in}}%
\pgfusepath{stroke}%
\end{pgfscope}%
\begin{pgfscope}%
\pgfsetrectcap%
\pgfsetmiterjoin%
\pgfsetlinewidth{0.803000pt}%
\definecolor{currentstroke}{rgb}{0.000000,0.000000,0.000000}%
\pgfsetstrokecolor{currentstroke}%
\pgfsetdash{}{0pt}%
\pgfpathmoveto{\pgfqpoint{0.993989in}{1.805567in}}%
\pgfpathlineto{\pgfqpoint{2.146906in}{1.805567in}}%
\pgfusepath{stroke}%
\end{pgfscope}%
\begin{pgfscope}%
\definecolor{textcolor}{rgb}{0.000000,0.000000,0.000000}%
\pgfsetstrokecolor{textcolor}%
\pgfsetfillcolor{textcolor}%
\pgftext[x=1.570448in,y=1.888900in,,base]{\color{textcolor}\rmfamily\fontsize{12.000000}{14.400000}\selectfont n=4}%
\end{pgfscope}%
\begin{pgfscope}%
\pgfsetbuttcap%
\pgfsetmiterjoin%
\definecolor{currentfill}{rgb}{1.000000,1.000000,1.000000}%
\pgfsetfillcolor{currentfill}%
\pgfsetlinewidth{0.000000pt}%
\definecolor{currentstroke}{rgb}{0.000000,0.000000,0.000000}%
\pgfsetstrokecolor{currentstroke}%
\pgfsetstrokeopacity{0.000000}%
\pgfsetdash{}{0pt}%
\pgfpathmoveto{\pgfqpoint{2.746613in}{0.652650in}}%
\pgfpathlineto{\pgfqpoint{3.899530in}{0.652650in}}%
\pgfpathlineto{\pgfqpoint{3.899530in}{1.805567in}}%
\pgfpathlineto{\pgfqpoint{2.746613in}{1.805567in}}%
\pgfpathlineto{\pgfqpoint{2.746613in}{0.652650in}}%
\pgfpathclose%
\pgfusepath{fill}%
\end{pgfscope}%
\begin{pgfscope}%
\pgfpathrectangle{\pgfqpoint{2.746613in}{0.652650in}}{\pgfqpoint{1.152917in}{1.152917in}}%
\pgfusepath{clip}%
\pgfsys@transformcm{1.160000}{0.000000}{0.000000}{1.160000}{2.746613in}{0.652650in}%
\pgftext[left,bottom]{\includegraphics[interpolate=false,width=1.000000in,height=1.000000in]{numerical-solution-3-wave-function-plot-img4.png}}%
\end{pgfscope}%
\begin{pgfscope}%
\pgfsetbuttcap%
\pgfsetroundjoin%
\definecolor{currentfill}{rgb}{0.000000,0.000000,0.000000}%
\pgfsetfillcolor{currentfill}%
\pgfsetlinewidth{0.803000pt}%
\definecolor{currentstroke}{rgb}{0.000000,0.000000,0.000000}%
\pgfsetstrokecolor{currentstroke}%
\pgfsetdash{}{0pt}%
\pgfsys@defobject{currentmarker}{\pgfqpoint{0.000000in}{-0.048611in}}{\pgfqpoint{0.000000in}{0.000000in}}{%
\pgfpathmoveto{\pgfqpoint{0.000000in}{0.000000in}}%
\pgfpathlineto{\pgfqpoint{0.000000in}{-0.048611in}}%
\pgfusepath{stroke,fill}%
}%
\begin{pgfscope}%
\pgfsys@transformshift{2.746613in}{0.652650in}%
\pgfsys@useobject{currentmarker}{}%
\end{pgfscope}%
\end{pgfscope}%
\begin{pgfscope}%
\definecolor{textcolor}{rgb}{0.000000,0.000000,0.000000}%
\pgfsetstrokecolor{textcolor}%
\pgfsetfillcolor{textcolor}%
\pgftext[x=2.746613in,y=0.555428in,,top]{\color{textcolor}\rmfamily\fontsize{10.000000}{12.000000}\selectfont \(\displaystyle {\ensuremath{-}5}\)}%
\end{pgfscope}%
\begin{pgfscope}%
\pgfsetbuttcap%
\pgfsetroundjoin%
\definecolor{currentfill}{rgb}{0.000000,0.000000,0.000000}%
\pgfsetfillcolor{currentfill}%
\pgfsetlinewidth{0.803000pt}%
\definecolor{currentstroke}{rgb}{0.000000,0.000000,0.000000}%
\pgfsetstrokecolor{currentstroke}%
\pgfsetdash{}{0pt}%
\pgfsys@defobject{currentmarker}{\pgfqpoint{0.000000in}{-0.048611in}}{\pgfqpoint{0.000000in}{0.000000in}}{%
\pgfpathmoveto{\pgfqpoint{0.000000in}{0.000000in}}%
\pgfpathlineto{\pgfqpoint{0.000000in}{-0.048611in}}%
\pgfusepath{stroke,fill}%
}%
\begin{pgfscope}%
\pgfsys@transformshift{3.323071in}{0.652650in}%
\pgfsys@useobject{currentmarker}{}%
\end{pgfscope}%
\end{pgfscope}%
\begin{pgfscope}%
\definecolor{textcolor}{rgb}{0.000000,0.000000,0.000000}%
\pgfsetstrokecolor{textcolor}%
\pgfsetfillcolor{textcolor}%
\pgftext[x=3.323071in,y=0.555428in,,top]{\color{textcolor}\rmfamily\fontsize{10.000000}{12.000000}\selectfont \(\displaystyle {0}\)}%
\end{pgfscope}%
\begin{pgfscope}%
\pgfsetbuttcap%
\pgfsetroundjoin%
\definecolor{currentfill}{rgb}{0.000000,0.000000,0.000000}%
\pgfsetfillcolor{currentfill}%
\pgfsetlinewidth{0.803000pt}%
\definecolor{currentstroke}{rgb}{0.000000,0.000000,0.000000}%
\pgfsetstrokecolor{currentstroke}%
\pgfsetdash{}{0pt}%
\pgfsys@defobject{currentmarker}{\pgfqpoint{0.000000in}{-0.048611in}}{\pgfqpoint{0.000000in}{0.000000in}}{%
\pgfpathmoveto{\pgfqpoint{0.000000in}{0.000000in}}%
\pgfpathlineto{\pgfqpoint{0.000000in}{-0.048611in}}%
\pgfusepath{stroke,fill}%
}%
\begin{pgfscope}%
\pgfsys@transformshift{3.899530in}{0.652650in}%
\pgfsys@useobject{currentmarker}{}%
\end{pgfscope}%
\end{pgfscope}%
\begin{pgfscope}%
\definecolor{textcolor}{rgb}{0.000000,0.000000,0.000000}%
\pgfsetstrokecolor{textcolor}%
\pgfsetfillcolor{textcolor}%
\pgftext[x=3.899530in,y=0.555428in,,top]{\color{textcolor}\rmfamily\fontsize{10.000000}{12.000000}\selectfont \(\displaystyle {5}\)}%
\end{pgfscope}%
\begin{pgfscope}%
\pgfsetbuttcap%
\pgfsetroundjoin%
\definecolor{currentfill}{rgb}{0.000000,0.000000,0.000000}%
\pgfsetfillcolor{currentfill}%
\pgfsetlinewidth{0.803000pt}%
\definecolor{currentstroke}{rgb}{0.000000,0.000000,0.000000}%
\pgfsetstrokecolor{currentstroke}%
\pgfsetdash{}{0pt}%
\pgfsys@defobject{currentmarker}{\pgfqpoint{-0.048611in}{0.000000in}}{\pgfqpoint{-0.000000in}{0.000000in}}{%
\pgfpathmoveto{\pgfqpoint{-0.000000in}{0.000000in}}%
\pgfpathlineto{\pgfqpoint{-0.048611in}{0.000000in}}%
\pgfusepath{stroke,fill}%
}%
\begin{pgfscope}%
\pgfsys@transformshift{2.746613in}{0.652650in}%
\pgfsys@useobject{currentmarker}{}%
\end{pgfscope}%
\end{pgfscope}%
\begin{pgfscope}%
\pgfsetbuttcap%
\pgfsetroundjoin%
\definecolor{currentfill}{rgb}{0.000000,0.000000,0.000000}%
\pgfsetfillcolor{currentfill}%
\pgfsetlinewidth{0.803000pt}%
\definecolor{currentstroke}{rgb}{0.000000,0.000000,0.000000}%
\pgfsetstrokecolor{currentstroke}%
\pgfsetdash{}{0pt}%
\pgfsys@defobject{currentmarker}{\pgfqpoint{-0.048611in}{0.000000in}}{\pgfqpoint{-0.000000in}{0.000000in}}{%
\pgfpathmoveto{\pgfqpoint{-0.000000in}{0.000000in}}%
\pgfpathlineto{\pgfqpoint{-0.048611in}{0.000000in}}%
\pgfusepath{stroke,fill}%
}%
\begin{pgfscope}%
\pgfsys@transformshift{2.746613in}{0.940879in}%
\pgfsys@useobject{currentmarker}{}%
\end{pgfscope}%
\end{pgfscope}%
\begin{pgfscope}%
\pgfsetbuttcap%
\pgfsetroundjoin%
\definecolor{currentfill}{rgb}{0.000000,0.000000,0.000000}%
\pgfsetfillcolor{currentfill}%
\pgfsetlinewidth{0.803000pt}%
\definecolor{currentstroke}{rgb}{0.000000,0.000000,0.000000}%
\pgfsetstrokecolor{currentstroke}%
\pgfsetdash{}{0pt}%
\pgfsys@defobject{currentmarker}{\pgfqpoint{-0.048611in}{0.000000in}}{\pgfqpoint{-0.000000in}{0.000000in}}{%
\pgfpathmoveto{\pgfqpoint{-0.000000in}{0.000000in}}%
\pgfpathlineto{\pgfqpoint{-0.048611in}{0.000000in}}%
\pgfusepath{stroke,fill}%
}%
\begin{pgfscope}%
\pgfsys@transformshift{2.746613in}{1.229108in}%
\pgfsys@useobject{currentmarker}{}%
\end{pgfscope}%
\end{pgfscope}%
\begin{pgfscope}%
\pgfsetbuttcap%
\pgfsetroundjoin%
\definecolor{currentfill}{rgb}{0.000000,0.000000,0.000000}%
\pgfsetfillcolor{currentfill}%
\pgfsetlinewidth{0.803000pt}%
\definecolor{currentstroke}{rgb}{0.000000,0.000000,0.000000}%
\pgfsetstrokecolor{currentstroke}%
\pgfsetdash{}{0pt}%
\pgfsys@defobject{currentmarker}{\pgfqpoint{-0.048611in}{0.000000in}}{\pgfqpoint{-0.000000in}{0.000000in}}{%
\pgfpathmoveto{\pgfqpoint{-0.000000in}{0.000000in}}%
\pgfpathlineto{\pgfqpoint{-0.048611in}{0.000000in}}%
\pgfusepath{stroke,fill}%
}%
\begin{pgfscope}%
\pgfsys@transformshift{2.746613in}{1.517338in}%
\pgfsys@useobject{currentmarker}{}%
\end{pgfscope}%
\end{pgfscope}%
\begin{pgfscope}%
\pgfsetbuttcap%
\pgfsetroundjoin%
\definecolor{currentfill}{rgb}{0.000000,0.000000,0.000000}%
\pgfsetfillcolor{currentfill}%
\pgfsetlinewidth{0.803000pt}%
\definecolor{currentstroke}{rgb}{0.000000,0.000000,0.000000}%
\pgfsetstrokecolor{currentstroke}%
\pgfsetdash{}{0pt}%
\pgfsys@defobject{currentmarker}{\pgfqpoint{-0.048611in}{0.000000in}}{\pgfqpoint{-0.000000in}{0.000000in}}{%
\pgfpathmoveto{\pgfqpoint{-0.000000in}{0.000000in}}%
\pgfpathlineto{\pgfqpoint{-0.048611in}{0.000000in}}%
\pgfusepath{stroke,fill}%
}%
\begin{pgfscope}%
\pgfsys@transformshift{2.746613in}{1.805567in}%
\pgfsys@useobject{currentmarker}{}%
\end{pgfscope}%
\end{pgfscope}%
\begin{pgfscope}%
\pgfsetrectcap%
\pgfsetmiterjoin%
\pgfsetlinewidth{0.803000pt}%
\definecolor{currentstroke}{rgb}{0.000000,0.000000,0.000000}%
\pgfsetstrokecolor{currentstroke}%
\pgfsetdash{}{0pt}%
\pgfpathmoveto{\pgfqpoint{2.746613in}{0.652650in}}%
\pgfpathlineto{\pgfqpoint{2.746613in}{1.805567in}}%
\pgfusepath{stroke}%
\end{pgfscope}%
\begin{pgfscope}%
\pgfsetrectcap%
\pgfsetmiterjoin%
\pgfsetlinewidth{0.803000pt}%
\definecolor{currentstroke}{rgb}{0.000000,0.000000,0.000000}%
\pgfsetstrokecolor{currentstroke}%
\pgfsetdash{}{0pt}%
\pgfpathmoveto{\pgfqpoint{3.899530in}{0.652650in}}%
\pgfpathlineto{\pgfqpoint{3.899530in}{1.805567in}}%
\pgfusepath{stroke}%
\end{pgfscope}%
\begin{pgfscope}%
\pgfsetrectcap%
\pgfsetmiterjoin%
\pgfsetlinewidth{0.803000pt}%
\definecolor{currentstroke}{rgb}{0.000000,0.000000,0.000000}%
\pgfsetstrokecolor{currentstroke}%
\pgfsetdash{}{0pt}%
\pgfpathmoveto{\pgfqpoint{2.746613in}{0.652650in}}%
\pgfpathlineto{\pgfqpoint{3.899530in}{0.652650in}}%
\pgfusepath{stroke}%
\end{pgfscope}%
\begin{pgfscope}%
\pgfsetrectcap%
\pgfsetmiterjoin%
\pgfsetlinewidth{0.803000pt}%
\definecolor{currentstroke}{rgb}{0.000000,0.000000,0.000000}%
\pgfsetstrokecolor{currentstroke}%
\pgfsetdash{}{0pt}%
\pgfpathmoveto{\pgfqpoint{2.746613in}{1.805567in}}%
\pgfpathlineto{\pgfqpoint{3.899530in}{1.805567in}}%
\pgfusepath{stroke}%
\end{pgfscope}%
\begin{pgfscope}%
\definecolor{textcolor}{rgb}{0.000000,0.000000,0.000000}%
\pgfsetstrokecolor{textcolor}%
\pgfsetfillcolor{textcolor}%
\pgftext[x=3.323071in,y=1.888900in,,base]{\color{textcolor}\rmfamily\fontsize{12.000000}{14.400000}\selectfont n=5}%
\end{pgfscope}%
\begin{pgfscope}%
\pgfsetbuttcap%
\pgfsetmiterjoin%
\definecolor{currentfill}{rgb}{1.000000,1.000000,1.000000}%
\pgfsetfillcolor{currentfill}%
\pgfsetlinewidth{0.000000pt}%
\definecolor{currentstroke}{rgb}{0.000000,0.000000,0.000000}%
\pgfsetstrokecolor{currentstroke}%
\pgfsetstrokeopacity{0.000000}%
\pgfsetdash{}{0pt}%
\pgfpathmoveto{\pgfqpoint{4.499236in}{0.652650in}}%
\pgfpathlineto{\pgfqpoint{5.652153in}{0.652650in}}%
\pgfpathlineto{\pgfqpoint{5.652153in}{1.805567in}}%
\pgfpathlineto{\pgfqpoint{4.499236in}{1.805567in}}%
\pgfpathlineto{\pgfqpoint{4.499236in}{0.652650in}}%
\pgfpathclose%
\pgfusepath{fill}%
\end{pgfscope}%
\begin{pgfscope}%
\pgfpathrectangle{\pgfqpoint{4.499236in}{0.652650in}}{\pgfqpoint{1.152917in}{1.152917in}}%
\pgfusepath{clip}%
\pgfsys@transformcm{1.160000}{0.000000}{0.000000}{1.160000}{4.499236in}{0.652650in}%
\pgftext[left,bottom]{\includegraphics[interpolate=false,width=1.000000in,height=1.000000in]{numerical-solution-3-wave-function-plot-img5.png}}%
\end{pgfscope}%
\begin{pgfscope}%
\pgfsetbuttcap%
\pgfsetroundjoin%
\definecolor{currentfill}{rgb}{0.000000,0.000000,0.000000}%
\pgfsetfillcolor{currentfill}%
\pgfsetlinewidth{0.803000pt}%
\definecolor{currentstroke}{rgb}{0.000000,0.000000,0.000000}%
\pgfsetstrokecolor{currentstroke}%
\pgfsetdash{}{0pt}%
\pgfsys@defobject{currentmarker}{\pgfqpoint{0.000000in}{-0.048611in}}{\pgfqpoint{0.000000in}{0.000000in}}{%
\pgfpathmoveto{\pgfqpoint{0.000000in}{0.000000in}}%
\pgfpathlineto{\pgfqpoint{0.000000in}{-0.048611in}}%
\pgfusepath{stroke,fill}%
}%
\begin{pgfscope}%
\pgfsys@transformshift{4.499236in}{0.652650in}%
\pgfsys@useobject{currentmarker}{}%
\end{pgfscope}%
\end{pgfscope}%
\begin{pgfscope}%
\definecolor{textcolor}{rgb}{0.000000,0.000000,0.000000}%
\pgfsetstrokecolor{textcolor}%
\pgfsetfillcolor{textcolor}%
\pgftext[x=4.499236in,y=0.555428in,,top]{\color{textcolor}\rmfamily\fontsize{10.000000}{12.000000}\selectfont \(\displaystyle {\ensuremath{-}5}\)}%
\end{pgfscope}%
\begin{pgfscope}%
\pgfsetbuttcap%
\pgfsetroundjoin%
\definecolor{currentfill}{rgb}{0.000000,0.000000,0.000000}%
\pgfsetfillcolor{currentfill}%
\pgfsetlinewidth{0.803000pt}%
\definecolor{currentstroke}{rgb}{0.000000,0.000000,0.000000}%
\pgfsetstrokecolor{currentstroke}%
\pgfsetdash{}{0pt}%
\pgfsys@defobject{currentmarker}{\pgfqpoint{0.000000in}{-0.048611in}}{\pgfqpoint{0.000000in}{0.000000in}}{%
\pgfpathmoveto{\pgfqpoint{0.000000in}{0.000000in}}%
\pgfpathlineto{\pgfqpoint{0.000000in}{-0.048611in}}%
\pgfusepath{stroke,fill}%
}%
\begin{pgfscope}%
\pgfsys@transformshift{5.075695in}{0.652650in}%
\pgfsys@useobject{currentmarker}{}%
\end{pgfscope}%
\end{pgfscope}%
\begin{pgfscope}%
\definecolor{textcolor}{rgb}{0.000000,0.000000,0.000000}%
\pgfsetstrokecolor{textcolor}%
\pgfsetfillcolor{textcolor}%
\pgftext[x=5.075695in,y=0.555428in,,top]{\color{textcolor}\rmfamily\fontsize{10.000000}{12.000000}\selectfont \(\displaystyle {0}\)}%
\end{pgfscope}%
\begin{pgfscope}%
\pgfsetbuttcap%
\pgfsetroundjoin%
\definecolor{currentfill}{rgb}{0.000000,0.000000,0.000000}%
\pgfsetfillcolor{currentfill}%
\pgfsetlinewidth{0.803000pt}%
\definecolor{currentstroke}{rgb}{0.000000,0.000000,0.000000}%
\pgfsetstrokecolor{currentstroke}%
\pgfsetdash{}{0pt}%
\pgfsys@defobject{currentmarker}{\pgfqpoint{0.000000in}{-0.048611in}}{\pgfqpoint{0.000000in}{0.000000in}}{%
\pgfpathmoveto{\pgfqpoint{0.000000in}{0.000000in}}%
\pgfpathlineto{\pgfqpoint{0.000000in}{-0.048611in}}%
\pgfusepath{stroke,fill}%
}%
\begin{pgfscope}%
\pgfsys@transformshift{5.652153in}{0.652650in}%
\pgfsys@useobject{currentmarker}{}%
\end{pgfscope}%
\end{pgfscope}%
\begin{pgfscope}%
\definecolor{textcolor}{rgb}{0.000000,0.000000,0.000000}%
\pgfsetstrokecolor{textcolor}%
\pgfsetfillcolor{textcolor}%
\pgftext[x=5.652153in,y=0.555428in,,top]{\color{textcolor}\rmfamily\fontsize{10.000000}{12.000000}\selectfont \(\displaystyle {5}\)}%
\end{pgfscope}%
\begin{pgfscope}%
\pgfsetbuttcap%
\pgfsetroundjoin%
\definecolor{currentfill}{rgb}{0.000000,0.000000,0.000000}%
\pgfsetfillcolor{currentfill}%
\pgfsetlinewidth{0.803000pt}%
\definecolor{currentstroke}{rgb}{0.000000,0.000000,0.000000}%
\pgfsetstrokecolor{currentstroke}%
\pgfsetdash{}{0pt}%
\pgfsys@defobject{currentmarker}{\pgfqpoint{-0.048611in}{0.000000in}}{\pgfqpoint{-0.000000in}{0.000000in}}{%
\pgfpathmoveto{\pgfqpoint{-0.000000in}{0.000000in}}%
\pgfpathlineto{\pgfqpoint{-0.048611in}{0.000000in}}%
\pgfusepath{stroke,fill}%
}%
\begin{pgfscope}%
\pgfsys@transformshift{4.499236in}{0.652650in}%
\pgfsys@useobject{currentmarker}{}%
\end{pgfscope}%
\end{pgfscope}%
\begin{pgfscope}%
\pgfsetbuttcap%
\pgfsetroundjoin%
\definecolor{currentfill}{rgb}{0.000000,0.000000,0.000000}%
\pgfsetfillcolor{currentfill}%
\pgfsetlinewidth{0.803000pt}%
\definecolor{currentstroke}{rgb}{0.000000,0.000000,0.000000}%
\pgfsetstrokecolor{currentstroke}%
\pgfsetdash{}{0pt}%
\pgfsys@defobject{currentmarker}{\pgfqpoint{-0.048611in}{0.000000in}}{\pgfqpoint{-0.000000in}{0.000000in}}{%
\pgfpathmoveto{\pgfqpoint{-0.000000in}{0.000000in}}%
\pgfpathlineto{\pgfqpoint{-0.048611in}{0.000000in}}%
\pgfusepath{stroke,fill}%
}%
\begin{pgfscope}%
\pgfsys@transformshift{4.499236in}{0.940879in}%
\pgfsys@useobject{currentmarker}{}%
\end{pgfscope}%
\end{pgfscope}%
\begin{pgfscope}%
\pgfsetbuttcap%
\pgfsetroundjoin%
\definecolor{currentfill}{rgb}{0.000000,0.000000,0.000000}%
\pgfsetfillcolor{currentfill}%
\pgfsetlinewidth{0.803000pt}%
\definecolor{currentstroke}{rgb}{0.000000,0.000000,0.000000}%
\pgfsetstrokecolor{currentstroke}%
\pgfsetdash{}{0pt}%
\pgfsys@defobject{currentmarker}{\pgfqpoint{-0.048611in}{0.000000in}}{\pgfqpoint{-0.000000in}{0.000000in}}{%
\pgfpathmoveto{\pgfqpoint{-0.000000in}{0.000000in}}%
\pgfpathlineto{\pgfqpoint{-0.048611in}{0.000000in}}%
\pgfusepath{stroke,fill}%
}%
\begin{pgfscope}%
\pgfsys@transformshift{4.499236in}{1.229108in}%
\pgfsys@useobject{currentmarker}{}%
\end{pgfscope}%
\end{pgfscope}%
\begin{pgfscope}%
\pgfsetbuttcap%
\pgfsetroundjoin%
\definecolor{currentfill}{rgb}{0.000000,0.000000,0.000000}%
\pgfsetfillcolor{currentfill}%
\pgfsetlinewidth{0.803000pt}%
\definecolor{currentstroke}{rgb}{0.000000,0.000000,0.000000}%
\pgfsetstrokecolor{currentstroke}%
\pgfsetdash{}{0pt}%
\pgfsys@defobject{currentmarker}{\pgfqpoint{-0.048611in}{0.000000in}}{\pgfqpoint{-0.000000in}{0.000000in}}{%
\pgfpathmoveto{\pgfqpoint{-0.000000in}{0.000000in}}%
\pgfpathlineto{\pgfqpoint{-0.048611in}{0.000000in}}%
\pgfusepath{stroke,fill}%
}%
\begin{pgfscope}%
\pgfsys@transformshift{4.499236in}{1.517338in}%
\pgfsys@useobject{currentmarker}{}%
\end{pgfscope}%
\end{pgfscope}%
\begin{pgfscope}%
\pgfsetbuttcap%
\pgfsetroundjoin%
\definecolor{currentfill}{rgb}{0.000000,0.000000,0.000000}%
\pgfsetfillcolor{currentfill}%
\pgfsetlinewidth{0.803000pt}%
\definecolor{currentstroke}{rgb}{0.000000,0.000000,0.000000}%
\pgfsetstrokecolor{currentstroke}%
\pgfsetdash{}{0pt}%
\pgfsys@defobject{currentmarker}{\pgfqpoint{-0.048611in}{0.000000in}}{\pgfqpoint{-0.000000in}{0.000000in}}{%
\pgfpathmoveto{\pgfqpoint{-0.000000in}{0.000000in}}%
\pgfpathlineto{\pgfqpoint{-0.048611in}{0.000000in}}%
\pgfusepath{stroke,fill}%
}%
\begin{pgfscope}%
\pgfsys@transformshift{4.499236in}{1.805567in}%
\pgfsys@useobject{currentmarker}{}%
\end{pgfscope}%
\end{pgfscope}%
\begin{pgfscope}%
\pgfsetrectcap%
\pgfsetmiterjoin%
\pgfsetlinewidth{0.803000pt}%
\definecolor{currentstroke}{rgb}{0.000000,0.000000,0.000000}%
\pgfsetstrokecolor{currentstroke}%
\pgfsetdash{}{0pt}%
\pgfpathmoveto{\pgfqpoint{4.499236in}{0.652650in}}%
\pgfpathlineto{\pgfqpoint{4.499236in}{1.805567in}}%
\pgfusepath{stroke}%
\end{pgfscope}%
\begin{pgfscope}%
\pgfsetrectcap%
\pgfsetmiterjoin%
\pgfsetlinewidth{0.803000pt}%
\definecolor{currentstroke}{rgb}{0.000000,0.000000,0.000000}%
\pgfsetstrokecolor{currentstroke}%
\pgfsetdash{}{0pt}%
\pgfpathmoveto{\pgfqpoint{5.652153in}{0.652650in}}%
\pgfpathlineto{\pgfqpoint{5.652153in}{1.805567in}}%
\pgfusepath{stroke}%
\end{pgfscope}%
\begin{pgfscope}%
\pgfsetrectcap%
\pgfsetmiterjoin%
\pgfsetlinewidth{0.803000pt}%
\definecolor{currentstroke}{rgb}{0.000000,0.000000,0.000000}%
\pgfsetstrokecolor{currentstroke}%
\pgfsetdash{}{0pt}%
\pgfpathmoveto{\pgfqpoint{4.499236in}{0.652650in}}%
\pgfpathlineto{\pgfqpoint{5.652153in}{0.652650in}}%
\pgfusepath{stroke}%
\end{pgfscope}%
\begin{pgfscope}%
\pgfsetrectcap%
\pgfsetmiterjoin%
\pgfsetlinewidth{0.803000pt}%
\definecolor{currentstroke}{rgb}{0.000000,0.000000,0.000000}%
\pgfsetstrokecolor{currentstroke}%
\pgfsetdash{}{0pt}%
\pgfpathmoveto{\pgfqpoint{4.499236in}{1.805567in}}%
\pgfpathlineto{\pgfqpoint{5.652153in}{1.805567in}}%
\pgfusepath{stroke}%
\end{pgfscope}%
\begin{pgfscope}%
\definecolor{textcolor}{rgb}{0.000000,0.000000,0.000000}%
\pgfsetstrokecolor{textcolor}%
\pgfsetfillcolor{textcolor}%
\pgftext[x=5.075695in,y=1.888900in,,base]{\color{textcolor}\rmfamily\fontsize{12.000000}{14.400000}\selectfont n=6}%
\end{pgfscope}%
\begin{pgfscope}%
\definecolor{textcolor}{rgb}{0.000000,0.000000,0.000000}%
\pgfsetstrokecolor{textcolor}%
\pgfsetfillcolor{textcolor}%
\pgftext[x=3.000000in,y=3.920000in,,top]{\color{textcolor}\rmfamily\fontsize{12.000000}{14.400000}\selectfont \(\displaystyle \Phi_n(x_1, x_2)\)}%
\end{pgfscope}%
\begin{pgfscope}%
\definecolor{textcolor}{rgb}{0.000000,0.000000,0.000000}%
\pgfsetstrokecolor{textcolor}%
\pgfsetfillcolor{textcolor}%
\pgftext[x=3.000000in,y=0.040000in,,bottom]{\color{textcolor}\rmfamily\fontsize{12.000000}{14.400000}\selectfont \(\displaystyle x_1\)}%
\end{pgfscope}%
\begin{pgfscope}%
\definecolor{textcolor}{rgb}{0.000000,0.000000,0.000000}%
\pgfsetstrokecolor{textcolor}%
\pgfsetfillcolor{textcolor}%
\pgftext[x=0.235740in, y=1.920645in, left, base,rotate=90.000000]{\color{textcolor}\rmfamily\fontsize{12.000000}{14.400000}\selectfont \(\displaystyle x_2\)}%
\end{pgfscope}%
\end{pgfpicture}%
\makeatother%
\endgroup%

  \end{center}
  \caption{Wave functions for the numerical solution of the one-dimensional, time-independent Schr\"{o}dinger's describing two interacting electrons (Coulomb repulsion taken into account) in an infinitely-deep potential well of width $10a$ and discretization of $N=25$. Each plot represents two electrons, with the x and y axis corresponding to the position of each electron in the well: $\Phi_n(x_1,x_2)$.}
  \label{numerical-plot-3}
\end{figure}

\begin{table}
\begin{center}
\begin{tabular}{l|llllll}\hline
$n$    & $1$    & $2$     & $3$     & $4$      & $5$      & $6$      \\\hline
$E_n$  & $1$  & $1.36143053$  & $8.65467932$  & $1.93060894$  & $2.30840873$  & $2.60201186$ \\\hline

\end{tabular}
\end{center}
  \caption{The first six energy levels calculated for this problem, normalized to the first level. All of the solution energies were computed and these are the first six to come out of the sorting process. The associated plots are related to these values through their $n$ values, although the $n$ value doesn't correlate exactly to the actual $n$ for each electron.}
  \label{numerical-energies-2}
\end{table}

\section{Discussion}

\begin{itemize}
    \item The analytical solution obtains exact answers, but the problem has been greatly simplified. I don't believe exact answers can be easily obtained for more complicated.
    \item I am unsure of what causes the discretized solution to switch the orientation of the wave function, however, I do recognize that the inverse of the wavefunction is a valid solution to the problem that is presented. By modifying the value for N, I was able to get both orientations of the wavefunction. The likely culprit is the diagonalization algorithm on the Hermitian matrix settling on one solution versus another.
    \item In the two electron case, it makes sense that the electrons would each have their own energy level, which would lead to NxM different possibilities for the first N=M energy levels of each. However, I am unsure how to extract a specific combination of energy levels from the numerical data. When sorting the energy values from low to high, we are sorting the combination of N and M electrons for both electrons.
    \item In the two electron case, I expected there to be symmetry in both axes. Especially in the lowest energy case, where I expected a dome-shaped response. I am unsure why we see a linear gradient type of response. This response was evident as the discretization level was increased, so there is likely an error in the implementation or the Hamiltonian.
    \item When comparing the wavefunctions between the last two numerical solutions, we see that the Coulomb repulsion between the two electrons is highest at $x_1 = x_2$. This manifests as a diagonal line along $x_1 = x_2$ which splits the original wavefunction. Ignoring the sign change between the two wavefunctions, the original wavefunction shape is preserved except for the diagonal where the Coulomb attraction is highest.
\end{itemize}

\newpage
\section{Code Listings and Data}

\subsection{Python Code Listing}
\label{code-listing-python}
The following is the code written in Python to generate the solutions and plots used in this report.
\lstinputlisting[language=Python]{plots.py}

\newpage
\section{References}

These aren't citing anything, but they were useful in helping me figure out this assignment.

\begin{itemize}
    \item\url{https://medium.com/modern-physics/finite-difference-solution-of-the-schrodinger-equation-c49039d161a8}
    \item\url{https://www.12000.org/my_notes/mma_matlab_control/KERNEL/KEse82.htm}
    \item\url{https://digitalcommons.calpoly.edu/cgi/viewcontent.cgi?article=1119&context=physsp}
\end{itemize}

\end{document}

